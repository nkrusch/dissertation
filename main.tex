%! suppress = TooLargeSection
%! suppress = MissingLabel
\documentclass[letterpaper,11pt,draft]{report}
\usepackage{.latex/packages}
\usepackage{latex/packages}
% Syntax colors
\definecolor{codecomments}{HTML}{455A64}
\definecolor{codekeywords}{HTML}{3D5AFE}
\definecolor{codestrings}{HTML}{00838F}
\input{latex/unicodes}
% custom thematic icons
\newcommand{\iconOPT}{\textnormal{\scalebox{.65}{\faWaveSquare}}} % optimization
\newcommand{\iconSEC}{\textnormal{\scalebox{.65}{\faShield*}}} % security
\newcommand{\iconFM}{\textnormal{\scalebox{.65}{\faCheck}}} % formal methods, verification
\newcommand{\iconSPA}{\textnormal{\scalebox{.65}{\faCode}}} % static program analysis

\newcommand{\types}{\textnormal{\raisebox{0.15\height}{\scalebox{.65}{\faStarOfLife}}}}
\newcommand{\logics}{\textnormal{\raisebox{0.25\height}{\scalebox{.65}{\faChessKing}}}}
\newcommand{\dataflow}{\textnormal{\raisebox{0.15\height}{\scalebox{.65}{\faShare*}}}}
\newcommand{\syntax}{\textnormal{\raisebox{0.15\height}{\scalebox{.65}{\faFont}}}}
\newcommand{\lalg}{\textnormal{\raisebox{0.15\height}{\scalebox{.65}{\faSuperscript}}}}
\newcommand{\recs}{\textnormal{\raisebox{0.15\height}{\scalebox{.65}{\faRedo*}}}}

% small circled characters
\newcommand{\circled}[2][black]{\hspace{-.5ex}\raisebox{-0.2\height}{\scalebox{.6}{%
    \tikz{\node[font={\selectfont\color{#1}},
        circle,inner sep=2pt,outer sep=.4pt,fill=white,
        draw=#1,solid,thick]{\textbf{#2}};}}}}

\newcommand{\circledb}[2][black]{\hspace{-.5ex}\raisebox{-0.2\height}{\scalebox{.6}{%
    \tikz{\node[font={\selectfont\color{#1}},
        circle,inner sep=1.5pt,outer sep=.4pt,fill=white,
        draw=#1,solid,line width=1.5pt]{\textbf{#2}};}}}}

% small color square
\newcommand{\langclr}[1]{%
    \tikz{\node[rectangle,minimum size=2mm,rounded corners=2pt,fill=#1]{};}}

\input{latex/rocqlang}
\documentclass[crop,border={-20pt 10pt 0pt 0pt}]{standalone}
\usepackage{tikz}
\usepackage{fontawesome5}
\usepackage{fontspec}
\usepackage{color}
\usetikzlibrary{
    calc,
    arrows.meta,
    decorations.pathmorphing,
    positioning,
    shapes.geometric
}
\input{../latex/colors}
\input{../latex/symbols}
\setmainfont{TeXGyreTermes}[
    Path = ../fonts/,
    UprightFont = *-Regular,
    BoldFont = *-Bold,
    ItalicFont = *-Italic,
    BoldItalicFont = *-BoldItalic]

% drawings
\tikzstyle{thread} = [
  thick,decorate,
  decoration={snake,amplitude=.5mm,segment length=3mm,post length=1.8mm},
  arrows={-Latex[width=5pt, length=5pt]}
]

\begin{document}
 
\tikzstyle{arrow} = [thick,arrows={-Latex[width=5pt, length=5pt]}]
\tikzstyle{comment} = [densely dashed, gray!50!black, thick]
\tikzstyle{comme} = [font={\selectfont\itshape\color{gray!50!black}},text width=3cm]
\tikzstyle{label} = [midway]

\tikzstyle{phase}  = [
	font={\selectfont\bfseries},
	draw=black,
	thick,
	fill=none,
	text centered,
	rounded corners=3pt,
	minimum height=1.3cm,
	text width=2cm]


\begin{tikzpicture}[align=center]
%\draw[gray!20!white,step=.25] (-1.5,-2.5) grid (16,3.5);

\path  (0,  0.0) node[phase] (0B)  {Dafny source file}
	  +(0, -1.5) node[comme] (0C)  {typically *.dfy}
   	 ++(2.5,2.5) node[phase] (1B)  {Dafny AST}
	 ++(3, -2.5) node[phase] (2B)  {Boogie}	 
	 ++(0, -1.8) node[comme] (2C)  {intermediate verification language}
	 ++(3,  4.3) node[phase] (3B)  {SMT formulas}
	 ++(3, -2.5) node[phase] (4B)  {Verification result} 
	  +(0, -1.9) node[comme] (4C)  {Correctness, error model, or timeout}
	 ++(0, 2) node[phase,draw=none] (5B)  {Output};

\path[every node/.style={font=\small\color{sorange}}]
    (0B) edge[arrow, bend right=-30] node [left,label,xshift=-5pt,yshift=5pt] {Parsing} (1B)
    (1B) edge[arrow, bend right=30]  node [left,label,yshift=-15pt] {Translation} (2B)
    (2B) edge[arrow, bend right=-30] node [left,label,xshift=0pt,yshift=20pt] {Verification\\condition\\generator} (3B)
    (3B) edge[arrow, bend right=30]  node [left,label,yshift=-15pt] {SMT\\solver} (4B)
    (4B) edge[arrow]  node {} (5B);

\draw [comment] (0C) -- (0B) node {}; 
\draw [comment] (2C) -- (2B) node {}; 
\draw [comment] (4C) -- (4B) node {}; 

\end{tikzpicture}
\end{document}

\lstdefinelanguage{OpenMP}{
    mathescape=true,
    morekeywords={for,int},
    moredirectives={pragma,omp,parallel,teams,schedule,single,for,%
    distribute,barrier,critical,atomic,private,shared,
    nowait,target,loop,order,unroll,full,partial
    },
    moredelim=*[directive]\#,%
    morecomment=[l]{//},
    morecomment=[s]{/*}{*/},
    morestring=[b]",%
    breaklines=true,
    tabsize=2,
    sensitive=true,
    breaklines=true,
    columns=fullflexible,
    prebreak={\space\textbackslash},
    postbreak={},
}[keywords,comments,strings,directives]

\lstdefinestyle{inlineomp}{
    morekeywords={pragma,omp,parallel,teams,schedule,%
    single,for,distribute,barrier,critical,atomic,private,
    shared,nowait,target,loop,order,unroll,full,partial
    },
    keywordstyle={\ttfamily\color{black}}}
\newcommand{\pr}{\lstinline[mathescape,basicstyle=\ttfamily,escapeinside={(*}{*)}]}
\newcommand{\prc}{\lstinline[language=C,basicstyle=\ttfamily,keywordstyle=\mdseries]}
\newcommand{\omp}[1]{\lstinline[language=InlineOmp]|#1|\index{OpenMP directives!\texttt{#1}}}
\newcommand{\prm}[1]{\text{\pr|#1|}}

\lstset{
    columns=[l]flexible,
    basicstyle=\small\ttfamily\linespread{4},
    identifierstyle=\color{codeid},
    commentstyle=\color{codecomments},
    keywordstyle=\color{codekeywords}\bfseries,
    ndkeywordstyle=\color{codekeywords}\bfseries,
    stringstyle=\color{codestrings},
    abovecaptionskip=-30pt,
    aboveskip=2em,
    belowskip=2em,
    breakatwhitespace=false,
    breaklines=true,
    captionpos=b,
    escapechar=!,
    escapeinside=||,
    extendedchars=true,
    float=H,
    frame=tb,
    keepspaces=true,
    mathescape=false,
    numbers=left,
    prebreak=\raisebox{0ex}[0ex][0ex]{↲},
    sensitive=true,
    showspaces=false,
    showstringspaces=false,
    showtabs=false,
    tabsize=2,
    upquote=true,
}

\lstdefinestyle{Plain}{
    language=C,
    identifierstyle=\color{black},
    commentstyle=\color{black},
    keywordstyle=\color{black}\bfseries,
    ndkeywordstyle=\color{black}\bfseries,
    stringstyle=\color{black}\ttfamily,
}

\lstdefinestyle{Dafny}{
    language=dafny,
    basicstyle=\small\ttfamily\linespread{4},
    identifierstyle=\color{codeid},
    keywordstyle={\color{codekeywords}\ttfamily\bfseries},
    ndkeywordstyle={\color{codekeywords}\ttfamily\bfseries},
    commentstyle=\color{codecomments},
    stringstyle=\color{codestrings}
}

\lstdefinestyle{openmp}{
    language=OpenMP,
    directivestyle={\color{ompdirective}\ttfamily\bfseries\itshape}
}

\lstnewenvironment{console}{\lstset{
    backgroundcolor=\color{consolebg},
    columns=fullflexible,
	basicstyle=\ttfamily\footnotesize\mdseries,
    showstringspaces=false,
    tabsize=4,
    keepspaces=true,
    showtabs=true,
    showspaces=false,
    framesep=2pt,
    frame=single,
    framerule=.5pt,
    framexleftmargin=5pt,
    framexrightmargin=5pt,
    xleftmargin=5.5pt,
    xrightmargin=5.5pt,
    breaklines=true,
    prebreak=\raisebox{0ex}[0ex][0ex]{↲},
    escapeinside={||},
    numbers=none,
    emph={}
    aboveskip=0em,
    belowskip=0em,
}}{}
%! suppress = NonBreakingSpace
%-------------------------------------------------------------------------------
% Custom macros
%-------------------------------------------------------------------------------

% Manuscript headers
\newcommand\CTNT{Clément Aubert, Thomas Rubiano, Neea Rusch, Thomas Seiller}
\newcommand\ainfo[2]{\noindent{#1}\newline\noindent{#2}\newpage}
\newcommand\ainfoX[3]{\noindent{#1}\newline\noindent{#2}\vspace{1em}\newline{#3}\newpage}

% Document commands
\newcommand{\aref}[1]{\hyperref[#1]{\ref*{#1}}}
\newcommand{\swlink}[2]{\texttt{\footnotesize{\href{#1}{#2}}}}
\newcommand{\nonterm}[1]{#1} % ott
\newcommand*\lstinputpath[1]{\lstset{inputpath=#1}}
\newcommand{\smtabularnote}[1]{\tabularnote{\small{#1}}}
\newcommand{\ompi}[1]{\index{OpenMP directives!{#1}}}
\newcommand{\ccxi}[1]{\index{complexity classes!\textsc{#1}}}
\newcommand{\ccx}[1]{\textsc{#1}\ccxi{#1}}
\newcommand{\ndx}[1]{#1\index{#1}}

% Inline codes
\newcommand{\pr}{\lstinline[mathescape,basicstyle=\normalfont\ttfamily,escapeinside={(*}{*)},prebreak={}]}
\newcommand{\prm}[1]{\text{\pr|#1|}}
\newcommand{\prc}{\lstinline[mathescape=true]}
\newcommand{\omp}[1]{\lstinline[language=OpenMP,basicstyle=\ttfamily\color{black},keywordstyle=\ttfamily\color{black}]|#1|\ompi{#1}}

% apa-style citation fix
\renewcommand{\cite}[2][]{\autocite[#1]{#2}}

% continuous appendix
\newcommand{\continousappendix}{
 \renewcommand{\thechapter}{\arabic{chapter}}
 \setcounter{chapter}{6}} % chapters are fixed anyway

% paper category icons
\newcommand{\pageIcon}[2]{
\raisebox{-0.3\height}{\begin{tikzpicture}
\node[circle,inner sep=2pt,outer sep=.4pt,fill=white,draw=black,solid,thick]{#1};
\end{tikzpicture}}\hspace{1em}\textit{Implicit computational complexity \& #2}\par}
\newcommand{\pageIconAnalysis}{\pageIcon{\iconSPA}{static analysis}}
\newcommand{\pageIconFm}{\pageIcon{\iconFM}{formal methods}}
\newcommand{\pageIconOpt}{\pageIcon{\iconOPT}{program optimization}}
\newcommand{\pageIconSecurity}{\pageIcon{\iconSEC}{security}}

% boxed text
\makeatletter
\renewcommand{\boxed}[1]{\text{\fboxsep=.2em\fbox{\m@th$\displaystyle#1$}}}
\makeatother

% text with background color
\newcommand{\hilight}[2]{\makebox[5pt][l]{\color{#1!20!white}\rule[-4pt]{#2}{14pt}}}

% Fix \sc undefined
\providecommand{\sc}{}
\renewcommand{\sc}[1]{#1}

% tabularx custom maximally-wide column
\newcolumntype{C}{>{\centering\arraybackslash}X}
\newcolumntype{R}{>{\raggedleft\arraybackslash}X}
\newcolumntype{Z}{>{\raggedleft\arraybackslash}X}

% Math operators etc.
\newcommand{\zmat}{\mathbf{0}}
\newcommand{\umat}{\mathbf{1}}
\newcommand\hypo{\Hypo}
%\newcommand\infer{\Infer}
\DeclareMathOperator{\In}{In}
\DeclareMathOperator{\Out}{Out}
\DeclareMathOperator{\PrD}{PrD}
\DeclareMathOperator{\id}{id}
\DeclareMathOperator{\Id}{Id}
\DeclareMathOperator{\var}{var}
\DeclareMathOperator{\Card}{Card}
\DeclareMathOperator{\Occ}{Occ}
\DeclareMathOperator{\poly}{poly}
\renewcommand{\gets}{=} % algorithms

\DeclarePairedDelimiter{\sem}{\llbracket}{\rrbracket} % Semantics of program.
\newcommand{\BNF}{\enspace \ensuremath{\Vert} \enspace} % BNF separator
\newcommand{\mat}[1]{\left(\begin{smallmatrix}#1\end{smallmatrix}\right)} % Shorthand for matrices
\ebproofnewstyle{small}{separation = 1em, rule margin = .5ex}

\newtheorem{thm}{Theorem}
\newtheorem{corollary}[thm]{Corollary}
\newtheorem{lemma}[thm]{Lemma}
\newtheorem{definition}{Definition}
\newtheorem{conjecture}[thm]{Conjecture}
\newtheorem{proposition}{Proposition}[section]
\newtheorem{claim}{Claim}
\newtheorem{remark}{Remark}
\newtheorem{notation}{Notation}
\newtheorem{example}{Example}
\newtheorem{examples}{Examples}
\newtheorem{remarks}{Remarks}
\newtheorem{facts}{Facts}
\newtheorem{fact}{Fact}

% Custom terms
\newcommand\mwp{{mwp}\xspace}
\newcommand{\mwpsc}{\textnormal{\textsc{mwp}}\xspace}
\newcommand{\vdashJK}{\vdash} %_{\textnormal{\textsc{jk}}}}
\newcommand{\DFG}{\textsc{dfg}\xspace}
\newcommand{\dfg}[1]{\mathbb{M}(\mathtt{#1})}
\newcommand{\dfgtilde}[1]{\mathbb{\tilde{M}}(\mathtt{#1})}
\newcommand{\scc}{\textsc{scc}\xspace}
\newcommand{\sccs}{\textnormal{{\scshape SCC}s\xspace}}
\newcommand{\condcorr}[1]{\mathrm{Corr}^{\mathtt{if}}(#1)}
\newcommand{\whilecorr}[1]{\mathrm{Corr}^{\mathtt{while}}(#1)}
\newcommand{\corr}[1]{\mathrm{Cr}(#1)}
\newcommand{\corrc}[1]{\mathrm{Corr}(#1)}
\newcommand{\Var}{\mathrm{Vars}}
\newcommand{\lvl}[1]{\ell(#1)}
\newcommand{\vi}{\textnormal{\scalebox{.65}{\faTint}}}
\newcommand{\nv}{\cdot} % Command for "no leak" / "no violation.
\newcommand{\lname}{\(\top^{\ast}_{\textsc{ni}}\)\@\xspace} % change it if you don't like it
\newcommand{\tool}{\textsc{tyni}\@\xspace}
\newcommand{\nupce}[1]{\buildrel #1 \over \nsim}% Not up to c equilavence

\newcommand{\impl}{$\text{mwp}_\ell$\xspace}
\newcommand{\impf}{$\text{mwp}_f$\xspace}
\newcommand{\explain}{LucidLoop}
\newcommand{\exname}{{\explain}\xspace}
\newcommand{\sfull}{\scalebox{.75}{\faIcon{circle}}\xspace}
\newcommand{\spart}{\scalebox{.75}{\faIcon{adjust}}\xspace}
\newcommand{\snone}{\scalebox{.75}{\faIcon[regular]{circle}}\xspace}
\newcommand{\qsymb}{\faQuestionCircle[regular]}
\newcommand{\qtext}{\scriptsize\qsymb\normalsize}
\newcommand{\myqm}{\scalebox{.75}{\qsymb}}
\newcommand{\myok}{\scalebox{.8}{$\checkmark$}}

\newcommand{\SFM}{{\sc sfm}\xspace}
\newcommand{\SFMs}{{\sc sfm}s\xspace}
\newcommand{\sfm}[1]{\mathbb{M}(#1)}
\newcommand{\sfmb}[1]{\ensuremath{\mathbb{M}^{\mathtt{e}}(#1)}}
\renewcommand{\qedsymbol}{$\square$}
\newcommand{\replabel}{\label} % will be redefined in restatements
\newcommand{\SC}{\mathrm{SC}}
\newcommand{\SCset}{\mathit{SC}}
\newcommand{\LH}{\mathrm{LH}}
\newcommand{\HMO}{\mathrm{HMO}}
\newcommand{\scl}[1]{\ensuremath{\mathit{#1}}}
\newcommand{\SSG}{\mathrm{SSG}}
\newcommand{\orth}{\mathbin{\bot}} % orthogonal
\newcommand{\upce}[1]{\buildrel #1 \over\sim}% Up to c equilavence
\makeatletter
\renewcommand{\nsim}{\mathrel{\mathpalette\n@sim\relax}}
\newcommand{\n@sim}[2]{%
\ooalign{%
    $\m@th#1\sim$\cr
    \hidewidth$\m@th#1\rotatebox[origin=c]{50}{$#1-$}$\hidewidth\cr
}%
}
\makeatother

% Comma after eg and ie
\newcommand*{\eg}{e.g.\@,\xspace}
\newcommand*{\Eg}{E.g.\@,\xspace}
\newcommand*{\cf}{cf.\@\xspace}
\newcommand*{\ie}{i.e.\@,\xspace}
\newcommand*{\aka}{a.k.a.\@\xspace}
\newcommand*{\Ie}{I.e.,\@\xspace}
\newcommand{\stt}{s.t.\@\xspace}
\newcommand*{\resp}{resp.\@\xspace}
\newcommand*{\wrt}{w.r.t.\@\xspace}
\newcommand*{\wlg}{w.l.o.g.\@\xspace}
\renewcommand{\aka}{a.k.a.\@\xspace}
\newcommand{\stsup}{$^\text{st}$\@\xspace}
\newcommand{\ndsup}{$^\text{nd}$\@\xspace}
\newcommand{\thsup}{$^\text{th}$\@\xspace}
\makeatletter
\newcommand*{\etc}{%
    \@ifnextchar{.}%
{etc}%
{etc.\@\xspace}%
}
\makeatother

% HELPERS FOR TERM AND SYMBOLS INDICES
% shorthand for symbols
\newcommand\symbo[1]{\gls{symb:#1}}
\setsepchar{,}
% Includepdf helpers : terms
\makeatletter
\newcommand\addtoindex[2]{\ifnum\AM@page=#2\relax\index{#1}\fi}
\makeatother
\newcommand{\addtoindexm}[2]{\readlist\ipages{#2}\foreachitem\word\in\ipages{\addtoindex{#1}{\word}}}
% Includepdf helpers : symbols
\makeatletter
\newcommand\addtosymbols[2]{\ifnum\AM@page=#2\relax\gls{symb:#1}\fi}
\makeatother
\newcommand{\addtosymbolsm}[2]{\readlist\ipages{#2}\foreachitem\word\in\ipages{\addtosymbols{#1}{\word}}}
% All sections begin on a new page
\AddToHook{cmd/chapter/before}{\clearpage}
\renewcommand\thechapter{\Roman{chapter}}
\renewcommand\thesection{\Alph{section}}
\renewcommand{\cftchapnumwidth}{30pt}
\renewcommand{\cftsecnumwidth}{30pt}
\setlength{\cftbeforechapskip}{12pt} % Space before chapters
\setlength{\cftbeforesecskip}{10pt} % Space before sections
\setlength{\cftbeforesubsecskip}{4pt} % Space before subsections
\setlength{\cftbeforesubsubsecskip}{2pt} % Space before subsubsections
\setlength{\cftsecindent}{0pt} % Remove indent for section
\setlength{\cftsubsecindent}{0pt} % Remove indent for subsection
\setcounter{tocdepth}{2}
\addtocontents{toc}{\protect\thispagestyle{empty}}
\addtokomafont{disposition}{\rmfamily}
\addtokomafont{section}{\rmfamily}
\addtokomafont{subsection}{\rmfamily\rule{0pt}{30pt}}
\addtokomafont{subsubsection}{\rmfamily}
\renewcaptionname{english}{\contentsname}{Table of Contents}

% autorefs
\AtBeginDocument{%
\renewcommand{\chapterautorefname}{Chap.}
\renewcommand{\sectionautorefname}{Sect.}
\renewcommand{\subsectionautorefname}{Sect.}
\renewcommand{\figureautorefname}{Fig.}
\newcommand{\definitionautorefname}{Def.}}
\newcommand{\algorithmautorefname}{Algo.}
\newcommand{\notationautorefname}{Notation}

% Comma after eg and ie
\newcommand*{\eg}{e.g.\@,\xspace}
\newcommand*{\Eg}{E.g.\@,\xspace}
\newcommand*{\cf}{cf.\@\xspace}
\newcommand*{\ie}{i.e.\@,\xspace}
\newcommand*{\aka}{a.k.a.\@\xspace}
\newcommand*{\Ie}{I.e.,\@\xspace}
% \newcommand{\st}{s.t.\@\xspace}
\newcommand*{\resp}{resp.\@\xspace}
\newcommand*{\wrt}{w.r.t.\@\xspace}
\newcommand*{\wlg}{w.l.o.g.\@\xspace}
\renewcommand{\aka}{a.k.a.\@\xspace}

\makeatletter
\newcommand*{\etc}{%
    \@ifnextchar{.}%
{etc}%
{etc.\@\xspace}%
}
\makeatother


\makeindex
\makenoidxglossaries

\newacronym{ast}{AST}{abstract syntax tree}
\newacronym{api}{API}{application programming interface}
\newacronym{icc}{ICC}{Implicit Computational Complexity}
\newacronym{c99}{C99}{C language standard ISO/IEC 9899:1999}
\newacronym{sc}{SC}{security class}
\newacronym{sgg}{SGG}{security semi-group}
\newacronym{dfg}{DFG}{data flow graph}
\newacronym{sfm}{SFM}{security-flow matrix}
\newacronym{smt}{SMT}{Satisfiability Modulo Theories}
\newacronym{sql}{SQL}{Structured Query Language}
\newacronym{openmp}{OpenMP}{Open Multi-Processing, parallel programming API}
\newacronym{gcc}{GCC}{The GNU Compiler Collection, compiler infrastructure}
\newacronym{cli}{CLI}{command line interface}

\glsaddall
\inputott{latex/sts}

% resource paths
\graphicspath{{./pdf/}}
\lstset{inputpath={./code}}

\addbibresource{bib.bib}

%%%%%%%%%%%%%%%%%%%%%%%%%%%%%%%%%%%%%%%%%%%%%%%%%%%%%%%%%%%%%%%%%%%%%%%%%%%%%%%%
% GUIDES
% https://guides.augusta.edu/graduateschool/etd
% https://www.augusta.edu/gradschool/documents/thesis-dissertation-preparation-booklet.pdf
%%%%%%%%%%%%%%%%%%%%%%%%%%%%%%%%%%%%%%%%%%%%%%%%%%%%%%%%%%%%%%%%%%%%%%%%%%%%%%%%
% META
\newcommand\EDSTITLE{Some Title}
\newcommand\EDSAUTHOR{Neea Rusch}
\newcommand\EDSADVISOR{Cl{\'{e}}ment Aubert}
\title{\EDSTITLE}
\author{\EDSAUTHOR}
\date{\today}
\hypersetup{
bookmarks=true,
unicode=false,
pdftoolbar=true,
pdfmenubar=true,
pdffitwindow=false,
pdfstartview={FitH},
pdfnewwindow=true,
ocgcolorlinks,
pdftitle={\EDSTITLE},
pdfauthor={\EDSAUTHOR},
pdfkeywords={{implicit complexity}, {program analysis}, {software verification}},
pdfsubject={PhD Dissertation}}
%%%%%%%%%%%%%%%%%%%%%%%%%%%%%%%%%%%%%%%%%%%%%%%%%%%%%%%%%%%%%%%%%%%%%%%%%%%%%%%%
% TEMPORARY: if draft
\usepackage{everypage,pdfpages,eso-pic}
\usepackage[yyyymmdd,24hr]{datetime}
\renewcommand{\dateseparator}{-}
\def\PageTopMargin{1in}
\def\PageLeftMargin{2.2in}
\newcommand\atxy[3]{%
\AddEverypageHook{\smash{\hspace*{\dimexpr-\PageLeftMargin-\hoffset+#1\relax}%
\raisebox{\dimexpr\PageTopMargin+\voffset-#2\relax}{#3}}}}
\atxy{\dimexpr\paperwidth-1.5in}{0.1in}{%
\raisebox{-\height}{\small\texttt{\color{red}{%
DRAFT compiled: {\yyyymmdddate\today} \currenttime}}}}
%%%%%%%%%%%%%%%%%%%%%%%%%%%%%%%%%%%%%%%%%%%%%%%%%%%%%%%%%%%%%%%%%%%%%%%%%%%%%%%%

\newcounter{insertpages}

\begin{document}
\maketitle\clearpage


Acknowledgements – include a detailed summary of the work performed by other
authors on published or accepted manuscripts used in the thesis/dissertation,
if applicable.

Should additional authors (other than the candidate and the advisor) be listed on
one or more of the dissertation manuscripts, the candidate must provide a
detailed summary of the work performed by these other authors.
This summary may be provided in the “Acknowledgments” section of the dissertation.
Furthermore, a statement must be provided in the dissertation in which the
additional authors agree (by their signatures) with the candidate’s assessment of
their contribution to the manuscripts.\clearpage
The abstract must not exceed 350 words\clearpage
\tableofcontents\clearpage
\listoftables\clearpage
\listoffigures\clearpage

%----------------------------------------------------------------------------------------
%	1. INTRODUCTION
%----------------------------------------------------------------------------------------
\setcounter{chapter}{0}
\chapter{Introduction}\label{ch:introduction}
\clearpage\section{Problem statement}\label{sec:intro}

Statement of the problem and specific aims of the overall project.


\clearpage\section{Preliminaries}\label{sec:pre}

Literature review and discussion of the rationale of the project.

It is expected that the literature review will be more comprehensive than
those presented in the included publications.



%----------------------------------------------------------------------------------------
%	2. PUBLISHED MANUSCRIPTS
%----------------------------------------------------------------------------------------
\chapter{Published Manuscripts}\label{ch:published-manuscripts}\clearpage
% The candidate must be first author on all manuscripts bundled into the dissertation.

\section*{pymwp: A Static Analyzer Determining Polynomial Growth Bounds}
\addcontentsline{toc}{section}{pymwp: A Static Analyzer Determining Polynomial Growth Bounds}
\label{sec:atva}
\ainfoX{Cl{\'{e}}ment Aubert, Thomas Rubiano, Neea Rusch, Thomas Seiller}
{International Symposium on Automated Technology for Verification and Analysis, 2023}
{\noindent Software artifact: \href{https://doi.org/10.5281/zenodo.7908484}{10.5281/zenodo.7908484}
\newline\noindent Tool user guide in \autoref{sec:tool-guide}}

\includepdf[pages=-,pagecommand={\stepcounter{insertpages}},addtolist={
5,table,Comparison of obtained resource bounds.,tab:compare,
10,table,Benchmark results.,tab:eval,
10,table,Examples of obtained bounds.,tab:bounds}]{papers/atva/main.pdf}
\setcounter{insertpages}{0}

%----------------------------------------------------------------------------------------
%	3. UNPUBLISHED RESEARCH
%----------------------------------------------------------------------------------------
\chapter{Unpublished Research}\label{ch:unpublished-research}\clearpage
% This can include manuscripts, in the journal format, that have been submitted for
% publication that are under review at the time of dissertation submission. Manuscripts
% that have been rejected or that have not been submitted (in preparation) cannot be
% included in the journal format (i.e. must conform to the traditional dissertation
% format). For manuscripts that have been submitted and are under review, the first
% authorship of the candidate must be maintained when the manuscript is published.

\section*{An Information Flow Calculus for Non-Interference}
\addcontentsline{toc}{section}{An Information Flow Calculus for Non-Interference}
\label{sec:ni-analysis}

\ainfoX{Cl{\'{e}}ment Aubert, Neea Rusch}
{Presented at the ..., 2024}
{\abspage{Sensitive data exposure is persistently ranked among the top-ten web application security risks, thus every software developer should actively combat data exposure vulnerabilities.
Information flow controls offer mechanisms to enforce data confidentiality.
Unfortunately, strict controls are too restrictive for real applications and innovation is needed to obtain practical solutions.
We present a brave new idea: an information flow calculus for non-interference.
Our formulation enforces that command composition does not create nor erase non-interference violations, and the sound calculus pinpoints precisely where violations occur.
The calculus can be implemented as an automatic, compositional, and annotation-free static security analyzer to obtain confidentiality guarantees in practice.}}

%\includepdf[pages={1-},pagecommand={\thispagestyle{empty}\stepcounter{insertpages}
%\ifnum\value{insertpages}=1\addcontentsline{toc}{section}{
%An Information Flow Calculus for Non-Interference}\fi
%}]{papers/plas/main.pdf}\setcounter{insertpages}{0}

\section*{Certifying Complexity Analysis}
\addcontentsline{toc}{section}{Certifying Complexity Analysis}
\label{sec:certifying-complexity-analysis}

\ainfoX{Cl{\'{e}}ment Aubert, Thomas Rubiano, Neea Rusch, Thomas Seiller}
{Presented at the Ninth International Workshop on Coq for Programming Languages, 2023}
{\abspage{This work drafts a strategy that leverages the field of Implicit Computational Complexity to certify resource usage in imperative programs.
This original approach sidesteps some of the most common--and difficult--obstacles \enquote{traditional} complexity theory face when implemented in Coq.}}
\clearpage\input{complete/coqpl}

%----------------------------------------------------------------------------------------
%	4. DISCUSSION
%----------------------------------------------------------------------------------------
\chapter{Discussion}\label{ch:discussion}
\input{text/_discussion}

%----------------------------------------------------------------------------------------
%	5. SUMMARY
%----------------------------------------------------------------------------------------
\chapter{Summary}\label{ch:summary}
\input{text/_summary}

%----------------------------------------------------------------------------------------
%	6. REFERENCES
%----------------------------------------------------------------------------------------
\chapter{References}\label{ch:references}
% All literature cited in the dissertation will be included here. Style should conform
% to selected style manual. References included in manuscripts must be renumbered and
% reformatted (if needed) and included in this section. Literature Cited sections within
% the manuscripts should be omitted.
\printbibliography[heading=none,sorting=none]

%----------------------------------------------------------------------------------------
%	7. APPENDICES
%----------------------------------------------------------------------------------------
\chapter{Appendices}\label{ch:appendices}\clearpage

\section*{mwp-Analysis Improvement and Implementation: Realizing Implicit Computational Complexity}\label{sec:fscd}
\addcontentsline{toc}{section}{mwp-Analysis Improvement and Implementation: Realizing Implicit Computational Complexity}
\ainfo{Cl{\'{e}}ment Aubert, Thomas Rubiano, Neea Rusch, Thomas Seiller}{International Conference on Formal Structures for Computation and Deduction, 2022}
\includepdf[pages={1-},pagecommand={\thispagestyle{empty}\stepcounter{insertpages}}]
{res/pubs_fscd.2022.pdf}\setcounter{insertpages}{0}

\section*{Distributing and Parallelizing Non-canonical Loops}\label{sec:vmcai}
\addcontentsline{toc}{section}{Distributing and Parallelizing Non-canonical Loops}
\ainfo{Cl{\'{e}}ment Aubert, Thomas Rubiano, Neea Rusch, Thomas Seiller}
{International Conference on Verification, Model Checking, and Abstract Interpretation, 2023}
\includepdf[pages={1-},pagecommand={\thispagestyle{empty}\stepcounter{insertpages}}]
{res/pubs_vmcai.2023.pdf}\setcounter{insertpages}{0}

\section*{Tool User Guide for \enquote{pymwp: A Static Analyzer Determining Polynomial Growth Bounds}}\label{sec:tool-guide}
\addcontentsline{toc}{section}{Tool User Guide for \enquote{pymwp: A Static Analyzer Determining Polynomial Growth Bounds}}
\ainfo{Cl{\'{e}}ment Aubert, Thomas Rubiano, Neea Rusch, Thomas Seiller}
{A companion tool user guide}
\includepdf[pages={1-},pagecommand={\thispagestyle{empty}\stepcounter{insertpages}}]
{res/pubs_pymwp_guide.pdf}\setcounter{insertpages}{0}

\end{document}