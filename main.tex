% !TeX program = xelatex
% !BIB program = biber

%%%%%%%%%%%%%%%%%%%%%%%%%%%%%%%%%%%%%%%%%%%%%%%%%%%%%%%
% Welcome! Do not anything until "%%% To be filled %" %
%%%%%%%%%%%%%%%%%%%%%%%%%%%%%%%%%%%%%%%%%%%%%%%%%%%%%%%

% Options for packages loaded elsewhere
\PassOptionsToPackage{unicode}{hyperref}
\PassOptionsToPackage{hyphens}{url}
\documentclass[
  12pt,
  letterpaper,
  oneside,
  BCOR = 0 pt,
  singlespacing = true,
  numbers = noenddot]{scrbook}
\usepackage{xcolor}
\usepackage[top=1in, left=1.5in, right=1in, bottom=1in, truedimen,
twoside=false, bindingoffset=0pt, nohead]{geometry}
\usepackage{amsmath,amssymb}
\setcounter{secnumdepth}{5}
\usepackage{iftex}
\ifPDFTeX
  \usepackage[T1]{fontenc}
  \usepackage[utf8]{inputenc}
  \usepackage{textcomp} % provide euro and other symbols
\else % if luatex or xetex
  \usepackage{unicode-math} % this also loads fontspec
  \defaultfontfeatures{Scale=MatchLowercase}
  \defaultfontfeatures[\rmfamily]{Ligatures=TeX,Scale=1}
\fi
\usepackage{lmodern}
\ifPDFTeX\else
  % xetex/luatex font selection
    \setmainfont[Path=fonts/,Extension=.otf,UprightFont=*-regular,BoldFont=*-bold,ItalicFont=*-italic,BoldItalicFont=*-BoldItalic]{TeXGyreTermes}
\fi
% Use upquote if available, for straight quotes in verbatim environments
\IfFileExists{upquote.sty}{\usepackage{upquote}}{}
\IfFileExists{microtype.sty}{% use microtype if available
  \usepackage[]{microtype}
  \UseMicrotypeSet[protrusion]{basicmath} % disable protrusion for tt fonts
}{}
\makeatletter
\@ifundefined{KOMAClassName}{% if non-KOMA class
  \IfFileExists{parskip.sty}{%
    \usepackage{parskip}
  }{% else
    \setlength{\parindent}{0pt}
    \setlength{\parskip}{6pt plus 2pt minus 1pt}}
}{% if KOMA class
  \KOMAoptions{parskip=half}}
\makeatother
\usepackage{listings}
\newcommand{\passthrough}[1]{#1}
\lstset{defaultdialect=[5.3]Lua}
\lstset{defaultdialect=[x86masm]Assembler}
\usepackage{longtable,booktabs,array}
\usepackage{calc} % for calculating minipage widths
% Correct order of tables after \paragraph or \subparagraph
\usepackage{etoolbox}
\makeatletter
\patchcmd\longtable{\par}{\if@noskipsec\mbox{}\fi\par}{}{}
\makeatother
% Allow footnotes in longtable head/foot
\IfFileExists{footnotehyper.sty}{\usepackage{footnotehyper}}{\usepackage{footnote}}
\makesavenoteenv{longtable}
\usepackage{graphicx}
\makeatletter
\newsavebox\pandoc@box
\newcommand*\pandocbounded[1]{% scales image to fit in text height/width
  \sbox\pandoc@box{#1}%
  \Gscale@div\@tempa{\textheight}{\dimexpr\ht\pandoc@box+\dp\pandoc@box\relax}%
  \Gscale@div\@tempb{\linewidth}{\wd\pandoc@box}%
  \ifdim\@tempb\p@<\@tempa\p@\let\@tempa\@tempb\fi% select the smaller of both
  \ifdim\@tempa\p@<\p@\scalebox{\@tempa}{\usebox\pandoc@box}%
  \else\usebox{\pandoc@box}%
  \fi%
}
% Set default figure placement to htbp
\def\fps@figure{htbp}
\makeatother
\ifLuaTeX
\usepackage[bidi=basic]{babel}
\else
\usepackage[bidi=default]{babel}
\fi
\babelprovide[main,import]{english}
\ifPDFTeX
\else
\babelfont{rm}[Path=fonts/,Extension=.otf,UprightFont=*-regular,BoldFont=*-bold,ItalicFont=*-italic,BoldItalicFont=*-BoldItalic]{TeXGyreTermes}
\fi
% get rid of language-specific shorthands (see #6817):
\let\LanguageShortHands\languageshorthands
\def\languageshorthands#1{}
\ifLuaTeX
  \usepackage[english]{selnolig} % disable illegal ligatures
\fi
\setlength{\emergencystretch}{3em} % prevent overfull lines
\providecommand{\tightlist}{%
  \setlength{\itemsep}{0pt}\setlength{\parskip}{0pt}}
\usepackage[backend=biber,style=apa,uniquename=false,mincitenames=1,maxcitenames=6,citetracker]{biblatex}
\addbibresource{./references/references.bib}
\addbibresource{./references/manuscripts.bib}
% The following package and toggle
% are needed to implement the switch
% between the PhD and MS templates.
% For now, it is used only on the title
% page.
\newtoggle{ms}
\toggletrue{ms}
%%%%%%%%%%%%%%%%%%
%%% To be filled %
%%%%%%%%%%%%%%%%%%
\usepackage{latex/packages}
% Syntax colors
\definecolor{codecomments}{HTML}{455A64}
\definecolor{codekeywords}{HTML}{3D5AFE}
\definecolor{codestrings}{HTML}{00838F}
\input{latex/unicodes}
% custom thematic icons
\newcommand{\iconOPT}{\textnormal{\scalebox{.65}{\faWaveSquare}}} % optimization
\newcommand{\iconSEC}{\textnormal{\scalebox{.65}{\faShield*}}} % security
\newcommand{\iconFM}{\textnormal{\scalebox{.65}{\faCheck}}} % formal methods, verification
\newcommand{\iconSPA}{\textnormal{\scalebox{.65}{\faCode}}} % static program analysis

\newcommand{\types}{\textnormal{\raisebox{0.15\height}{\scalebox{.65}{\faStarOfLife}}}}
\newcommand{\logics}{\textnormal{\raisebox{0.25\height}{\scalebox{.65}{\faChessKing}}}}
\newcommand{\dataflow}{\textnormal{\raisebox{0.15\height}{\scalebox{.65}{\faShare*}}}}
\newcommand{\syntax}{\textnormal{\raisebox{0.15\height}{\scalebox{.65}{\faFont}}}}
\newcommand{\lalg}{\textnormal{\raisebox{0.15\height}{\scalebox{.65}{\faSuperscript}}}}
\newcommand{\recs}{\textnormal{\raisebox{0.15\height}{\scalebox{.65}{\faRedo*}}}}

% small circled characters
\newcommand{\circled}[2][black]{\hspace{-.5ex}\raisebox{-0.2\height}{\scalebox{.6}{%
    \tikz{\node[font={\selectfont\color{#1}},
        circle,inner sep=2pt,outer sep=.4pt,fill=white,
        draw=#1,solid,thick]{\textbf{#2}};}}}}

\newcommand{\circledb}[2][black]{\hspace{-.5ex}\raisebox{-0.2\height}{\scalebox{.6}{%
    \tikz{\node[font={\selectfont\color{#1}},
        circle,inner sep=1.5pt,outer sep=.4pt,fill=white,
        draw=#1,solid,line width=1.5pt]{\textbf{#2}};}}}}

% small color square
\newcommand{\langclr}[1]{%
    \tikz{\node[rectangle,minimum size=2mm,rounded corners=2pt,fill=#1]{};}}

\input{latex/rocqlang}
\documentclass[crop,border={-20pt 10pt 0pt 0pt}]{standalone}
\input{../latex/pics}
\begin{document}
 
\tikzstyle{arrow} = [thick,arrows={-Latex[width=5pt, length=5pt]}]
\tikzstyle{comment} = [densely dashed, gray!50!black, thick]
\tikzstyle{comme} = [font={\selectfont\itshape\color{gray!50!black}},text width=3cm]
\tikzstyle{label} = [midway]

\tikzstyle{phase}  = [
	font={\selectfont\bfseries},
	draw=black,
	thick,
	fill=none,
	text centered,
	rounded corners=3pt,
	minimum height=1.3cm,
	text width=2cm]


\begin{tikzpicture}[align=center]
%\draw[gray!20!white,step=.25] (-1.5,-2.5) grid (16,3.5);

\path  (0,  0.0) node[phase] (0B)  {Dafny source file}
	  +(0, -1.5) node[comme] (0C)  {typically *.dfy}
   	 ++(2.5,2.5) node[phase] (1B)  {Dafny AST}
	 ++(3, -2.5) node[phase] (2B)  {Boogie}	 
	 ++(0, -1.8) node[comme] (2C)  {intermediate verification language}
	 ++(3,  4.3) node[phase] (3B)  {SMT formulas}
	 ++(3, -2.5) node[phase] (4B)  {Verification result} 
	  +(0, -1.9) node[comme] (4C)  {Correctness, error model, or timeout}
	 ++(0, 2) node[phase,draw=none] (5B)  {Output};

\path[every node/.style={font=\small\color{sorange}}]
    (0B) edge[arrow, bend right=-30] node [left,label,xshift=-5pt,yshift=5pt] {Parsing} (1B)
    (1B) edge[arrow, bend right=30]  node [left,label,yshift=-15pt] {Translation} (2B)
    (2B) edge[arrow, bend right=-30] node [left,label,xshift=0pt,yshift=20pt] {Verification\\condition\\generator} (3B)
    (3B) edge[arrow, bend right=30]  node [left,label,yshift=-15pt] {SMT\\solver} (4B)
    (4B) edge[arrow]  node {} (5B);

\draw [comment] (0C) -- (0B) node {}; 
\draw [comment] (2C) -- (2B) node {}; 
\draw [comment] (4C) -- (4B) node {}; 

\end{tikzpicture}
\end{document}

\lstdefinelanguage{OpenMP}{
    mathescape=true,
    morekeywords={for,int},
    moredirectives={pragma,omp,parallel,teams,schedule,single,for,%
    distribute,barrier,critical,atomic,private,shared,
    nowait,target,loop,order,unroll,full,partial
    },
    moredelim=*[directive]\#,%
    morecomment=[l]{//},
    morecomment=[s]{/*}{*/},
    morestring=[b]",%
    breaklines=true,
    tabsize=2,
    sensitive=true,
    breaklines=true,
    columns=fullflexible,
    prebreak={\space\textbackslash},
    postbreak={},
}[keywords,comments,strings,directives]

\lstdefinestyle{inlineomp}{
    morekeywords={pragma,omp,parallel,teams,schedule,%
    single,for,distribute,barrier,critical,atomic,private,
    shared,nowait,target,loop,order,unroll,full,partial
    },
    keywordstyle={\ttfamily\color{black}}}
\newcommand{\pr}{\lstinline[mathescape,basicstyle=\ttfamily,escapeinside={(*}{*)}]}
\newcommand{\prc}{\lstinline[language=C,basicstyle=\ttfamily,keywordstyle=\mdseries]}
\newcommand{\omp}[1]{\lstinline[language=InlineOmp]|#1|\index{OpenMP directives!\texttt{#1}}}
\newcommand{\prm}[1]{\text{\pr|#1|}}

\lstset{
    columns=[l]flexible,
    basicstyle=\small\ttfamily\linespread{4},
    identifierstyle=\color{codeid},
    commentstyle=\color{codecomments},
    keywordstyle=\color{codekeywords}\bfseries,
    ndkeywordstyle=\color{codekeywords}\bfseries,
    stringstyle=\color{codestrings},
    abovecaptionskip=-30pt,
    aboveskip=2em,
    belowskip=2em,
    breakatwhitespace=false,
    breaklines=true,
    captionpos=b,
    escapechar=!,
    escapeinside=||,
    extendedchars=true,
    float=H,
    frame=tb,
    keepspaces=true,
    mathescape=false,
    numbers=left,
    prebreak=\raisebox{0ex}[0ex][0ex]{↲},
    sensitive=true,
    showspaces=false,
    showstringspaces=false,
    showtabs=false,
    tabsize=2,
    upquote=true,
}

\lstdefinestyle{Plain}{
    language=C,
    identifierstyle=\color{black},
    commentstyle=\color{black},
    keywordstyle=\color{black}\bfseries,
    ndkeywordstyle=\color{black}\bfseries,
    stringstyle=\color{black}\ttfamily,
}

\lstdefinestyle{Dafny}{
    language=dafny,
    basicstyle=\small\ttfamily\linespread{4},
    identifierstyle=\color{codeid},
    keywordstyle={\color{codekeywords}\ttfamily\bfseries},
    ndkeywordstyle={\color{codekeywords}\ttfamily\bfseries},
    commentstyle=\color{codecomments},
    stringstyle=\color{codestrings}
}

\lstdefinestyle{openmp}{
    language=OpenMP,
    directivestyle={\color{ompdirective}\ttfamily\bfseries\itshape}
}

\lstnewenvironment{console}{\lstset{
    backgroundcolor=\color{consolebg},
    columns=fullflexible,
	basicstyle=\ttfamily\footnotesize\mdseries,
    showstringspaces=false,
    tabsize=4,
    keepspaces=true,
    showtabs=true,
    showspaces=false,
    framesep=2pt,
    frame=single,
    framerule=.5pt,
    framexleftmargin=5pt,
    framexrightmargin=5pt,
    xleftmargin=5.5pt,
    xrightmargin=5.5pt,
    breaklines=true,
    prebreak=\raisebox{0ex}[0ex][0ex]{↲},
    escapeinside={||},
    numbers=none,
    emph={}
    aboveskip=0em,
    belowskip=0em,
}}{}
%! suppress = NonBreakingSpace
%-------------------------------------------------------------------------------
% Custom macros
%-------------------------------------------------------------------------------

% Manuscript headers
\newcommand\CTNT{Clément Aubert, Thomas Rubiano, Neea Rusch, Thomas Seiller}
\newcommand\ainfo[2]{\noindent{#1}\newline\noindent{#2}\newpage}
\newcommand\ainfoX[3]{\noindent{#1}\newline\noindent{#2}\vspace{1em}\newline{#3}\newpage}

% Document commands
\newcommand{\aref}[1]{\hyperref[#1]{\ref*{#1}}}
\newcommand{\swlink}[2]{\texttt{\footnotesize{\href{#1}{#2}}}}
\newcommand{\nonterm}[1]{#1} % ott
\newcommand*\lstinputpath[1]{\lstset{inputpath=#1}}
\newcommand{\smtabularnote}[1]{\tabularnote{\small{#1}}}
\newcommand{\ompi}[1]{\index{OpenMP directives!{#1}}}
\newcommand{\ccxi}[1]{\index{complexity classes!\textsc{#1}}}
\newcommand{\ccx}[1]{\textsc{#1}\ccxi{#1}}
\newcommand{\ndx}[1]{#1\index{#1}}

% Inline codes
\newcommand{\pr}{\lstinline[mathescape,basicstyle=\normalfont\ttfamily,escapeinside={(*}{*)},prebreak={}]}
\newcommand{\prm}[1]{\text{\pr|#1|}}
\newcommand{\prc}{\lstinline[mathescape=true]}
\newcommand{\omp}[1]{\lstinline[language=OpenMP,basicstyle=\ttfamily\color{black},keywordstyle=\ttfamily\color{black}]|#1|\ompi{#1}}

% apa-style citation fix
\renewcommand{\cite}[2][]{\autocite[#1]{#2}}

% continuous appendix
\newcommand{\continousappendix}{
 \renewcommand{\thechapter}{\arabic{chapter}}
 \setcounter{chapter}{6}} % chapters are fixed anyway

% paper category icons
\newcommand{\pageIcon}[2]{
\raisebox{-0.3\height}{\begin{tikzpicture}
\node[circle,inner sep=2pt,outer sep=.4pt,fill=white,draw=black,solid,thick]{#1};
\end{tikzpicture}}\hspace{1em}\textit{Implicit computational complexity \& #2}\par}
\newcommand{\pageIconAnalysis}{\pageIcon{\iconSPA}{static analysis}}
\newcommand{\pageIconFm}{\pageIcon{\iconFM}{formal methods}}
\newcommand{\pageIconOpt}{\pageIcon{\iconOPT}{program optimization}}
\newcommand{\pageIconSecurity}{\pageIcon{\iconSEC}{security}}

% boxed text
\makeatletter
\renewcommand{\boxed}[1]{\text{\fboxsep=.2em\fbox{\m@th$\displaystyle#1$}}}
\makeatother

% text with background color
\newcommand{\hilight}[2]{\makebox[5pt][l]{\color{#1!20!white}\rule[-4pt]{#2}{14pt}}}

% Fix \sc undefined
\providecommand{\sc}{}
\renewcommand{\sc}[1]{#1}

% tabularx custom maximally-wide column
\newcolumntype{C}{>{\centering\arraybackslash}X}
\newcolumntype{R}{>{\raggedleft\arraybackslash}X}
\newcolumntype{Z}{>{\raggedleft\arraybackslash}X}

% Math operators etc.
\newcommand{\zmat}{\mathbf{0}}
\newcommand{\umat}{\mathbf{1}}
\newcommand\hypo{\Hypo}
%\newcommand\infer{\Infer}
\DeclareMathOperator{\In}{In}
\DeclareMathOperator{\Out}{Out}
\DeclareMathOperator{\PrD}{PrD}
\DeclareMathOperator{\id}{id}
\DeclareMathOperator{\Id}{Id}
\DeclareMathOperator{\var}{var}
\DeclareMathOperator{\Card}{Card}
\DeclareMathOperator{\Occ}{Occ}
\DeclareMathOperator{\poly}{poly}
\renewcommand{\gets}{=} % algorithms

\DeclarePairedDelimiter{\sem}{\llbracket}{\rrbracket} % Semantics of program.
\newcommand{\BNF}{\enspace \ensuremath{\Vert} \enspace} % BNF separator
\newcommand{\mat}[1]{\left(\begin{smallmatrix}#1\end{smallmatrix}\right)} % Shorthand for matrices
\ebproofnewstyle{small}{separation = 1em, rule margin = .5ex}

\newtheorem{thm}{Theorem}
\newtheorem{corollary}[thm]{Corollary}
\newtheorem{lemma}[thm]{Lemma}
\newtheorem{definition}{Definition}
\newtheorem{conjecture}[thm]{Conjecture}
\newtheorem{proposition}{Proposition}[section]
\newtheorem{claim}{Claim}
\newtheorem{remark}{Remark}
\newtheorem{notation}{Notation}
\newtheorem{example}{Example}
\newtheorem{examples}{Examples}
\newtheorem{remarks}{Remarks}
\newtheorem{facts}{Facts}
\newtheorem{fact}{Fact}

% Custom terms
\newcommand\mwp{{mwp}\xspace}
\newcommand{\mwpsc}{\textnormal{\textsc{mwp}}\xspace}
\newcommand{\vdashJK}{\vdash} %_{\textnormal{\textsc{jk}}}}
\newcommand{\DFG}{\textsc{dfg}\xspace}
\newcommand{\dfg}[1]{\mathbb{M}(\mathtt{#1})}
\newcommand{\dfgtilde}[1]{\mathbb{\tilde{M}}(\mathtt{#1})}
\newcommand{\scc}{\textsc{scc}\xspace}
\newcommand{\sccs}{\textnormal{{\scshape SCC}s\xspace}}
\newcommand{\condcorr}[1]{\mathrm{Corr}^{\mathtt{if}}(#1)}
\newcommand{\whilecorr}[1]{\mathrm{Corr}^{\mathtt{while}}(#1)}
\newcommand{\corr}[1]{\mathrm{Cr}(#1)}
\newcommand{\corrc}[1]{\mathrm{Corr}(#1)}
\newcommand{\Var}{\mathrm{Vars}}
\newcommand{\lvl}[1]{\ell(#1)}
\newcommand{\vi}{\textnormal{\scalebox{.65}{\faTint}}}
\newcommand{\nv}{\cdot} % Command for "no leak" / "no violation.
\newcommand{\lname}{\(\top^{\ast}_{\textsc{ni}}\)\@\xspace} % change it if you don't like it
\newcommand{\tool}{\textsc{tyni}\@\xspace}
\newcommand{\nupce}[1]{\buildrel #1 \over \nsim}% Not up to c equilavence

\newcommand{\impl}{$\text{mwp}_\ell$\xspace}
\newcommand{\impf}{$\text{mwp}_f$\xspace}
\newcommand{\explain}{LucidLoop}
\newcommand{\exname}{{\explain}\xspace}
\newcommand{\sfull}{\scalebox{.75}{\faIcon{circle}}\xspace}
\newcommand{\spart}{\scalebox{.75}{\faIcon{adjust}}\xspace}
\newcommand{\snone}{\scalebox{.75}{\faIcon[regular]{circle}}\xspace}
\newcommand{\qsymb}{\faQuestionCircle[regular]}
\newcommand{\qtext}{\scriptsize\qsymb\normalsize}
\newcommand{\myqm}{\scalebox{.75}{\qsymb}}
\newcommand{\myok}{\scalebox{.8}{$\checkmark$}}

\newcommand{\SFM}{{\sc sfm}\xspace}
\newcommand{\SFMs}{{\sc sfm}s\xspace}
\newcommand{\sfm}[1]{\mathbb{M}(#1)}
\newcommand{\sfmb}[1]{\ensuremath{\mathbb{M}^{\mathtt{e}}(#1)}}
\renewcommand{\qedsymbol}{$\square$}
\newcommand{\replabel}{\label} % will be redefined in restatements
\newcommand{\SC}{\mathrm{SC}}
\newcommand{\SCset}{\mathit{SC}}
\newcommand{\LH}{\mathrm{LH}}
\newcommand{\HMO}{\mathrm{HMO}}
\newcommand{\scl}[1]{\ensuremath{\mathit{#1}}}
\newcommand{\SSG}{\mathrm{SSG}}
\newcommand{\orth}{\mathbin{\bot}} % orthogonal
\newcommand{\upce}[1]{\buildrel #1 \over\sim}% Up to c equilavence
\makeatletter
\renewcommand{\nsim}{\mathrel{\mathpalette\n@sim\relax}}
\newcommand{\n@sim}[2]{%
\ooalign{%
    $\m@th#1\sim$\cr
    \hidewidth$\m@th#1\rotatebox[origin=c]{50}{$#1-$}$\hidewidth\cr
}%
}
\makeatother

% Comma after eg and ie
\newcommand*{\eg}{e.g.\@,\xspace}
\newcommand*{\Eg}{E.g.\@,\xspace}
\newcommand*{\cf}{cf.\@\xspace}
\newcommand*{\ie}{i.e.\@,\xspace}
\newcommand*{\aka}{a.k.a.\@\xspace}
\newcommand*{\Ie}{I.e.,\@\xspace}
\newcommand{\stt}{s.t.\@\xspace}
\newcommand*{\resp}{resp.\@\xspace}
\newcommand*{\wrt}{w.r.t.\@\xspace}
\newcommand*{\wlg}{w.l.o.g.\@\xspace}
\renewcommand{\aka}{a.k.a.\@\xspace}
\newcommand{\stsup}{$^\text{st}$\@\xspace}
\newcommand{\ndsup}{$^\text{nd}$\@\xspace}
\newcommand{\thsup}{$^\text{th}$\@\xspace}
\makeatletter
\newcommand*{\etc}{%
    \@ifnextchar{.}%
{etc}%
{etc.\@\xspace}%
}
\makeatother

% HELPERS FOR TERM AND SYMBOLS INDICES
% shorthand for symbols
\newcommand\symbo[1]{\gls{symb:#1}}
\setsepchar{,}
% Includepdf helpers : terms
\makeatletter
\newcommand\addtoindex[2]{\ifnum\AM@page=#2\relax\index{#1}\fi}
\makeatother
\newcommand{\addtoindexm}[2]{\readlist\ipages{#2}\foreachitem\word\in\ipages{\addtoindex{#1}{\word}}}
% Includepdf helpers : symbols
\makeatletter
\newcommand\addtosymbols[2]{\ifnum\AM@page=#2\relax\gls{symb:#1}\fi}
\makeatother
\newcommand{\addtosymbolsm}[2]{\readlist\ipages{#2}\foreachitem\word\in\ipages{\addtosymbols{#1}{\word}}}
% All sections begin on a new page
\AddToHook{cmd/chapter/before}{\clearpage}
\renewcommand\thechapter{\Roman{chapter}}
\renewcommand\thesection{\Alph{section}}
\renewcommand{\cftchapnumwidth}{30pt}
\renewcommand{\cftsecnumwidth}{30pt}
\setlength{\cftbeforechapskip}{12pt} % Space before chapters
\setlength{\cftbeforesecskip}{10pt} % Space before sections
\setlength{\cftbeforesubsecskip}{4pt} % Space before subsections
\setlength{\cftbeforesubsubsecskip}{2pt} % Space before subsubsections
\setlength{\cftsecindent}{0pt} % Remove indent for section
\setlength{\cftsubsecindent}{0pt} % Remove indent for subsection
\setcounter{tocdepth}{2}
\addtocontents{toc}{\protect\thispagestyle{empty}}
\addtokomafont{disposition}{\rmfamily}
\addtokomafont{section}{\rmfamily}
\addtokomafont{subsection}{\rmfamily\rule{0pt}{30pt}}
\addtokomafont{subsubsection}{\rmfamily}
\renewcaptionname{english}{\contentsname}{Table of Contents}

% autorefs
\AtBeginDocument{%
\renewcommand{\chapterautorefname}{Chap.}
\renewcommand{\sectionautorefname}{Sect.}
\renewcommand{\subsectionautorefname}{Sect.}
\renewcommand{\figureautorefname}{Fig.}
\newcommand{\definitionautorefname}{Def.}}
\newcommand{\algorithmautorefname}{Algo.}
\newcommand{\notationautorefname}{Notation}

% Comma after eg and ie
\newcommand*{\eg}{e.g.\@,\xspace}
\newcommand*{\Eg}{E.g.\@,\xspace}
\newcommand*{\cf}{cf.\@\xspace}
\newcommand*{\ie}{i.e.\@,\xspace}
\newcommand*{\aka}{a.k.a.\@\xspace}
\newcommand*{\Ie}{I.e.,\@\xspace}
% \newcommand{\st}{s.t.\@\xspace}
\newcommand*{\resp}{resp.\@\xspace}
\newcommand*{\wrt}{w.r.t.\@\xspace}
\newcommand*{\wlg}{w.l.o.g.\@\xspace}
\renewcommand{\aka}{a.k.a.\@\xspace}

\makeatletter
\newcommand*{\etc}{%
    \@ifnextchar{.}%
{etc}%
{etc.\@\xspace}%
}
\makeatother


\makeindex
\makenoidxglossaries

\newacronym{ast}{AST}{abstract syntax tree}
\newacronym{api}{API}{application programming interface}
\newacronym{icc}{ICC}{Implicit Computational Complexity}
\newacronym{c99}{C99}{C language standard ISO/IEC 9899:1999}
\newacronym{sc}{SC}{security class}
\newacronym{sgg}{SGG}{security semi-group}
\newacronym{dfg}{DFG}{data flow graph}
\newacronym{sfm}{SFM}{security-flow matrix}
\newacronym{smt}{SMT}{Satisfiability Modulo Theories}
\newacronym{sql}{SQL}{Structured Query Language}
\newacronym{openmp}{OpenMP}{Open Multi-Processing, parallel programming API}
\newacronym{gcc}{GCC}{The GNU Compiler Collection, compiler infrastructure}
\newacronym{cli}{CLI}{command line interface}

\glsaddall
\inputott{latex/sts}

% resource paths
\graphicspath{{./pdf/}}
\lstset{inputpath={./code}}


%%%%%%%%%%%%%%%%%%
%%% To be filled %
%%%%%%%%%%%%%%%%%%
\togglefalse{ms}
\toggletrue{hal} % set true to use HAL format cover

\newcommand{\yourtitle}{Applied Implicit \mbox{Computational Complexity}}
\newcommand{\yourname}{Neea Rusch}
\newcommand{\youradvisor}{Dr. Cl{\'{e}}ment Aubert}
\newcommand{\yourkeywords}{
    {programming languages},
    {static program analysis},
    {implicit computational complexity},
    {verification},
    {formal methods}}
\newcommand{\yourdate}{2025-09-01}

\newcommand{\committee}{
    Dissertation defended on Friday, 15 August 2025, in Augusta, Georgia\\
    before the dissertation committee composed of:

    \vspace{1em}\begin{tabular}{p{.9\textwidth}}
        Dr. Clément Aubert, \textit{major advisor} \mydotfill{ }Augusta University, United States  \\
        Dr. Martin Avanzini \mydotfill{ }Centre Inria d’Université Côte d’Azur, France \\
        Dr. Yuyan Bao \mydotfill{ }Augusta University, United States \\
        Dr. Bogdan Chlebus \mydotfill{ }Augusta University, United States \\
        Dr. Harley Eades III \mydotfill{ }Augusta University, United States \\
    \end{tabular}}

%%%%%%%%%%%%%%%%%%%%%%%%%
%%%% Optional arguments %
%%%%%%%%%%%%%%%%%%%%%%%%%

\newcommand{\yourlicence}{\href{https://creativecommons.org/licenses/by/4.0/}{CC Attribution 4.0 International}}
% You can add a "mention", typically to
% indicate that your manuscript is a draft
\newcommand{\yourmention}{}

%%%%%%%%%%%%%%%%%%%%%%%%
% ⚠ Do not edit ⚠      %
% what is below at all %
%%%%%%%%%%%%%%%%%%%%%%%%


%%%%%%%%%%%%
% Packages %
%%%%%%%%%%%%

\usepackage{scrhack}                               % Hack from the koma-script for various packages to play more nicely with this class.
\usepackage{xstring}                               % Used to perform substitution, using \StrSubstitut.
\usepackage[figure, table, lstlisting]{totalcount} % To conditionally insert list of figures, tables, and listings.
% https://tex.stackexchange.com/a/297657
% https://tex.stackexchange.com/a/297655
\usepackage[english]{datetime2}                    % To extract the month and year from the date.
\usepackage{chngcntr}                              % To obtain a "global numbering" of tables and figures.
\usepackage{xpatch}                                % To patch some commands, for the title page.

%%%%%%%%%%%%%%
% Cover page %
%%%%%%%%%%%%%%

% We add some space after the subtitle if it
% is defined, after the title if not.
\ifdefined\yoursubtitle
    \makeatletter
        \xapptocmd{\@subtitle}{\par}{}{}
    \makeatother
\else
    \let\yoursubtitle\par\vspace{1em}
\fi

% We add "By", followed by a new line,
% before your name, and change the font 
% to 16pts.
\makeatletter
    \xpatchcmd{\maketitle}{\@author}{By \\ \@author}{}{}
\makeatother

% The following extract the month and year
% from the date, and display them in the title.
\DTMsavedate{mydate}{\yourdate}
\makeatletter
\renewcommand{\yourdate}{%
    \DTMenglishmonthname{\@dtm@month}\\
    \@dtm@year
}
\makeatother

% This add some information between 
% your name and the date.
\makeatletter
\pretocmd{\yourdate}{%
    \vspace{1em}
    Submitted to the Faculty of The Graduate School\\
    of Augusta University in partial fulfillment\\
    of the Requirements of the Degree of\\
    \iftoggle{ms}{Master of Science}{Doctor of Philosophy} % This toggle will display either 
    % "Master of Science" or "Doctor of Philosophy"
    % based on the choice made in info.tex.
    \vspace{1em}\par
}{}{}
\makeatother

% This add copyright information, 
% abusing the "publisher" field.
\makeatletter
\publishers{%
    \textcopyright~\@dtm@year{} by \yourname%
    \ifdefined\yourlicence{\\[.1em] \yourlicence}
    \else
    \relax
    \fi
    \pagenumbering{gobble} % No page number on next page.
}
\makeatother

%%%%%%%%%%%
% Margins %
%%%%%%%%%%%

% We do not want to add any additional margin space
% on the cover page
\renewcommand*{\coverpagetopmargin}{2in}
\renewcommand*{\coverpageleftmargin}{1.5in}
\renewcommand*{\coverpagerightmargin}{1in}
\renewcommand*{\coverpagebottommargin}{1in}

% We set all the margins to 0.
% Sorry about that, I know it is not pretty,
% but that is the only way I could content
% TGS's requirements on the template.
\setlength{\topmargin}{0pt}
\setlength{\headheight}{0pt}
\setlength{\headsep}{0pt}
\setlength{\columnsep}{0pt}
\setlength{\marginparsep}{0pt}
\setlength{\marginparpush}{0pt}
\setlength{\marginparwidth}{0pt}

% No space above chapters.
% https://tex.stackexchange.com/a/231940
\RedeclareSectionCommand[
    beforeskip=0pt
]{chapter}

%%%%%%%%%%
% Titles %
%%%%%%%%%%

% Chapter titles should be centered.
\renewcommand\raggedchapter{\centering}

%%%%%%%%%%%%%
% Meta-data %
%%%%%%%%%%%%%

\AtEndPreamble{
    \hypersetup{
        pdftitle={\yourtitle},
        pdfauthor={\yourname},
        pdflang={en},
        pdfkeywords={\yourkeywords},
        pdfauthor={\yourname},
        plainpages=false % Not sure this is helpful -- I thought it would help with https://github.com/the-au-forml-lab/au_ccs_dissertation_template/issues/13
    }
}


%%%%%%%%%
% Fonts %
%%%%%%%%%

% Some of the fonts parameters are set-up in the info.md file.
% The title is in small caps.
\setkomafont{title}{\scshape\Large}
% Figures and legends should be 10-point font. 
\addtokomafont{caption}{\small}
\addtokomafont{captionlabel}{\small}
% Subtitle and title should use the same font size.
\setkomafont{subtitle}{\scshape\Large}
% Author name should be 16 pts
\addtokomafont{author}{\Large}
% Date should be in smaller font.
\setkomafont{date}{\large}


%%%%%%%%%%%
% Spacing %
%%%%%%%%%%%

% Everything must be double spaced
% "Double space" has multiple definitions,
% we implement two options below, and use the 
% first by default.
% Refer to https://tex.stackexchange.com/q/13742
% for more details.

% Option A:
\usepackage[nodisplayskipstretch]{setspace}
\doublespacing

% Option B:
% Uncomment lines with two % to use.
% and comment Option A to use.
%% \linespread{2}
% But the sectionning commands
% (titles, subtitles, etc.)
% should *not* be double-spaced.
% https://tex.stackexchange.com/a/365269/
%% \addtokomafont{disposition}{\linespread{1}}
% Since biblatex is loaded quite late in the template 
% (cf. https://github.com/jgm/pandoc-templates/blob/master/default.latex )
% we defer this command, that let the line space in the references be only 
% single-spaced, to the end of the preamble.
%% \AtEndPreamble{
%%    \AtNextBibliography{\linespread{1}}
%% }


%%%%%%%%%%%%%%%%%%%%%%%%%%%%%%%%%%%%%%%
% Dedication, a.k.a. Acknowledgements % 
%%%%%%%%%%%%%%%%%%%%%%%%%%%%%%%%%%%%%%%

\pretocmd{\dedication}{%
    \pdfbookmark[0]{Acknowledgements}{toc}
    \chapter*{Acknowledgements}
    \pagenumbering{gobble}      % No page number
    \par                        % New paragraph
    {
        #1                      % Actual text
    }
    \clearpage                  % New page
}{}{}


%%%%%%%%%%%%
% Abstract % 
%%%%%%%%%%%%


% https://tex.stackexchange.com/a/40547
% https://tex.stackexchange.com/a/68227
\makeatletter
\newenvironment{abstract}{%
    % At the beginning of the environment:
    \pdfbookmark[0]{Abstract}{abs} % We add it to the pdf toc
    \chapter*{Abstract}\label{abs}
    {                       % This group must be single-spaced.
        \linespread{1}
        \textsc{\yourname} \\
        \@title \\
        (Under the direction of \textsc{\youradvisor})
    }
    \\[3em]
}%
{%
    %  At the end of the environment:
    \\[3em] \textsc{Keywords}: \StrSubstitute{\yourkeywords}{,}{\textperiodcentered} % We list the keywords.
    \clearpage
    % The table of content
    % follows immediately (and automatically)
    % the abstract.
    \renewcaptionname{english}%
    {\contentsname}%
    {Table of Contents}                    % We rename the table of contents.
    \hypertarget{tableofcontents}{}\bookmark[level=chapter,dest=tableofcontents]{\contentsname}
    \tableofcontents\clearpage                 % Table of contents.
    \iftotaltables                             % If there are tables in the document…
        \hypertarget{listoftables}{}\bookmark[level=chapter,dest=listoftables]{\listtablename}   %…we add the List of Tables to the pdf toc.
        \listoftables\clearpage                          % …and write it.
    \fi
    \iftotalfigures                          % If there are figures in the document…
        \hypertarget{listoffigures}{}\bookmark[level=chapter,dest=listoffigures]{\listfigurename}   % …and add it to the pdf toc.
        \listoffigures\clearpage                         % …write out the List of Figures
        \fi
    \iftotallstlistings                      % If there are listings in the document…
        \renewcommand{\lstlistlistingname}%
        {List of Listings}                     % …we rename the simple "Listings" to "List of Listings"
        % https://stackoverflow.com/a/2709986
        \hypertarget{listoflistings}{}\bookmark[level=chapter,dest=listoflistings]{\lstlistlistingname}   %…we add the List of Tables to the pdf toc.
        \lstlistoflistings\clearpage                     % …write out the List of Listings.
    \fi
    \hypertarget{listofalgorithms}{}\bookmark[level=chapter,dest=listofalgorithms]{\listalgorithmname}
    \listofalgorithms\clearpage
}
\makeatother

\pagestyle{empty}

%%%%%%%%%%%%%%%%%%%%%%%%%%%%%%%%
% Names of tables and captions %
%%%%%%%%%%%%%%%%%%%%%%%%%%%%%%%%

% Since biblatex is loaded quite late in the template 
% (cf. https://github.com/jgm/pandoc-templates/blob/master/default.latex )
% we defer those commands, that let the references being in the table of contents.
% and make sure that the bibliography is treated as a chapter, and not a section.
\AtEndPreamble{
    \DeclarePrintbibliographyDefaults{heading=bibintoc} % We add the references in the table of contents
    \defbibheading{bibliography}[\bibname]{%
        \chapter*{#1}%
        \markboth{#1}{#1}}
}
% https://tex.stackexchange.com/a/544718


% Counters for figures, tables and listings
% are global, and not per chapter.
% https://tex.stackexchange.com/q/371184
\counterwithout{figure}{chapter}
\counterwithout{table}{chapter}
\lstset{numberbychapter=false}
% https://tex.stackexchange.com/a/595356

%%%%%%%%%%%%%%%%%%%%%%%%%%%%%%
% Mention (draft annotation) %
%%%%%%%%%%%%%%%%%%%%%%%%%%%%%%

\ifdefined\yourmention
    \usepackage{accsupp} % We make it impossible to select the mention
    % cf. https://tex.stackexchange.com/a/309878
    \usepackage{draftwatermark}
    \SetWatermarkText{\BeginAccSupp{method=escape,ActualText={}}\normalfont\yourmention\EndAccSupp{}}
    \SetWatermarkScale{2}
    \SetWatermarkColor{augustagrey!20}
\fi
%%%%%%%%%%%%%%%%%%%%%%%%%%%%%%%%%%%%%%%%%%%%%%%%%%%
% Everything below can be freely edited.          % 
% Beware that it may break the compilation of     %
% the main demo file, though.                     %
%%%%%%%%%%%%%%%%%%%%%%%%%%%%%%%%%%%%%%%%%%%%%%%%%%%


%%%%%%%%%%%%%%%%%%%%%%
% Debugging packages %
%%%%%%%%%%%%%%%%%%%%%%

% \usepackage{showframe} % show the page layout
\usepackage{layouts}   
% Allow to use 
% \printinunitsof{in}{\pagevalues}
% to "see" the margins.

% The following commands allows to "see" the font sizes.
% cf. https://tex.stackexchange.com/a/24600
% https://texfaq.org/FAQ-csname
\makeatletter
\newcommand\thefontsize[1]{\csname #1\endcsname #1 is equivalent to \f@size pt\par}
\makeatother

\newcommand{\getfontsize}{
    \thefontsize{tiny}
    \thefontsize{scriptsize}
    \thefontsize{footnotesize}
    \thefontsize{small}
    \thefontsize{normalsize}
    \thefontsize{large}
    \thefontsize{Large}
    \thefontsize{LARGE}
    \thefontsize{huge}
    \thefontsize{Huge}
    \normalsize
}

%%%%%%%%%%%%%%%%%%%%%%%%%%%
% Emoji support for latex %
%%%%%%%%%%%%%%%%%%%%%%%%%%%

\usepackage[verbose]{newunicodechar}

% List of symbols supported by Symbola at
% https://www.fileformat.info/info/unicode/font/symbola/list.htm
\defaultfontfeatures[Symbola]{Path=fonts/, Extension={.ttf}, UprightFont={*}}
\newfontfamily\sym{Symbola}
\DeclareTextFontCommand{\symb}{\sym}

% The following unicode symbols are rendered in the 
% symbola font, as they do not exist in the main
% font of the template:
\newunicodechar{🔒}{\symb 🔒} % U+1F512, "LOCK"
\newunicodechar{✘}{\symb ✘} % U+2718,  "HEAVY BALLOT X"
\newunicodechar{⚠}{\symb ⚠} % U+26A0,  "WARNING SIGN"
\newunicodechar{❓}{\symb ❓} % U+2753, "BLACK QUESTION MARK ORNAMENT"
\newunicodechar{🔜}{\symb 🔜} % U+1F51C, "SOON WITH RIGHTWARDS ARROW ABOVE"
\newunicodechar{ℕ}{\symb ℕ} % U+2115,  "DOUBLE-STRUCK CAPITAL N"
\newunicodechar{ℤ}{\symb ℤ} % U+2124,  "DOUBLE-STRUCK CAPITAL Z"
\newunicodechar{✔}{\symb ✔} % U+2714,  "HEAVY CHECK MARK"
% More symbols for new lines: https://stackoverflow.com/a/18931703
\newunicodechar{↵}{\symb ↵} % U+21B5,  "DOWNWARDS ARROW WITH CORNER LEFTWARDS"
\newunicodechar{↲}{\symb ↲} % U+21B2,  "DOWNWARDS ARROW WITH TIP LEFTWARDS"
\newunicodechar{🛡}{\symb 🛡} % U+1F6E1, "SHIELD"
\newunicodechar{ℝ}{\symb ℝ} % U+211D,  "Double-Struck Capital R"
\newunicodechar{□}{\symb □} % U+25A1,  "White square"

% Note that you can also define unicode characters
% to be interpreted as latex commands, following
% https://tex.stackexchange.com/a/522961 :

\newunicodechar{↔}{\ensuremath{\leftrightarrow}}

%%%%%%%%%%%%%%%%%%%%%%%%%%%%%%%%%%%% 
% Colors                           %
% https://brand.augusta.edu/color/ %
%%%%%%%%%%%%%%%%%%%%%%%%%%%%%%%%%%%%

% Those are the "official" AU color, but we are assuming black-and-white printing.
\definecolor{augustablue}{HTML}{002f55} % Used for "external" links.
\definecolor{augustagrey}{HTML}{A5ACAF} % Used for "internal" links.

% Non-colored links, with underline, cf. https://tex.stackexchange.com/a/26085
% This allow the links to be visually present only if the document is viewed on a screen.
% Colored distribution inspired by https://tex.stackexchange.com/a/526148

% Setup new colors
\AtEndPreamble{
    \hypersetup{
        pdfborder={0 0 1},
        pdfborderstyle={/S/U/W 1}, % border style will be underline of width 1pt
        linkbordercolor=augustagrey,
        citebordercolor=augustagrey,
        filebordercolor=augustablue,
        urlbordercolor=augustablue,
        menubordercolor=augustagrey,
        runbordercolor=augustablue
    }
}

%%%%%%%%%%%%%%%%%%%%%
% Code Presentation %
%%%%%%%%%%%%%%%%%%%%%

% We make listings be a bit more pretty,
% cf. https://tex.stackexchange.com/a/272133

\lstset{
    % Space skipped before code block
    % Default style fors listings
    basicstyle=\small\ttfamily\linespread{4},
    % flexible columns
    columns=[l]flexible,
    %commentstyle=\color[rgb]{0.127,0.427,0.514}\ttfamily\itshape,
    escapechar=@,
    % Enables ASCII chars 128 to 255
    extendedchars=true,
    % Frame only at the top and bottom
    frame=tb,
    % Coloring schemes
    %identifierstyle=\color{black},
    %keywordstyle=\color[HTML]{228B22}\bfseries,
    %ndkeywordstyle=\color[HTML]{228B22}\bfseries,
    % Style for strings
    %stringstyle=\ttfamily,
    % Style for comments
    %commentstyle={\ttfamily\color[HTML]{228B22}},
    % Line numbers at the left, and font left by default
    numbers=left,
    % numberstyle=\tiny, % uncomment if you want smaller line numbers.
    % Automatic breaking of long lines
    breaklines=true,
    % Add the "↲" symbol whenever a line is broken.
    % The "↲" symbol is declared as a unicode symbol
    % in the other header file (head_a.tex).
    prebreak=\raisebox{0ex}[0ex][0ex]{↲},
    % How strings are formatted, and style of quote sign.
    %stringstyle=\color[rgb]{0.639,0.082,0.082}\ttfamily,
    upquote=true,
    % Anything betweeen $ does not become LaTeX math mode
    mathescape=false,
    % Spaces are not displayed as a special character
    showstringspaces=false,
    % Size of tabulations
    tabsize=3,
    % Case sensitivity
    sensitive=true,
    % Position of captions is bottom
    captionpos=b,
    % We reduce the space between the caption and the code.
    % This seems to come from a strange interaction between
    % the listings and caption packages,
    % cf. https://tex.stackexchange.com/a/365260/
    abovecaptionskip=-30pt,
    % Floating option
    float=H,
    % We increment the space before and after the listings.
    aboveskip=2em,
    belowskip=2em,
}

\lstset{literate=%
   *{0}{{{\color{darkgray}0}}}1
    {1}{{{\color{darkgray}1}}}1
    {2}{{{\color{darkgray}2}}}1
    {3}{{{\color{darkgray}3}}}1
    {4}{{{\color{darkgray}4}}}1
    {5}{{{\color{darkgray}5}}}1
    {6}{{{\color{darkgray}6}}}1
    {7}{{{\color{darkgray}7}}}1
    {8}{{{\color{darkgray}8}}}1
    {9}{{{\color{darkgray}9}}}1
}

% We define a style for the Coq programming language
% https://tex.stackexchange.com/a/620012/
% lstlisting coq style (inspired from a file of Assia Mahboubi)

\definecolor{dkgreen}{rgb}{0,0.6,0}
\definecolor{ltblue}{rgb}{0,0.4,0.4}
\definecolor{dkviolet}{rgb}{0.3,0,0.5}

\lstdefinelanguage{coq}{ 
    % Comments may or not include Latex commands
    texcl=false, 
    % Vernacular commands
    morekeywords=[1]{Section, Module, End, Require, Import, Export,
        Variable, Variables, Parameter, Parameters, Axiom, Hypothesis,
        Hypotheses, Notation, Local, Tactic, Reserved, Scope, Open, Close,
        Bind, Delimit, Definition, Let, Ltac, Fixpoint, CoFixpoint, Add,
        Morphism, Relation, Implicit, Arguments, Unset, Contextual,
        Strict, Prenex, Implicits, Inductive, CoInductive, Record,
        Structure, Canonical, Coercion, Context, Class, Global, Instance,
        Program, Infix, Theorem, Lemma, Corollary, Proposition, Fact,
        Remark, Example, Proof, Goal, Save, Qed, Defined, Hint, Resolve,
        Rewrite, View, Search, Show, Print, Printing, All, Eval, Check,
        Projections, inside, outside, Def},
    % Gallina
    morekeywords=[2]{forall, exists, exists2, fun, fix, cofix, struct,
        match, with, end, as, in, return, let, if, is, then, else, for, of,
        nosimpl, when},
    % Sorts
    morekeywords=[3]{Type, Prop, Set, true, false, option},
    % Various tactics, some are std Coq subsumed by ssr, for the manual purpose
    morekeywords=[4]{pose, set, move, case, elim, apply, clear, hnf,
        intro, intros, generalize, rename, pattern, after, destruct,
        induction, using, refine, inversion, injection, rewrite, congr,
        unlock, compute, ring, field, fourier, replace, fold, unfold,
        change, cutrewrite, simpl, have, suff, wlog, suffices, without,
        loss, nat_norm, assert, cut, trivial, revert, bool_congr, nat_congr,
        symmetry, transitivity, auto, split, left, right, autorewrite},
    % Terminators
    morekeywords=[5]{by, done, exact, reflexivity, tauto, romega, omega,
        assumption, solve, contradiction, discriminate},
    % Control
    morekeywords=[6]{do, last, first, try, idtac, repeat},
    % Comments delimiters, we do turn this off for the manual
    morecomment=[s]{(*}{*)},
    % String delimiters
    morestring=[b]",
    morestring=[d],
    % Style for (listings') identifiers
    identifierstyle={\ttfamily\color{black}},
    % Style for declaration keywords
    keywordstyle=[1]{\ttfamily\color{dkviolet}},
    % Style for gallina keywords
    keywordstyle=[2]{\ttfamily\color{dkgreen}},
    % Style for sorts keywords
    keywordstyle=[3]{\ttfamily\color{ltblue}},
    % Style for tactics keywords
    keywordstyle=[4]{\ttfamily\color{dkblue}},
    % Style for terminators keywords
    keywordstyle=[5]{\ttfamily\color{dkred}},
    %Style for iterators
    % keywordstyle=[6]{\ttfamily\color{dkpink}},
%    literate=
%    {\\forall}{{\color{dkgreen}{$\forall\;$}}}1
%    {\\exists}{{$\exists\;$}}1
%    {<-}{{$\leftarrow\;$}}1
%    {=>}{{$\Rightarrow\;$}}1
%    {==}{{\code{==}\;}}1
%    {==>}{{\code{==>}\;}}1
%    %    {:>}{{\code{:>}\;}}1
%    {->}{{$\rightarrow\;$}}1
%    {<->}{{$\leftrightarrow\;$}}1
%    {<==}{{$\leq\;$}}1
%    {\#}{{$^\star$}}1
%    {\\o}{{$\circ\;$}}1
%    {\@}{{$\cdot$}}1
%    {\/\\}{{$\wedge\;$}}1
%    {\\\/}{{$\vee\;$}}1
%    {++}{{\code{++}}}1
%    {~}{{\ }}1
%    {\@\@}{{$@$}}1
%    {\\mapsto}{{$\mapsto\;$}}1
%    {\\hline}{{\rule{\linewidth}{0.5pt}}}1
    %
}[keywords,comments,strings]

%%%%%%%%%%%%%%%
% Derivations %
%%%%%%%%%%%%%%%

% We recommend using the more modern ebproof over
% the more "traditional" bussproofs,
\usepackage{ebproof}

%%%%%%%%%%%%%%%%%%%%%%%%%%%%%%%%%%%%%%%%%%%%%%%%%%%%%%%
% Nice frames, for the documents we will be including %
%%%%%%%%%%%%%%%%%%%%%%%%%%%%%%%%%%%%%%%%%%%%%%%%%%%%%%%

\usepackage[breakable]{tcolorbox} % Will be used for frame around included documents.
% https://tex.stackexchange.com/a/66156
\DeclareRobustCommand\titleforcurrentframe{temp}  % Title for the current page of the frame.
% This macro is re-defined in \modifiedincludepdf
% and \modifiedincludetxt
\newcommand{\mybox}[1]{%
    \begin{tcolorbox}[
        colframe=augustablue,
        colback=white,
        width={\dimexpr\textwidth},
        breakable,
        adjusted title={\hypersetup{citecolor=white}\titleforcurrentframe}
    ]
    #1
    \end{tcolorbox}
}

%%%%%%%%%%%%%%%%%%%%%%%%%%%
% Commands to include pdf %
%%%%%%%%%%%%%%%%%%%%%%%%%%%

\usepackage{pdfpages}
% https://tex.stackexchange.com/questions/198091/get-number-of-pages-of-external-pdf/198095#198095
\newcommand*{\numberofpages}[1]{%
    \the\XeTeXpdfpagecount"#1" %
}

% Counter to add label to individual pages of the pdf.
% https://tex.stackexchange.com/a/25113
\newcounter{currentpagecounter}
\newcounter{totaldocpages}
% Command to include pdf document.
% Usage:
% \modifiedincludepdf{options for includepdf}{label}{full path to the document}{title of the document}{"level" (e.g., section, subsection, etc.)}
\newcommand{\modifiedincludepdf}[6]{
    \let\fbox\mybox % includepsf, with the option "frame", use \fbox
    % to draw the frame. We change the command, to use
    % our custom frame, that uses tcolorbox.
    \setcounter{currentpagecounter}{0}
    \setcounter{totaldocpages}{\numberofpages{#3}}
    \renewcommand{\titleforcurrentframe}{#4 (p.\ \thecurrentpagecounter\ / \thetotaldocpages) \hfill #6} % #6 acts as a subtitle
    \includepdf[#1,%
    width=\textwidth,%
    pages=-,%
    frame,
    clip,
    pagecommand={
        \stepcounter{currentpagecounter} % We increment the counter for the number of pages.
        \label{#2.\thecurrentpagecounter}% We add a label of the form "label.pagenumber".
    },
    link = true,
    linkname = {#2},
    addtotoc={1, #5, 1, #4, #2} % We add an entry to the table of content.
    ]{#3}
}

%%%%%%%%%
% Misc. %
%%%%%%%%%

% To set maximum width and height, courtesy of https://github.com/jgm/pandoc-templates/blob/master/default.latex.orig
\makeatletter
    \def\maxwidth{\ifdim\Gin@nat@width>\linewidth\linewidth\else\Gin@nat@width\fi}
    \def\maxheight{\ifdim\Gin@nat@height>\textheight\textheight\else\Gin@nat@height\fi}
\makeatother

\usepackage{pdflscape}  % For sideways figures and landscape pages.
\usepackage{csquotes}   % For proper quotations.
\usepackage{mathtools}  % For math. environments
\usepackage{amsthm}     % For math. environments

% For the BibTex logo, courtesy of https://tex.stackexchange.com/a/198472
\def\BibTeX{\textrm{B\kern-.05em\textsc{i\kern-.025em b}\kern-.08em T\kern-.1667em\lower.7ex\hbox{E}\kern-.125emX}}

\newtheorem{theorem}{Theorem}
\usepackage{bookmark}
\IfFileExists{xurl.sty}{\usepackage{xurl}}{} % add URL line breaks if available
\urlstyle{same}
\hypersetup{
  pdflang={en},
  hidelinks,
  pdfcreator={LaTeX via pandoc}}

\title{\yourtitle}
\makeatletter
\providecommand{\subtitle}[1]{% add subtitle to \maketitle
  \apptocmd{\@title}{\par {\large #1 \par}}{}{}
}
\makeatother
\subtitle{\yoursubtitle}
\author{\yourname}
\date{\yourdate}

\begin{document}
\frontmatter
\maketitle

\mainmatter
\frontmatter

%! suppress = LabelConvention
\dedication{\input{text/acks}}

\begin{abstract}
The abstract must not exceed 350 words
\end{abstract}

\mainmatter

%-------------------------------------------------------------------------------
%	1. INTRODUCTION
%-------------------------------------------------------------------------------
\chapter{Introduction}\label{introduction}
\clearpage\section{Statement of the Problem and Specific Aims}\label{intro}

Statement of the problem and specific aims of the overall project.


\begin{figure}[p]
\centering
\includegraphics[width=\linewidth,height=\textheight,keepaspectratio]{pdf/fig_conn_papers}
\caption{Dissertation manuscript connections.}\label{fig:conn_papers}
\end{figure}

% How to read notes
%
% All PDF files referenced in this dissertation are archived on Internet Archive.
% To recover a URL from the archive:
% https://web.archive.org/web/<URL>
% For example,
% https://web.archive.org/web/https://types22.inria.fr/files/2022/06/TYPES_2022_paper_14.pdf
% produces the corresponding document.
%
% TODO: Software notes

\clearpage\section{Review of the Literature}\label{sec:pre}

Literature review and discussion of the rationale of the project.
It is expected that the literature review will be more comprehensive than
those presented in the included publications.

\begin{lstlisting}[style=Dafny,caption={a code block test}]
/* long comment */
method DafnyEx() {
  var x := 100;
  while x > 0 {
    x := x - 1;
  }
  assert x == 0;
}
\end{lstlisting}

%-------------------------------------------------------------------------------
%	2. PUBLISHED MANUSCRIPTS
%-------------------------------------------------------------------------------
\chapter{Published Manuscripts}\label{published-manuscripts}\clearpage

\section{pymwp: A Static Analyzer Determining Polynomial Growth Bounds}\label{sec:atva}
\pageIconAnalysis
\ainfoX{\CTNT}{\href{https://atva-conference.org/2023}
{The 21\stsup International Symposium on Automated Technology for Verification and Analysis (ATVA), 2023}}
{\noindent Companion tool user guide: \aref{app:toolguide}
\newline\noindent Software artifact: \href{https://doi.org/10.5281/zenodo.7908484}{10.5281/zenodo.7908484}
\newline\noindent Citation:~\cite{aubert2023b}
\newline\newline\textit{Reproduced with permission from Springer Nature.}}
\includepdf[pages={1-}, addtotoc={
1,subsection,2,{Introduction -- Making Use of Implicit Complexity},sec:intro,
3,subsection,2,{Calculating Bounds with mwp-Analysis},sec:idea,
5,subsection,2,{Technical Overview of pymwp},sec:tool,
6,subsection,2,{Implementation Advancements},sec:solutions,
8,subsection,2,{Experimental Evaluation},sec:eval,
11,subsection,2,{Conclusion},sec:conc},
addtolist={
5,table,Comparison of obtained resource bounds,tab:compare,
10,table,Benchmark results,tab:eval,
10,table,Examples of obtained bounds,tab:bounds},
pagecommand={\thispagestyle{empty}%
\addtoindex{C4B}{4}
\addtoindex{C4B}{5}
\addtoindex{LOOPUS}{4}
\addtoindex{LOOPUS}{5}
\addtoindex{Abstract Syntax Tree}{5}
\addtoindex{Abstract Syntax Tree}{6}
\addtoindex{domain-specific languages}{11}
\addtoindex{honest polynomial}{3}
\addtoindex{intermediate representation}{11}
\addtoindex{mwp-bound}{3}
\addtoindex{mwp-bound}{8}
\addtoindex{nondeterminism}{1}
\addtoindex{nondeterminism}{6}
\addtoindex{nondeterminism}{7}
\addtoindex{nondeterminism}{9}
\addtoindex{nondeterminism}{11}
\addtoindex{pycparser}{6}
\addtoindex{pymwp}{11}
\addtoindex{pymwp}{1}
\addtoindex{pymwp}{2}
\addtoindex{pymwp}{4}
\addtoindex{pymwp}{5}
\addtoindex{pymwp}{6}
\addtoindex{pymwp}{7}
\addtoindex{pymwp}{8}
\addtoindex{pymwp}{9}
\addtoindex{matrix!mwp}{9}
\addtoindex{choice vector}{8}
\addtosymbols{Xprime}{10}
\addtosymbols{Xprime}{2}
\addtosymbols{Xprime}{3}
\addtosymbols{Xprime}{4}
\addtosymbols{Xprime}{5}
\addtosymbols{Xprime}{9}
\addtosymbols{hp}{3}
\addtosymbols{infty}{10}
\addtosymbols{infty}{3}
\addtosymbols{infty}{4}
\addtosymbols{infty}{7}
\addtosymbols{infty}{8}
\addtosymbols{infty}{9}
\addtosymbols{m}{3}
\addtosymbols{p}{3}
\addtosymbols{w}{3}
\addtosymbols{zero}{10}
\addtosymbols{zero}{3}
\addtosymbols{zero}{5}
}]{pdf/pubs_atva.2023.pdf}
\clearpage

\section{Distributing and Parallelizing Non-canonical Loops}\label{sec:vmcai}
\pageIconOpt
\ainfoX{\CTNT}{The 24\thsup International Conference on Verification, Model Checking, and Abstract Interpretation (VMCAI), 2023}
{\noindent Software artifact: \href{https://zenodo.org/records/7080145}{10.5281/zenodo.7080145}
\newline\noindent Citation:~\cite{aubert202213}
\newline\newline\textit{Reproduced with permission from Springer Nature.}}
\includepdf[pages={1-},addtotoc={
2,subsection,2,{Original Approaches to Automatic Parallelization},sec:vmcai-intro,
4,subsection,2,{Background: Language and Dependency Analysis},sec:vmcai-background,
11,subsection,2,{Loop Fission Algorithm},sec:fission-algo,
14,subsection,2,{Limitations of Existing Alternative Approaches},sec:vmcai-limitations,
16,subsection,2,{Evaluation},sec:vmcai-eval,
21,subsection,2,{Conclusion},sec:vmcai-conc},
addtolist={
5,figure,{Simple imperative while language},fig-grammar,
6,table,{Definition of Out, In and Occ for commands},table:def-out-in-occ,
8,figure,{Statement examples, sets, and representations of their dependencies},fig:dependences,
9,figure,Data-Flow Graph of composition,fig:composition,
13,figure,Distributing a more complex while loop,fig:exampe-complex,
16,table,Parallelization tools feature support comparison,tab:comparison,
17,figure,Code transformation example,fig:code-example,
19,figure,Speedup of selected benchmarks,fig:while,
20,table,Speedup comparison between parallel benchmarks,tab:speedup,
20,table,Descriptions of evaluated parallel benchmarks,tab:benchmarks},
pagecommand={\thispagestyle{empty}%
\addtoindex{Clava}{15}
\addtoindex{Clava}{16}
\addtoindex{Clava}{19}
\addtoindex{Cetus}{15}
\addtoindex{Cetus}{16}
\addtoindex{MiBench}{17}
\addtoindex{MiBench}{20}
\addtoindex{NAS Parallel Benchmarks}{17}
\addtoindex{NAS Parallel Benchmarks}{20}
\addtoindex{OpenMP}{14}
\addtoindex{OpenMP}{15}
\addtoindex{OpenMP}{16}
\addtoindex{OpenMP}{18}
\addtoindex{OpenMP}{19}
\addtoindex{OpenMP}{20}
\addtoindex{OpenMP}{2}
\addtoindex{OpenMP}{3}
\addtoindex{OpenMP}{4}
\addtoindex{matrix!empty}{7}
\addtoindex{Par4All}{15}
\addtoindex{Par4All}{16}
\addtoindex{Pluto}{15}
\addtoindex{Pluto}{16}
\addtoindex{Polybench/C}{16}
\addtoindex{Polybench/C}{17}
\addtoindex{Polybench/C}{18}
\addtoindex{ROSE}{14}
\addtoindex{ROSE}{15}
\addtoindex{ROSE}{16}
\addtoindex{ROSE}{17}
\addtoindex{ROSE}{18}
\addtoindex{ROSE}{19}
\addtoindex{ROSE}{20}
\addtoindex{TRACO}{15}
\addtoindex{TRACO}{16}
\addtoindex{compilation!parallelizing}{15}
\addtoindex{compilation!parallelizing}{16}
\addtoindex{compilation!parallelizing}{17}
\addtoindex{compilation!parallelizing}{19}
\addtoindex{compilation!parallelizing}{3}
\addtoindex{compilation!source-to-source}{15}
\addtoindex{compilation!source-to-source}{1}
\addtoindex{compilation!source-to-source}{3}
\addtoindex{dependency analysis}{14}
\addtoindex{dependency analysis}{21}
\addtoindex{dependency analysis}{2}
\addtoindex{dependency analysis}{4}
\addtoindex{dependency analysis}{6}
\addtoindex{graph!bi-partite}{7}
\addtoindex{graph!condensation}{11}
\addtoindex{graph!condensation}{12}
\addtoindex{graph!condensation}{13}
\addtoindex{graph!covering}{11}
\addtoindex{graph!covering}{12}
\addtoindex{graph!covering}{13}
\addtoindex{graph!covering}{14}
\addtoindex{graph!data-flow}{6}
\addtoindex{graph!data-flow}{7}
\addtoindex{graph!dependence}{11}
\addtoindex{graph!dependence}{14}
\addtoindex{graph!dependence}{15}
\addtoindex{saturated covering}{12}
\addtoindex{saturated covering}{13}
\addtoindex{Intel's C++ compiler}{15}
\addtoindex{Intel's C++ compiler}{16}
\addtoindex{loop form!canonical}{15}
\addtoindex{loop form!canonical}{16}
\addtoindex{loop form!canonical}{17}
\addtoindex{loop form!canonical}{2}
\addtoindex{loop form!canonical}{4}
\addtoindex{loop form!non-canonical}{21}
\addtoindex{loop form!non-canonical}{2}
\addtoindex{loop transformation!distribution}{2}
\addtoindex{loop transformation!fission}{11}
\addtoindex{loop transformation!fission}{12}
\addtoindex{loop transformation!fission}{14}
\addtoindex{loop transformation!fission}{15}
\addtoindex{loop transformation!fission}{16}
\addtoindex{loop transformation!fission}{17}
\addtoindex{loop transformation!fission}{18}
\addtoindex{loop transformation!fission}{19}
\addtoindex{loop transformation!fission}{20}
\addtoindex{loop transformation!fission}{2}
\addtoindex{loop transformation!fission}{3}
\addtoindex{loop transformation!fission}{4}
\addtoindex{loop transformation!fission}{6}
\addtoindex{loop transformation!tiling}{4}
\addtoindex{loop transformation!fusion}{21}
\addtoindex{loop transformation!unrolling}{21}
\addtoindex{loop transformation!unrolling}{3}
\addtoindex{loop-carried dependency}{15}
\addtoindex{loop-level parallelism}{2}
\addtoindex{automatic parallelization}{15}
\addtoindex{automatic parallelization}{16}
\addtoindex{automatic parallelization}{19}
\addtoindex{automatic parallelization}{20}
\addtoindex{automatic parallelization}{3}
\addtoindex{automatic parallelization}{4}
\addtoindex{parallelization potential}{3}
\addtoindex{polyhedral optimization}{4}
\addtoindex{state explosion}{3}
\addtosymbols{corro}{9}
\addtosymbols{corr}{10}
\addtosymbols{corr}{8}
\addtosymbols{corr}{9}
\addtosymbols{dfg}{10}
\addtosymbols{dfg}{11}
\addtosymbols{dfg}{7}
\addtosymbols{dfg}{8}
\addtosymbols{et}{9}
\addtosymbols{graphw}{11}
\addtosymbols{graphw}{12}
\addtosymbols{graph}{12}
\addtosymbols{infty2}{10}
\addtosymbols{infty2}{11}
\addtosymbols{infty2}{7}
\addtosymbols{infty2}{8}
\addtosymbols{infty2}{9}
\addtosymbols{looppw}{12}
\addtosymbols{looppw}{13}
\addtosymbols{looppw}{14}
\addtosymbols{loopw}{10}
\addtosymbols{loopw}{11}
\addtosymbols{loopw}{12}
\addtosymbols{loopw}{13}
\addtosymbols{loopw}{14}
\addtosymbols{mdfg}{10}
\addtosymbols{mdfg}{11}
\addtosymbols{mdfg}{13}
\addtosymbols{mdfg}{7}
\addtosymbols{mdfg}{8}
\addtosymbols{mdfg}{9}
\addtosymbols{occ}{10}
\addtosymbols{occ}{6}
\addtosymbols{occ}{7}
\addtosymbols{occ}{9}
\addtosymbols{one}{10}
\addtosymbols{one}{7}
\addtosymbols{one}{8}
\addtosymbols{one}{9}
\addtosymbols{vin}{11}
\addtosymbols{vin}{6}
\addtosymbols{vin}{7}
\addtosymbols{vin}{8}
\addtosymbols{vout}{10}
\addtosymbols{vout}{11}
\addtosymbols{vout}{6}
\addtosymbols{vout}{7}
\addtosymbols{vout}{8}
\addtosymbols{vout}{9}
\addtosymbols{zero2}{10}
\addtosymbols{zero2}{7}
\addtosymbols{zero2}{8}
\addtosymbols{zero2}{9}
}]{pdf/pubs_vmcai.2023.pdf}
\clearpage

%-------------------------------------------------------------------------------
%	3. UNPUBLISHED RESEARCH
%-------------------------------------------------------------------------------
\chapter{Unpublished Research}\label{ch:unpublished-research}\clearpage

\section{Polynomial Postconditions via mwp-Bounds}\label{sec:postcond}
\pageIconFm
\ainfoX{Neea Rusch}{In review at \href{https://2025.ecoop.org/}
{The European Conference on Object-Oriented Programming (ECOOP) 2025}}{}
\includepdf[pages={1-}, addtotoc={
1,subsection,2,Introduction,pc-intro,
4,subsection,2,Preliminaries,prelim,
6,subsection,2,Technical foundations,terms,
8,subsection,2,Variable-guided matrix exploration,enhancements,
12,subsection,2,mwp-bounds as postconditions,analysis,
14,subsection,2,Implementation and evaluation,expr,
16,subsection,2,Analysis of experiment results,results,
19,subsection,2,Related works,related-works,
20,subsection,2,Conclusion and future directions,conclusion},
addtolist={
15,table,Benchmarks characteristics summarized,tab:suites,
17,table,{Analysis results for mwp analyses},tab:linear,
17,table,Daikon analysis results,tab:daikon,
19,table,Postcondition comparison on select benchmarks,tab:pc},
pagecommand={\thispagestyle{empty}%
\addtoindex{DIG}{20}
\addtoindex{Dafny}{15}
\addtoindex{Dafny}{18}
\addtoindex{Dafny}{4}
\addtoindex{Daikon}{14}
\addtoindex{Daikon}{15}
\addtoindex{Daikon}{16}
\addtoindex{Daikon}{18}
\addtoindex{Daikon}{20}
\addtoindex{EvoSpex}{20}
\addtoindex{Frama-C}{19}
\addtoindex{Hoare triple}{4}
\addtoindex{Turing completeness}{3}
\addtoindex{Viper}{4}
\addtoindex{abstract interpretation}{19}
\addtoindex{assertion inference paradox}{20}
\addtoindex{choice vector}{10}
\addtoindex{choice vector}{11}
\addtoindex{choice vector}{12}
\addtoindex{choice vector}{9}
\addtoindex{deductive verification}{4}
\addtoindex{dynamic program analysis}{15}
\addtoindex{dynamic program analysis}{16}
\addtoindex{dynamic program analysis}{20}
\addtoindex{honest polynomial}{8}
\addtoindex{imperative programs}{15}
\addtoindex{imperative programs}{19}
\addtoindex{imperative programs}{2}
\addtoindex{imperative programs}{5}
\addtoindex{imperative programs}{6}
\addtoindex{invariant}{13}
\addtoindex{invariant}{14}
\addtoindex{invariant}{15}
\addtoindex{invariant}{19}
\addtoindex{invariant}{1}
\addtoindex{invariant}{21}
\addtoindex{invariant}{3}
\addtoindex{invariant}{4}
\addtoindex{matrix!mwp}{6}
\addtoindex{matrix!mwp}{8}
\addtoindex{matrix!mwp}{13}
\addtoindex{matrix!mwp}{18}
\addtoindex{monomial}{10}
\addtoindex{monomial}{11}
\addtoindex{monomial}{12}
\addtoindex{monomial}{7}
\addtoindex{mwp-bound}{10}
\addtoindex{mwp-bound}{11}
\addtoindex{mwp-bound}{12}
\addtoindex{mwp-bound}{13}
\addtoindex{mwp-bound}{14}
\addtoindex{mwp-bound}{16}
\addtoindex{mwp-bound}{18}
\addtoindex{mwp-bound}{20}
\addtoindex{mwp-bound}{2}
\addtoindex{mwp-bound}{3}
\addtoindex{mwp-bound}{5}
\addtoindex{mwp-bound}{6}
\addtoindex{mwp-bound}{8}
\addtoindex{nondeterminism}{15}
\addtoindex{nondeterminism}{5}
\addtoindex{nondeterminism}{6}
\addtoindex{nondeterminism}{8}
\addtoindex{precondition}{15}
\addtoindex{precondition}{18}
\addtoindex{precondition}{19}
\addtoindex{precondition}{1}
\addtoindex{precondition}{2}
\addtoindex{precondition}{4}
\addtoindex{program trace}{15}
\addtoindex{program trace}{16}
\addtoindex{program trace}{17}
\addtoindex{program trace}{18}
\addtoindex{pymwp}{10}
\addtoindex{pymwp}{14}
\addtoindex{pymwp}{15}
\addtoindex{pymwp}{3}
\addtoindex{quasi-invariant}{13}
\addtoindex{specifications}{14}
\addtoindex{specifications}{19}
\addtoindex{specifications}{1}
\addtoindex{specifications}{20}
\addtoindex{specifications}{21}
\addtoindex{specifications}{2}
\addtoindex{specifications}{3}
\addtoindex{specifications}{4}
\addtoindex{state explosion}{5}
\addtosymbols{cv}{10}
\addtosymbols{cv}{13}
\addtosymbols{cv}{9}
\addtosymbols{hoare}{4}
\addtosymbols{hp}{8}
\addtosymbols{infty}{10}
\addtosymbols{infty}{11}
\addtosymbols{infty}{12}
\addtosymbols{infty}{13}
\addtosymbols{infty}{17}
\addtosymbols{infty}{18}
\addtosymbols{infty}{5}
\addtosymbols{infty}{6}
\addtosymbols{infty}{7}
\addtosymbols{infty}{8}
\addtosymbols{infty}{9}
\addtosymbols{k}{10}
\addtosymbols{k}{13}
\addtosymbols{k}{7}
\addtosymbols{k}{8}
\addtosymbols{k}{9}
\addtosymbols{mwpi}{6}
\addtosymbols{mwpi}{7}
\addtosymbols{m}{5}
\addtosymbols{m}{6}
\addtosymbols{m}{7}
\addtosymbols{m}{8}
\addtosymbols{m}{11}
\addtosymbols{m}{12}
\addtosymbols{m}{17}
\addtosymbols{lnot}{4}
\addtosymbols{post}{4}
\addtosymbols{pre}{4}
\addtosymbols{p}{5}
\addtosymbols{p}{6}
\addtosymbols{p}{7}
\addtosymbols{p}{8}
\addtosymbols{p}{11}
\addtosymbols{p}{12}
\addtosymbols{p}{17}
\addtosymbols{ttrue}{4}
\addtosymbols{vlist}{8}
\addtosymbols{w}{5}
\addtosymbols{w}{6}
\addtosymbols{w}{7}
\addtosymbols{w}{8}
\addtosymbols{w}{11}
\addtosymbols{w}{12}
\addtosymbols{w}{17}
\addtosymbols{vdash}{4}
\addtosymbols{xprime2}{5}
\addtosymbols{zero}{5}
\addtosymbols{zero}{6}
\addtosymbols{zero}{7}
\addtosymbols{zero}{8}
\addtosymbols{zero}{11}
\addtosymbols{zero}{12}
}]{pdf/pubs_pc.2025.pdf}
\clearpage

\section{A Logic for Anytime Non-Interference}\label{sec:anytime}
\pageIconSecurity
\ainfoX{Clément Aubert and Neea Rusch}{In review at \href{https://fscd2025.github.io}
{FSCD 2025: Formal Structures for Computation and Deduction}}{}
\includepdf[pages={1-},addtotoc={
1,subsection,2,Introduction,sec:introduction,
4,subsection,2,High-level Overview,sec:overview,
4,subsection,2,The Non-interference Logic,ni-logic,
9,subsection,2,Capturing Anytime Non-Interference,sec:at-soundness,
10,subsection,2,Interpreting Function Calls in an Anytime Non-Interfering Context,sec:fct-calls,
13,subsection,2,Practical Applications and Comparison,sec:apps,
15,subsection,2,{Conclusion: Strengths, Limitations and Future Directions},sec:conclusion,
18,subsection,2,{Appendix A: Proof of Thm. 16},sec:app-a,
19,subsection,2,{Appendix B: Examples},sec:app-b},
addtolist={
5,figure,{A Simple Imperative while Language},fig:grammar,
5,table,{Definition of Out, In and Occ for commands},table:ni-def-out-in-occ,
7,figure,{Statement Examples, Sets, Representations of Possible Violation(s)},fig:ni-dependences,
7,figure,{Security-Flow Matrix of Compositions},fig:ni-composition,
12,figure,{Statement Examples, Interpretation and Sets -- Involving Effects},fig:fct-effect},
pagecommand={\thispagestyle{empty}%
\addtoindex{Dependency Core Calculus}{15}
\addtoindex{Hasse diagram}{20}
\addtoindex{Hasse diagram}{3}
\addtoindex{Hasse diagram}{6}
\addtoindex{Polybench/C}{21}
\addtoindex{Rice's Theorem}{2}
\addtoindex{attacker (adversary)}{13}
\addtoindex{attacker (adversary)}{3}
\addtoindex{attacker (adversary)}{7}
\addtoindex{attacker (adversary)}{9}
\addtoindex{attacker model!program centric}{3}
\addtoindex{attacker model}{3}
\addtoindex{bytecode}{13}
\addtoindex{bytecode}{14}
\addtoindex{complexity class}{15}
\addtoindex{confidentiality}{14}
\addtoindex{confidentiality}{1}
\addtoindex{declassification}{15}
\addtoindex{dependency analysis}{15}
\addtoindex{dependency analysis}{5}
\addtoindex{dynamic code loading}{2}
\addtoindex{matrix!hollow}{19}
\addtoindex{matrix!hollow}{5}
\addtoindex{matrix!hollow}{6}
\addtoindex{imperative programs}{15}
\addtoindex{imperative programs}{3}
\addtoindex{imperative programs}{4}
\addtoindex{information flow!control}{3}
\addtoindex{information flow!explicit}{2}
\addtoindex{information flow!explicit}{4}
\addtoindex{information flow!implicit}{15}
\addtoindex{information flow!implicit}{2}
\addtoindex{information flow!implicit}{4}
\addtoindex{information flow!implicit}{8}
\addtoindex{information flow!policy}{14}
\addtoindex{information flow!policy}{15}
\addtoindex{information flow!policy}{1}
\addtoindex{information flow!policy}{20}
\addtoindex{information flow!policy}{21}
\addtoindex{information flow!policy}{3}
\addtoindex{information flow!policy}{6}
\addtoindex{information flow}{14}
\addtoindex{information flow}{19}
\addtoindex{information flow}{1}
\addtoindex{information flow}{21}
\addtoindex{information flow}{2}
\addtoindex{information flow}{3}
\addtoindex{intermediate representation}{14}
\addtoindex{language-based security}{14}
\addtoindex{language-based security}{15}
\addtoindex{language-based security}{2}
\addtoindex{lattice}{3}
\addtoindex{matrix!empty}{7}
\addtoindex{monoid}{5}
\addtoindex{monoid}{6}
\addtoindex{monoid}{8}
\addtoindex{non-interference!anytime}{10}
\addtoindex{non-interference!anytime}{12}
\addtoindex{non-interference!anytime}{13}
\addtoindex{non-interference!anytime}{14}
\addtoindex{non-interference!anytime}{15}
\addtoindex{non-interference!anytime}{18}
\addtoindex{non-interference!anytime}{19}
\addtoindex{non-interference!anytime}{1}
\addtoindex{non-interference!anytime}{20}
\addtoindex{non-interference!anytime}{21}
\addtoindex{non-interference!anytime}{2}
\addtoindex{non-interference!anytime}{3}
\addtoindex{non-interference!anytime}{8}
\addtoindex{non-interference!anytime}{9}
\addtoindex{non-interference!progress-sensitive}{2}
\addtoindex{non-interference!termination-insensitive}{2}
\addtoindex{non-interference}{10}
\addtoindex{non-interference}{13}
\addtoindex{non-interference}{14}
\addtoindex{non-interference}{15}
\addtoindex{non-interference}{18}
\addtoindex{non-interference}{19}
\addtoindex{non-interference}{1}
\addtoindex{non-interference}{2}
\addtoindex{non-interference}{3}
\addtoindex{non-interference}{4}
\addtoindex{non-interference}{5}
\addtoindex{non-interference}{8}
\addtoindex{non-interference}{9}
\addtoindex{security class}{10}
\addtoindex{security class}{11}
\addtoindex{security class}{13}
\addtoindex{security class}{14}
\addtoindex{security class}{15}
\addtoindex{security class}{1}
\addtoindex{security class}{2}
\addtoindex{security class}{3}
\addtoindex{security class}{4}
\addtoindex{security class}{5}
\addtoindex{security class}{6}
\addtoindex{security class}{7}
\addtoindex{security class}{8}
\addtoindex{security class}{9}
\addtoindex{security objective}{3}
\addtoindex{security type system}{15}
\addtoindex{security type system}{1}
\addtoindex{security type}{15}
\addtoindex{security type}{3}
\addtoindex{matrix!security-flow}{11}
\addtoindex{matrix!security-flow}{14}
\addtoindex{matrix!security-flow}{21}
\addtoindex{matrix!security-flow}{4}
\addtoindex{matrix!security-flow}{5}
\addtoindex{matrix!security-flow}{6}
\addtoindex{matrix!security-flow}{7}
\addtoindex{side effect}{10}
\addtoindex{side effect}{12}
\addtoindex{side effect}{13}
\addtoindex{side effect}{2}
\addtoindex{side effect}{3}
\addtoindex{system model}{3}
\addtoindex{termination}{14}
\addtoindex{termination}{2}
\addtoindex{termination}{9}
}]{pdf/pubs_ni.2025.pdf}
\clearpage

\section{Certifying Complexity Analysis}\label{coqpl}
\pageIconFm
\ainfoX{\CTNT}
{\href{https://popl23.sigplan.org/home/CoqPL-2023}
{The Ninth International Workshop on Coq for Programming Languages (CoqPL), 2023}}
{\abspage{This work drafts a strategy that leverages the field of Implicit Computational Complexity to certify resource usage in imperative programs.
This original approach sidesteps some of the most common--and difficult--obstacles \enquote{traditional} complexity theory face when implemented in Coq.}}
\input{text/coqpl}
\clearpage

%-------------------------------------------------------------------------------
%	4. DISCUSSION
%-------------------------------------------------------------------------------
\chapter{Discussion}\label{ch:discussion}
\input{text/discussion}

%-------------------------------------------------------------------------------
%	5. SUMMARY
%-------------------------------------------------------------------------------
\chapter{Summary}\label{ch:summary}
A series of concise remarks summarizing experimental findings and conclusions


%-------------------------------------------------------------------------------
%	6. REFERENCES
%-------------------------------------------------------------------------------
\backmatter
\printbibliography[label=chap:references, title=References]
\let\printbibliography\relax

%-------------------------------------------------------------------------------
%	7. APPENDICES
%-------------------------------------------------------------------------------
\appendix

\chapter{Additional Manuscripts \& Publication Companions}\label{additional-manuscripts}\clearpage

\section{mwp-Analysis Improvement and Implementation: Realizing Implicit Computational Complexity}\label{sec:fscd}
\pageIconAnalysis
\ainfoX{\CTNT}{The 7\thsup International Conference on Formal Structures for Computation and Deduction (FSCD), 2022}
{\par\enquote{mwp-Analysis Improvement and Implementation: Realizing Implicit Computational Complexity} \textcopyright{ }by{ }\CTNT.
\newline{}This work is licensed under a Creative Commons Attribution 4.0 International License.
\newline{}You should have received a copy of the license along with this work. % (\aref{app:sec:cc-sa-40-license}).
\newline{}If not, see \url{https://creativecommons.org/licenses/by/4.0/}.}
\includepdf[pages={1-},addtotoc={
 2,subsection,2,{Introduction: letting ICC drive the development of static analyzers},sec:fscd-intro,
 3,subsection,2,{Background: the original flow analysis},sec:fscd-background,
 6,subsection,2,{A deterministic, always-terminating, declension of the mwp analysis},sec:fscd-det,
10,subsection,2,{Extending and improving the analysis: functions and efficiency},sec:fscd-extending,
15,subsection,2,{Implementing, testing and comparing the analysis},sec:fscd-eval,
16,subsection,2,{Conclusion: limitations, strengths and future work},sec:fscd-conc,
20,subsection,2,{Appendix A: Technical appendix on semi-rings (abridged)},sec:fscd-app-a,
21,subsection,2,{Appendix B: Omitted Proofs},sec:fscd-app-b,
22,subsection,2,{Appendix C: Benchmarks},sec:fscd-app-c
}, addtolist={
4,figure,{Original non-deterministic flow analysis rules},fig-jkrules,
6,figure,{Deterministic improved flow analysis rules.},fig-det-rules,
23,table,{Benchmark results produced by pymwp on C programs},table:fscd-bench},
pagecommand={\thispagestyle{empty}%
\addtoindex{AProVE}{17}
\addtoindex{COSTA}{15}
\addtoindex{Cerco}{15}
\addtoindex{CoFloCo}{17}
\addtoindex{CompCert}{16}
\addtoindex{ComplexityParser}{15}
\addtoindex{Coq}{16}
\addtoindex{Gröbner basis}{13}
\addtoindex{Resource Aware ML}{15}
\addtoindex{SPEED}{15}
\addtoindex{Verasco}{15}
\addtoindex{certified-llvm}{16}
\addtoindex{complexity class@!NP}{6}
\addtoindex{complexity class@!P}{3}
\addtoindex{delta graph}{14}
\addtoindex{delta graph}{15}
\addtoindex{monomial}{13}
\addtoindex{monomial}{14}
\addtoindex{mwp-bound}{14}
\addtoindex{mwp-bound}{15}
\addtoindex{matrix!dense}{15}
\addtoindex{matrix!mwp}{3}
\addtoindex{matrix!mwp}{5}
\addtoindex{matrix!mwp}{6}
\addtoindex{matrix!mwp}{7}
\addtoindex{matrix!mwp}{8}
\addtoindex{matrix!mwp}{10}
\addtoindex{matrix!mwp}{12}
\addtoindex{matrix!mwp}{13}
\addtoindex{matrix!mwp}{14}
\addtoindex{matrix!mwp}{15}
\addtoindex{matrix!mwp}{20}
\addtoindex{matrix!mwp}{21}
\addtoindex{matrix!mwp}{22}
\addtoindex{nondeterminism}{4}
\addtoindex{nondeterminism}{5}
\addtoindex{nondeterminism}{6}
\addtoindex{nondeterminism}{7}
\addtoindex{nondeterminism}{11}
\addtoindex{nondeterminism}{14}
\addtoindex{intermediate representation}{6}
\addtoindex{pymwp}{15}
\addtoindex{pymwp}{17}
\addtoindex{pymwp}{22}
\addtoindex{pymwp}{23}
\addtoindex{quasi-invariant}{15}
\addtoindex{state explosion}{5}
\addtoindex{state explosion}{6}
\addtoindex{static single assignment}{16}
\addtosymbols{PsiProd}{12}
\addtosymbols{Psi}{12}
\addtosymbols{ai}{12}
\addtosymbols{ai}{13}
\addtosymbols{ai}{14}
\addtosymbols{ai}{7}
\addtosymbols{alpha}{20}
\addtosymbols{alpha}{12}
\addtosymbols{alpha}{13}
\addtosymbols{alpha}{3}
\addtosymbols{alpha}{4}
\addtosymbols{alpha}{7}
\addtosymbols{assgn}{10}
\addtosymbols{assgn}{11}
\addtosymbols{assgn}{14}
\addtosymbols{assgn}{21}
\addtosymbols{assgn}{7}
\addtosymbols{assgn}{9}
\addtosymbols{beta}{20}
\addtosymbols{beta}{12}
\addtosymbols{beta}{13}
\addtosymbols{beta}{3}
\addtosymbols{beta}{4}
\addtosymbols{beta}{7}
\addtosymbols{card}{14}
\addtosymbols{card}{7}
\addtosymbols{cartpi}{12}
\addtosymbols{cartpi}{13}
\addtosymbols{cartpi}{14}
\addtosymbols{cartpi}{7}
\addtosymbols{chunk}{11}
\addtosymbols{chunk}{12}
\addtosymbols{ci}{12}
\addtosymbols{ci}{13}
\addtosymbols{ci}{7}
\addtosymbols{cj}{12}
\addtosymbols{cj}{13}
\addtosymbols{cj}{14}
\addtosymbols{cj}{7}
\addtosymbols{closure}{11}
\addtosymbols{closure}{3}
\addtosymbols{closure}{4}
\addtosymbols{closure}{6}
\addtosymbols{cmat}{11}
\addtosymbols{cmat}{4}
\addtosymbols{cmat}{6}
\addtosymbols{cmat}{9}
\addtosymbols{cmat}{21}
\addtosymbols{delta}{10}
\addtosymbols{delta}{11}
\addtosymbols{delta}{12}
\addtosymbols{delta}{13}
\addtosymbols{delta}{14}
\addtosymbols{delta}{15}
\addtosymbols{delta}{7}
\addtosymbols{delta}{8}
\addtosymbols{delta}{9}
\addtosymbols{f}{10}
\addtosymbols{f}{11}
\addtosymbols{gamma}{13}
\addtosymbols{gamma}{3}
\addtosymbols{hp}{4}
\addtosymbols{infty}{10}
\addtosymbols{infty}{13}
\addtosymbols{infty}{14}
\addtosymbols{infty}{17}
\addtosymbols{infty}{21}
\addtosymbols{infty}{22}
\addtosymbols{infty}{23}
\addtosymbols{infty}{6}
\addtosymbols{infty}{7}
\addtosymbols{infty}{8}
\addtosymbols{iso}{7}
\addtosymbols{iso}{8}
\addtosymbols{iso}{20}
\addtosymbols{k}{10}
\addtosymbols{k}{12}
\addtosymbols{k}{13}
\addtosymbols{k}{14}
\addtosymbols{k}{7}
\addtosymbols{map}{6}
\addtosymbols{map}{7}
\addtosymbols{map}{8}
\addtosymbols{matrix}{10}
\addtosymbols{matrix}{11}
\addtosymbols{matrix}{20}
\addtosymbols{matrix}{3}
\addtosymbols{matrix}{4}
\addtosymbols{matrix}{6}
\addtosymbols{matrix}{7}
\addtosymbols{matrix}{9}
\addtosymbols{matrix}{21}
\addtosymbols{mi}{3}
\addtosymbols{mi}{4}
\addtosymbols{mi}{6}
\addtosymbols{mi}{7}
\addtosymbols{mi}{20}
\addtosymbols{mj}{20}
\addtosymbols{mj}{3}
\addtosymbols{mj}{4}
\addtosymbols{mj}{6}
\addtosymbols{mj}{7}
\addtosymbols{msring}{21}
\addtosymbols{msring}{12}
\addtosymbols{msring}{13}
\addtosymbols{msring}{3}
\addtosymbols{msring}{7}
\addtosymbols{mstar}{3}
\addtosymbols{mstar}{4}
\addtosymbols{mstar}{6}
\addtosymbols{mwpi}{12}
\addtosymbols{mwpi}{13}
\addtosymbols{mwpi}{14}
\addtosymbols{mwpi}{7}
\addtosymbols{mwpi}{9}
\addtosymbols{mwpi}{21}
\addtosymbols{mwpset}{13}
\addtosymbols{mwpset}{3}
\addtosymbols{mwpset}{6}
\addtosymbols{mwpset}{7}
\addtosymbols{mwpset}{9}
\addtosymbols{mzeroj}{4}
\addtosymbols{mzeroj}{6}
\addtosymbols{m}{11}
\addtosymbols{m}{12}
\addtosymbols{m}{13}
\addtosymbols{m}{20}
\addtosymbols{m}{3}
\addtosymbols{m}{4}
\addtosymbols{m}{5}
\addtosymbols{m}{6}
\addtosymbols{m}{7}
\addtosymbols{m}{8}
\addtosymbols{m}{9}
\addtosymbols{m}{21}
\addtosymbols{oplus}{20}
\addtosymbols{oplus}{10}
\addtosymbols{oplus}{11}
\addtosymbols{oplus}{12}
\addtosymbols{oplus}{3}
\addtosymbols{oplus}{4}
\addtosymbols{oplus}{6}
\addtosymbols{oplus}{8}
\addtosymbols{oplus}{9}
\addtosymbols{oplus}{21}
\addtosymbols{otimes}{20}
\addtosymbols{otimes}{11}
\addtosymbols{otimes}{12}
\addtosymbols{otimes}{3}
\addtosymbols{otimes}{4}
\addtosymbols{otimes}{6}
\addtosymbols{otimes}{8}
\addtosymbols{plusi}{7}
\addtosymbols{p}{11}
\addtosymbols{p}{12}
\addtosymbols{p}{13}
\addtosymbols{p}{20}
\addtosymbols{p}{3}
\addtosymbols{p}{4}
\addtosymbols{p}{5}
\addtosymbols{p}{6}
\addtosymbols{p}{7}
\addtosymbols{p}{8}
\addtosymbols{p}{9}
\addtosymbols{slot}{11}
\addtosymbols{slot}{12}
\addtosymbols{slot}{21}
\addtosymbols{smpi}{12}
\addtosymbols{smpi}{5}
\addtosymbols{smpi}{8}
\addtosymbols{smpi}{9}
\addtosymbols{sring}{7}
\addtosymbols{sring}{20}
\addtosymbols{timesi}{17}
\addtosymbols{timesi}{7}
\addtosymbols{v0a}{4}
\addtosymbols{v0a}{6}
\addtosymbols{vadd}{4}
\addtosymbols{vadd}{6}
\addtosymbols{vadd}{8}
\addtosymbols{vare}{4}
\addtosymbols{vec2}{11}
\addtosymbols{vec2}{12}
\addtosymbols{vec2}{13}
\addtosymbols{vec2}{21}
\addtosymbols{vec2}{5}
\addtosymbols{vec2}{7}
\addtosymbols{vec2}{8}
\addtosymbols{vec}{11}
\addtosymbols{vec}{4}
\addtosymbols{vec}{8}
\addtosymbols{vlist}{4}
\addtosymbols{vrep}{10}
\addtosymbols{vrep}{12}
\addtosymbols{vrep}{4}
\addtosymbols{vrep}{6}
\addtosymbols{vrep}{8}
\addtosymbols{w}{11}
\addtosymbols{w}{12}
\addtosymbols{w}{13}
\addtosymbols{w}{20}
\addtosymbols{w}{3}
\addtosymbols{w}{4}
\addtosymbols{w}{5}
\addtosymbols{w}{6}
\addtosymbols{w}{7}
\addtosymbols{w}{8}
\addtosymbols{w}{9}
\addtosymbols{w}{21}
\addtosymbols{zero}{11}
\addtosymbols{zero}{12}
\addtosymbols{zero}{13}
\addtosymbols{zero}{20}
\addtosymbols{zero}{3}
\addtosymbols{zero}{4}
\addtosymbols{zero}{5}
\addtosymbols{zero}{6}
\addtosymbols{zero}{7}
\addtosymbols{zero}{8}
\addtosymbols{zero}{9}
}]{pdf/pubs_fscd.2022.pdf}
\clearpage

\section{Tool User Guide for \enquote{pymwp: A Static Analyzer Determining Polynomial Growth Bounds}}\label{app:toolguide}
\pageIconAnalysis
\ainfo{\CTNT}{Publication companion for \autoref{sec:atva}.}
\includepdf[pages={1-},scale=0.9,offset=0 0.25in,
addtotoc={
 2,subsection,2,{Introduction},sec:tool-guide-intro,
 4,subsection,2,{Installation},sec:tool-guide-intall,
 5,subsection,2,{Examples},sec:tool-guide-examples,
11,subsection,2,{Learn More},sec:tool-guide-learnmore
},pagecommand={\thispagestyle{empty}%
\addtoindex{mwp-bound}{3}
\addtoindex{mwp-bound}{8}
\addtoindex{nondeterminism}{2}
\addtoindex{nondeterminism}{6}
\addtoindex{pymwp}{10}
\addtoindex{pymwp}{11}
\addtoindex{pymwp}{1}
\addtoindex{pymwp}{2}
\addtoindex{pymwp}{4}
\addtoindex{pymwp}{5}
\addtoindex{pymwp}{6}
\addtoindex{pymwp}{7}
\addtoindex{pymwp}{8}
\addtoindex{pymwp}{9}
\addtosymbols{Xprime}{10}
\addtosymbols{Xprime}{2}
\addtosymbols{Xprime}{3}
\addtosymbols{Xprime}{6}
\addtosymbols{Xprime}{8}
\addtosymbols{hp}{3}
\addtosymbols{infty}{2}
\addtosymbols{infty}{7}
\addtosymbols{infty}{8}
\addtosymbols{m}{2}
\addtosymbols{m}{3}
\addtosymbols{p}{2}
\addtosymbols{p}{3}
\addtosymbols{vlist}{3}
\addtosymbols{w}{2}
\addtosymbols{w}{3}
\addtosymbols{zero}{2}
\addtosymbols{land}{3}
\addtosymbols{land}{6}
\addtosymbols{land}{8}
\addtosymbols{land}{10}
}]{pdf/pubs_pymwp_guide.pdf}
\clearpage

\chapter{Index of Artifacts}\label{app:sec:artifacts}
\input{text/artifacts}

\chapter{Co-author Statements}\label{app:sec:coauth}
\input{text/contrib}

% ACRONYMS
\clearpage\pagestyle{plain}
\printglossary[type=\acronymtype,nonumberlist,nopostdot,style=longragged]

% TERMS INDEX
\clearpage\printindex

% SYMBOLS INDEX
\clearpage\pagestyle{plain}
%\renewcommand{\glspostdescription}{,}
\printglossary[type=symbols,nopostdot,style=longragged]

%\chapter{Creative Commons Attribution 4.0 International Public License}
%\label{app:sec:cc-sa-40-license}
%\input{text/cc40}

%\backmatter
%\printbibliography
\end{document}
