

\newcommand\ainfo[2]{\noindent{#1}\newline\noindent{#2}\newpage}
\newcommand\ainfoX[3]{\noindent{#1}\newline\noindent{#2}\vspace{1em}\newline{#3}\newpage}
\newcommand\abspage[1]{\noindent\textbf{Abstract.}{ }{#1}\newpage}
\newcommand{\nocontentsline}[3]{}
\newcommand{\tocless}[2]{\bgroup\let\addcontentsline=\nocontentsline#1{#2}\egroup}

% MATH ETC
\newcommand\mwp{{mwp}}
\newcommand{\vdashJK}{\vdash_{\textnormal{\textsc{jk}}}}
\newcommand{\comm}[1]{\mathtt{#1}}
\newcommand\hypo{\Hypo}
\newcommand\infer{\Infer}
\DeclarePairedDelimiter{\sem}{\llbracket}{\rrbracket} % Semantics of program.
\newcommand{\BNF}{\enspace \ensuremath{\Vert} \enspace} % BNF separator
\DeclareMathOperator{\In}{In}
\DeclareMathOperator{\Out}{Out}
\DeclareMathOperator{\PrD}{PrD}
\DeclareMathOperator{\id}{id}
\DeclareMathOperator{\var}{var}
\DeclareMathOperator{\Card}{Card}
\newcommand{\mat}[1]{\left(\begin{smallmatrix}#1\end{smallmatrix}\right)} % Shorthand for matrices
\newcommand{\zmat}{\mathbf{0}} % O-element for matrices
\newcommand{\umat}{\mathbf{1}} % 1-element for matrices
\ebproofnewstyle{small}{separation = 1em, rule margin = .5ex,}

