Theoretical computer science studies the fundamental capabilities and limitations of computation.
A function is \emph{computable}\index{computable function} if it is algorithmically solvable in principle~\cite[p. 234]{sipser2012}.
However, computability is independent of the required computational resources, \ie \emph{cost} or \emph{\ndx{resource usage}} of computation.
A computable function that requires an excessive amount of resources may not be solvable in practice.

Computational \ndx{complexity theory} enhances the study of \ndx{computable function}s by classifying them into \emph{\ndx{complexity classes}}\footnote{For a comprehensive exploration of complexity classes, see~\textcite{complexityzoo}.}
based on resources requirements\index{resource usage}.
Algorithms computing those functions are categorized by the amount of required \emph{resources}\index{resource usage} when executed on a specified \emph{machine model}, such as a \ndx{Turing Machine} (TM).
Which particular model of computation\index{machine model} is used does not matter, as long as one adheres to Slot and van Emde Boas Invariance Thesis\index{Invariance Thesis}~\cite{slot1984}\footnote{
    The \ndx{Invariance Thesis} asserts that a time (resp., space) cost model is reasonable for a computational model C\index{machine model} if there are mutual simulations between TMs and C such that the overhead is polynomial in time\ccxi{p} (resp., linear in space)~\cite{vanoni2022}.},
and is concerned with \ndx{complexity classes}, \ie, order of magnitudes in the resources needed.
The resource cost\index{resource usage} is estimated relative to the input size.
This way algorithms are allowed to consume more resources when the input size grows while maintaining membership in the same complexity class\index{complexity classes}.
The common resources of interest are time\index{time complexity} and space (memory)\index{space complexity}.
Complexity measures\index{complexity measure} are defined explicitly relative to the \ndx{machine model}.
For example, \ndx{time complexity} refers to the maximum number of computation steps a machine\index{machine model}
uses on any input~\cite[p. 276]{sipser2012}.
Besides (deterministic) time and space, various other resources can be viewed as complexity measures\index{complexity measure};
for example, parallel time\index{parallel time} and circuit complexity\index{circuit complexity}~\cite[p. 428]{sipser2012}.

\emph{Asymptotic analysis}\index{asymptotic analysis} is an approach for obtaining an estimation of an algorithm's resource requirements~\cite[p. 267--277]{sipser2012}\index{resource usage}.
The analysis result is expressed as \emph{asymptotic bounds}\index{asymptotic bounds}.
The \enquote{big-oh} \(\mathcal{O}\)-notation\symbo{bigo} denotes an asymptotic upper bound~\cite[p. 47]{cormen2009}\index{upper bound}.
However, the bound may or may not be asymptotically tight\index{tight bound}.
The \enquote{small-oh} \(o\)-notation\symbo{smallo} distinguishes an upper bound that is not asymptotically tight~\cite[p. 50]{cormen2009}.
Other relevant notations are \(\Omega\)\symbo{bigomega} for an asymptotic lower bound\index{lower bound},
and \(\Theta\)\symbo{bigtheta} for an asymptotic tight bound\index{tight bound}~\cite[Sect. 3.1]{cormen2009}.
Among the benefits, complexity analysis\index{asymptotic analysis} develops insight of costs\index{resource usage} of computational patterns and identifies relationships between complexity classes\index{complexity classes}.
For algorithms, it produces measures of efficiency, and allows comparing relative performance of alternative algorithms~\cite{cormen2009}.

This overview of the classical complexity theory\index{complexity theory} should be familiar from textbooks.
However, a refresher on the basics gives a baseline for comparison for the discussion that is about to follow.
The remainder of this dissertation builds on \emph{implicit} computational complexity.
Implicit complexity complements the classical complexity theory\index{complexity theory} by offering alternative perspectives and systems for reasoning about cost of computation\index{resource usage}.

\subsubsection{Computational Complexity, \emph{Implicitly}}

\emph{Implicit Computational Complexity} (ICC) is a subfield of computational complexity theory\index{complexity theory} that explores \emph{machine-free characterizations} of complexity classes\index{complexity classes}.
It inherently shifts focus from computational models\index{machine model} to \emph{programs}.
\begin{quotation}
    \noindent By implicit, we here mean that classes are not given by constraining the amount of resources\index{resource usage} a machine\index{machine model} is allowed to use, but rather by imposing linguistic constraints on the way algorithms are formulated~\cite[p. 90]{dallago2011}.
\end{quotation}
In other words, complexity classes\index{complexity classes} are described independently of notions of time or space related to a machine's definition.
For example, polynomial time complexity\ccxi{p} has been thoroughly examined under these terms.
The Cobham–Edmonds's Thesis~\cite{cobham1965,edmonds1965}\index{Cobham–Edmonds's Thesis} relates polynomial time class with the class of feasible functions\index{}\footnote{
    The Cobham-Edmonds Thesis asserts that the time complexities in any two \enquote{reasonable and general} models of computation are polynomially related;
    that is, a problem has time complexity \(t\) in some reasonable and general model of computation
    if and only if it has time complexity poly(\(t\)) in the model of single-tape Turing machines~\cite[p. 33]{goldreich2008}.}.
Multiple distinct implicit characterizations are known for many complexity classes\index{complexity classes} -- \cf~\autoref{tab:icc-results}.

Implicit computational complexity sees complexity bounds as consequences of restrictions on program structures.
In this view, it studies the computational resources embedded in programming languages by \emph{construction}.
Another common feature is that the characterizations may omit references to explicit resource bounds~\cite{moyen2017} -- refer to \eg~%! suppress = MathFunctionText
\textcite{bellantoni1992} or~\textcite{kristiansen2005} for examples.
Thus, the focus is on studying the quality, rather than quantity, of the resources needed.
Since an implicit characterization, or an \emph{ICC system}, captures a specific complexity class\index{complexity classes}, complexity analysis becomes a task of determining class membership, rather than classifying computations among many classes as in classical asymptotic analysis\index{asymptotic analysis}.

There are several good motivations for researching implicit characterizations of complexity classes\index{complexity classes}~\cite[p. 36]{kristiansen2017,jones1999}, including the following.
\begin{itemize}

    \item Implicit complexity drive better understanding of complexity classes\index{complexity classes} and sheds light on the notorious hard open problems of complexity theory\index{complexity theory}.

    \item From a programming language perspective, implicit complexity is motivating because it suggests ways to control complexity properties, and quantifies the computational power present in programming languages by construction.

    \item A programming language-based approach yields more {natural definitions and proofs} of central results than the classical approach.

    \item The characterizations demonstrate that complexity classes are intrinsic mathematical entities that do not depend on a particular machine model\index{machine model}.

    \item Implicit complexity introduces orthogonal program analysis techniques with potential to convert complexity-theoretic insights to practical analyses.

\end{itemize}

\subsubsection{Complexity via Programming Languages}

Describing the goals of implicit complexity omits the details of \emph{how} such implicit characterizations should be designed.
Over time, numerous distinct systems have been developed based on, or inspired by, \eg
calculi,
term rewriting\index{term rewriting},
data flow analysis\index{data flow analysis},
primitive recursion\index{primitive recursion},
linear logic\index{linear logic},
modal logic\index{modal logic},
type systems\index{type system},
digital circuits\index{digital circuit}, and more~\cite{kristiansen2017,marion2011,moyen2017,baillot2012}.
With such rich versatility, it becomes challenging to identify ICC systems upon encounter and to find an overarching definition that matches them all\footnote{
    On the the challenges of describing ICC systems,~\textcite[p. 14]{moyen2017} writes:
    \enquote{For a long time, I have been convinced that ICC systems are like the proverbial blind men approaching an elephant:
    each touching a different part and describing it in terms that let the others think this is a different beast.}}.

\begin{figure}[t]
    \begin{mdframed}
        \paragraph*{Implicit complexity in a nutshell.}
        The goal of implicit computational complexity can be summarized as follows.
        Let \(\mathcal{L}\)\symbo{lang} be a programming language,
        \(\mathcal{C}\)\symbo{cclass} a complexity class\index{complexity classes},
        and \(\sem{p}\)\symbo{funprog} the function computed by program \(p\)\symbo{prog}.
        Then, the task is to find a restriction \(\mathcal{R}\)\symbo{restriction} \(\subseteq \mathcal{L}\), such that the following equality holds:
        \begin{align*}\{ \sem{p} \mid p \in \mathcal{R} \} = \mathcal{C}.\end{align*}
        The variables \(\mathcal{L}\),
        \(\mathcal{C}\), and
        \(\mathcal{R}\) are the parameters that vary greatly between different ICC systems.
    \end{mdframed}
    \caption[Implicit complexity in a nutshell]
    {Description of ICC systems -- adapted from~\textcite[p. 7]{pchoux2020}}
    \label{fig:icc_systems}
\end{figure}

Recent reflective works~\cite{pchoux2020,hoffmann2022} converge on descriptions centered on programming languages.
\autoref{fig:icc_systems} summarizes the idea succinctly.
In this view, the unifying factor of implicit complexity is to find natural characterizations of \ndx{complexity classes} through \emph{restrictions} embedded in \emph{programming languages}~\cite{hoffmann2022}.
It requires introducing a language-based \emph{syntactic} restriction that guarantees a \emph{semantic} property;
typically, that a satisfactory program is a member of the characterized complexity class~\cite{moyen2017}\index{complexity classes}.
Then, if the restriction is maintained, the complexity bound will also be maintained.

The programming language-based approach has minimally two significant effects compared to classical complexity theory\index{complexity theory}.
The first is a \emph{temporal} shift that arises from making complexity an \emph{apriori} consideration in the language design.
In other words, implicit complexity systems permit reasoning about runtime properties \emph{before a program exists}.
The second is a \emph{relational} shift between a program developer and the task of performing complexity analysis\index{asymptotic analysis}.
By embedding into a language a restriction by which a complexity bound will be guaranteed, the effort of satisfying that restriction is raised upstream to the developer~\cite{moyen2017}.
In return, to obtain runtime guarantees, the developer does not need to know about the intricacies of complexity analysis or classes.
The developer can focus on writing programs and the program's behavioral guarantees are obtained for free.
Of course, this assumes that the restriction is not excessively prohibitive in expressing algorithms.

\subsubsection{Implicit Complexity by Example: Safe Recursion}
\label{safe-rec}\index{safe recursion}

This example is adapted from a presentation by~\textcite{dallago2022}.
Consider Kleene's algebra\index{Kleene algebra} of general recursive functions\index{recursive function} as a programming language.
In this language, programs are built from basic building blocks (the initial functions) by %way of
composition, primitive recursion\index{primitive recursion}, and minimization.
%Dropping minimization makes this formalism strictly less expressive: the functions one represents (the so-called primitive recursive functions) are, at least, all total.
Primitive recursive functions\index{recursive function} are
%however
well beyond any reasonable complexity class\index{complexity classes}
%They are after all the class of functions which can be computed in time and space bounded by primitive recursive functions, the latter being strictly larger than Kalmar's elementary functions. %Essentially, this is
due to the possibility of having nested recursive definitions.
As an example, one can define addition from the initial functions

\begin{center}
\(\text{add}(0, x) = x\);

\(\text{add}(n + 1, x) = \text{add}(n, x) + 1\);
\end{center}

and then define exponentiation \(n \mapsto 2^n\) from addition as follows:

\begin{center}
\(\text{exp}(0) = 1\);

\(\text{exp}(n + 1) = \text{add}(\text{exp}(n), \text{exp}(n))\).
\end{center}

\noindent Composing exp with itself allows us to form towers of exponentials of arbitrary size. %, thus saturating the Kalmar elementary functions.
A recursion on exp leads us even beyond the class of elementary recursive functions\index{recursive function}; %aforementioned class,
already too broad complexity-wise.

Rather than removing primitive recursion\index{primitive recursion} from the algebra up-front (which would have too strong an impact on the expressive power), % of the underlying function algebra)
we want to refine the algebra by distinguishing arguments that matter for complexity from arguments that do not.
The arguments of any function \(f\)\symbo{f} are dubbed either as \emph{normal}, when their value can have a relatively big impact on \(f\text{'s}\)\symbo{f} complexity, or \emph{safe} when their value can only influence \(f\)\(\text{'s}\)\symbo{f} behavior very mildly. %, combinatorially speaking.
If \(\vec{x}\)\symbo{normx} are the normal parameters and \(\vec{y}\)\symbo{safey} are the safe ones,
we indicate the value of \(f\)\symbo{f} on \(\vec{x}\)\symbo{normx} and \(\vec{y}\)\symbo{safey} as \(f(\vec{x};\vec{y})\)\symbo{normx}\symbo{safey},
where the semicolon separates the normal from the safe.
This way, the usual scheme of primitive recursion\index{primitive recursion} can be restricted as follows:

\begin{center}
    \(f (0, \vec{x};\vec{y}) = h (\vec{x};\vec{y})\);

    \(f (n + 1,\vec{x};{ }\vec{y}) = g (n, \vec{x}; \vec{y}, f(n,\vec{x}; \vec{y})) \).
\end{center}

\noindent The way we define \(f\)\symbo{f} from \(g\) and \(h\) is morphologically identical to what happens in the usual primitive recursive scheme.
That it is a restriction is a consequence of the distinction between normal and safe arguments: the recursive call \(f(n,\vec{x}; \vec{y})\)\symbo{normx}\symbo{safey} is required to be forwarded to one of g's safe argument, while the first parameter of \(f\)\symbo{f} must be normal.
This makes exp not expressible anymore, giving rise to new function algebra called \emph{safe recursion}\index{safe recursion}, discovered by Bellantoni and Cook~\cite{bellantoni1992} (BC in the following).

\begin{theorem}
The class BC equals the class FP of polynomial time computable functions.
\end{theorem}\ccxi{p}

The Bellantoni and Cook result extends an earlier result by Cobham.
Cobham's thesis asserts that being feasible is the same as being computable in polynomial time~\cite{cobham1965}.
Although Cobham characterizes the polynomial time functions independently of any particular execution model, the characterization is not fully syntactic.
This prevents an automatic procedure to check whether a program satisfies the criteria of Cobham's class~\cite{heraud2011}.
In contrast, the Bellantoni and Cook characterization is a fully-syntactic mechanism and membership in BC can be checked automatically.
The Bellantoni and Cook result was proved formally~\cite{bc_formal} by~\textcite{heraud2011}.

\subsubsection{A Snapshot of Theoretical Results}
\label{icc-theories}

The breakthrough result of safe recursion\index{safe recursion} is regarded as the first implicit characterization of a complexity class\index{complexity classes}~\cite{dallago2011,kristiansen2017,rubiano17}.
However, the same result was independently discovered by~\textcite{leivant1993} in a slightly different setting of stratified recurrence\index{stratified recurrence}~\cite[p. 762]{dallago2022}.
\autoref{tab:icc-results} shows a {very small} collection of additional impactful theoretical results.
The inclusion criteria is varied and based on representing the rich variety of underlying techniques or complexity classes\index{complexity classes};
or for presenting foundational techniques that have been extended later;
or for reflecting shifts in theoretical developments and interests over time.
The table also marks with hyperlinks works that are discussed elsewhere in this dissertation.
These theoretical results provide a conceptual \enquote{toolbox} of techniques that form the foundation
for the rest of this dissertation.

%! suppress = LineBreak
%! suppress = NonBreakingSpace
\begin{tabularx}{\textwidth}{@{}lclX@{}r@{}}
    \multicolumn{5}{@{}c@{}}{\resizebox{\textwidth}{!}{
        \textbf{Symbols.}
        \hspace{.6em}\logics{ }logics
        \hspace{.6em}\types{ }type system
        \hspace{.6em}\dataflow{ }data flow
        \hspace{.6em}\syntax{ }syntactic
        \hspace{.6em}\lalg{ }algebras
        \hspace{.6em}\recs{ }recursion}} \\*[1em]
    \toprule
    \textbf{Year} & & \textbf{Venue} & \textbf{Description} & \textbf{Classes}
    \\
    \midrule
    \endfirsthead
    \multicolumn{5}{@{}c@{}}{\resizebox{\textwidth}{!}{
        \textbf{Symbols.}
        \hspace{.6em}\logics{ }logics
        \hspace{.6em}\types{ }type system
        \hspace{.6em}\dataflow{ }data flow
        \hspace{.6em}\syntax{ }syntactic
        \hspace{.6em}\lalg{ }algebras
        \hspace{.6em}\recs{ }recursion}} \\*[1em]
    \toprule
    \textbf{Year} & & \textbf{Venue} & \textbf{Description} & \textbf{Classes}
    \\
    \midrule\endhead
    \caption[]{(Cont.)}
    \endfoot
    \endlastfoot
    1992\(^*\) & \syntax & Comput  & Safe recursion\index{safe recursion} -- \autoref{safe-rec} & \ccx{p} \\*
    && Complex. & \textcite{bellantoni1992} \\*
    \midrule
    1993\(^*\) & \recs & POPL & Stratified recurrence\index{stratified recurrence} & \ccx{p} \\
    &&& \textcite{leivant1993} \\*
    \midrule
    1995 & \lalg & CSL & Tiering \aka ramification\index{tiering}\index{ramification} & \ccx{pspace} \\*
    &&& \textcite{leivant1995} \\*
    \midrule
    1999 & \types & LICS & Non-size-increasing programs\index{non-size-increasing program} & \ccx{p} \\*
    &&& \textcite{hofmann1999} \\*
    \midrule
    2000 & \recs & LPAR & Quasi-interpretations\index{quasi-interpretation} & \ccx{p} \\*
    &&& \textcite{marion2000} \\*
    \midrule
    2001 & \syntax & J. Funct. & CONS-free programs\index{CONS-free programs} & \ccx{p}, \ccx{exp} \\*
    && Program. & \textcite{jones2001} & \\*
    \midrule
    2001 & \dataflow & POPL & Size-change termination principle\index{size-change termination principle} -- \autoref{resource-analysis} & \ccx{pspace} \\*
    &&& \textcite{lee2001} \\*
    \midrule
    2005 & \lalg & Comput. & \textsc{loop}\(^{-}\) and \textsc{clip} programs & \ccx{l}, \ccx{linspace} \\*
    && Complex. & \textcite{kristiansen2005} \\*
    \midrule
    2009 & \logics & TOCL & Flow calculus of mwp-bounds\index{mwp-calculus} -- \autoref{flow-calculus} & \ccx{p} \\*
    &&& \textcite{jones2009}  \\*
    \midrule
    2010 & \logics & Theor. & Quantum implicit computational complexity & \ccx{eqp}, \ccx{bqp}, \ccx{zqp} \\*
    && Comput. Sci. & \textcite{dallago2010} \\*
    \midrule
    2011 & \types & LICS & SAFE programs\index{SAFE programs} -- \autoref{icc-sec} & \ccx{p} \\*
    &&& \textcite{marion2011} \\*
    \midrule
    2013 & \types & ICALP & Ramified programs\index{ramification}\index{ramified programs} -- \autoref{icc-sec}  & \ccx{p}, \ccx{l} \\*
    &&& \textcite{leivant2013} \\*
    \midrule
    2013 & \types & FoSSaCS & SAFE processes\index{SAFE programs!process} -- \autoref{icc-sec} & \ccx{p} \\*
    &&& \textcite{hainry2013} \\*
    \midrule
    2015 & \types & LPAR & \ndx{d\(\ell\)T} programs\(^a\) -- \autoref{icc-sec} & \ccx{p} \\*
    &&& \textcite{baillot2015} \\*
    \midrule
    2016 & \logics & Inf. Comput. & Complexity with pointer machines & \ccx{l} \\*
    &&& \textcite{aubert2016}  \\*
    \midrule
    2016 & \recs & Inf. Comput. & Complexity of uniform boolean circuits & \ccx{nc} \\*
    &&& \textcite{bonfante2016} \\*
    \midrule
    2018 & \types & Inf. Comput. & Object Oriented programs\index{SAFE programs!object-oriented} -- \autoref{icc-sec} & \ccx{p} \\*
    &&& \textcite{hainry2018} \\*
    \midrule
    2021 & \lalg & MFCS & Probabilistic characterization \(\text{ST}_\text{PP}\) & \ccx{pp} \\*
    &&& \textcite{dallago2021} \\*
    \midrule
    2023 & \types & POPL & Stratified (STR) programs\index{stratified programs} -- \autoref{icc-sec} & \ccx{p} \\*
    &&& \textcite{hainry2023} \\*
    \midrule
    2024 & \types & POPL & Quantitative type theory with dependent types\index{quantitative type theory}\index{dependent types} & \ccx{p}, \ccx{np}, \ccx{bpp} \\*
    &&& \textcite{atkey2024} \\*
    \midrule
    2024 & \types & LICS & Aperiodic programs\index{aperiodic programs} -- \autoref{icc-sec} & \ccx{p} \\*
    &&& \textcite{hainry2024} \\*
    \bottomrule
    \multicolumn{5}{@{}l}{\(^a\) The system can handle sub-computations not in \ccx{p}.} \\
    \caption[Theoretical implicit computational complexity systems and results]{
        A list of theoretical implicit computational complexity systems and results.
        The historical first publications are marked with \(^*\)-symbol.
        The graphical symbols on left describe the restriction techniques for enforcing complexity bounds (limited to one symbol).}
    \label{tab:icc-results}
\end{tabularx}