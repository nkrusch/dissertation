Implicit computational complexity (ICC) complements classic complexity theory by
aiming to discover machine-independent characterizations of complexity classes.
The key idea is to introduce a restriction, at the level of a programming
language, that guarantees every program satisfying the restriction belongs to a
particular complexity class. There are several strong motivations for reasoning
about computational complexity implicitly. ICC allows guaranteeing program
properties by construction, it drives better understanding of complexity
classes, and yields more natural definitions and proofs of central results than
the classical approach. In addition, due to its focus on programming languages,
ICC produces static techniques for program analysis for free.

However, despite the advantages, implicit computational complexity has remained
largely a theoretical novelty. The true potential of ICC-based
techniques---including its power, utility, and limitations---are not
well-understood. The goal of the work in this dissertation is to ``put
theoretical ICC techniques to test" and to investigate their capabilities. A
guiding principle is that the applications of ICC techniques extend beyond
theoretical reasoning about computational complexity.

Through a series of research projects---that cover static program analysis,
formal methods, parallel programming, and language-based security---the
dissertation culminates in three key findings. First, it shows ICC offers
complementary and orthogonal techniques for automatic complexity analysis.
Second, it shows it is possible to adjust the baseline techniques to track other
semantic properties. The final, and perhaps the most important finding, is
reflective and concerns a long-term perspective. A fundamental feature about ICC
is that, in principle, it offers the potential to achieve desirable properties
apriori; by language design and before any program exists, without ever having
to run post-analysis on programs. The practical attainability of this goal
depends crucially on a community shift in viewing ICC in the broader context it
offers, and the continued exploration of its applied potential.