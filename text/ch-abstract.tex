Implicit computational complexity (ICC) complements classic complexity theory by developing machine-independent characterizations of complexity classes.
The idea is to introduce a restriction, at the level of a programming language, that guarantees every program satisfying the restriction belongs to a particular complexity class.
There are several strong motivations for this approach.
ICC allows guaranteeing program properties by construction, it drives better understanding of complexity classes, and yields more natural definitions and proofs of central results than the classical approach.
Due to its focus on programming languages, ICC produces automatable techniques for program analysis for free.
However, despite these numerous advantages, ICC remains largely a theoretical novelty.
The true power and advantages of ICC-based program analyses are not well-understood.
The goal of the dissertation is to ``put theoretical ICC to test'' by investigating its capabilities outside the theoretical domain.
A guiding intuition is that, if applied, implicit computational complexity could provide us new techniques for program analysis and verification.

Over a series of research projects, the dissertation results yield three main findings.
The first shows that ICC offers complementary techniques to automatically analysis of resource consumption.
Beyond resources, it is possible to adjust ICC systems to track other program properties.
This second finding suggests that unrealized potential is available in the ICC systems;
but we must learn to leverage it.
The final finding is a reflective and visionary.
A fundamental aspect of ICC is that it enables guaranteeing program properties {by construction}, before any program exists.
This means we can ensure correct behavior without needing to run any post-analysis.
This is a powerful concept, in principle, but still mostly beyond reach.
The practical attainability of the goal depends crucially on a viewpoint shift---we should consider ICC in the broader context it offers and continue exploration of its applied potential.