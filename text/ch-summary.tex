%! suppress = LineBreak
%! suppress = TooLargeSection
\section{Seeking Strong Software Quality Guarantees}
\label{sec:aicc-intro}

The gold standard in \ndx{formal verification} is showing that an implementation
of a program is {totally correct}. Total correctness\index{total correctness}
requires proving two conditions about the program: that the program meets its
specification\index{specifications}, and that the computation
terminates~\cite{leino2023}. The former ensures the program computes the
expected output for all inputs. The latter ensures that the computed result will
become observable {eventually}. Thus, a proof of \ndx{total correctness} gives us
strong assurance that the program behaves expectedly at runtime, \ie when the
program is executed.

An additional consideration in reasoning about program correctness concerns
{non-functional} properties\index{non-functional property}. Non-functional
properties characterize the program's quality features like execution latency,
power consumption, and information flow security~\cite{terbeek2018}. Pure
functional correctness is \emph{practically unsatisfactory} if a program
requires exorbitant time to compute its result; aggressively drains a device
battery, or reveals secret information~\cite{heraud2011,aubert20222}.
Engineering truly high-quality programs thus requires additional techniques that
enable us to formally verify non-functional properties. Unfortunately, analyzing
non-functional properties is non-trivial and generally
undecidable~\cite{rice1953}\index{undecidability}. Developing specification
formalisms and reasoning methods, that can provide some guarantees despite the
undecidability, is an active area of research~\cite{etaps2025}.

The field of \emph{Implicit Computational Complexity (ICC)}~\cite{dallago2011}
offers a conceivable pathway toward formal verification of certain
non-functional properties. ICC is complementary to the classic \ndx{complexity
theory} because it uses programming languages as the {mechanism} to reason about
runtime behaviors. Broadly, complexity theory is concerned with resource
requirements of computations~\cite{goldreich2008}. Those requirements are
organized in a hierarchy of \ndx{complexity classes}. The classes describe
resource requirements \wrt an abstract (mathematical) \ndx{machine model}. The
classes correspond roughly to resources like running time and memory consumption
of programs that are executed on real-world computers. To streamline this
reasoning, \emph{implicit} computational complexity aims to discover
\emph{machine-independent} characterizations of the \ndx{complexity classes}.
ICC systems are designed by introducing a \emph{restriction}, at the level of a
programming language, that guarantees a complexity property. Then, every program
satisfying the restriction is known to belong to a particular complexity class.
This in turn provides a guarantee about the program's runtime resource
consumption.

Exchanging machine models for syntactic criteria is powerful in several ways.
For example, it facilitates \ndx{formal verification} of programs. ICC enables
defining {syntactical} \ndx{specifications} of resource consumption, and
developing fully automatable analyses to verify whether a program satisfies such
a specification~\cite{heraud2011}. ICC systems have been designed for numerous
combinations of programming languages (imperative\index{imperative programs},
lambda-calculus\index{\(\lambda\)-calculus}, \ndx{function algebras},
quantum\index{quantum languages}, \ndx{term rewrite systems},
\ndx{\(\pi\)-calculus}, \ldots), restriction-techniques (\ndx{linear logic},
\ndx{data flow analysis}, \ndx{type system}s, \ndx{category theory}, \ldots),
and complexity classes (\textsc{p}, \textsc{pspace}, \textsc{l}, \textsc{pp},
\ldots)~\cite{moyen2017,pchoux2020}.

%! suppress = TooLargeSection
\section{Addressed Problem and Goals}
\label{sec:aicc-goals}

Besides uses in \ndx{formal verification}, there are many
good motivations for implicit reasoning about computational complexity. ICC
allows guaranteeing program properties \emph{by construction}, before any
program exists. ICC drives better understanding of \ndx{complexity
classes} and yields more natural definitions and proofs
of central results than classical \ndx{complexity
theory}~\cite{kristiansen2017}. For example, ICC systems
show that complexity classes are intrinsic mathematical entities that do not
depend on a particular machine model. Multiple ICC systems,
\eg~\cite{jones2009,marion2011}, are rich enough to express natural algorithms.
Since the reasoning techniques differ from the existing alternatives, the
results ICC systems produce are often orthogonal~\cite{aubert20222}. Finally,
many ICC systems---\eg~\cite{jones2009,marion2011,hainry2023,atkey2024}---are
\emph{compositional}\index{compositionality}. This means they use modular
techniques that enable bottom-up reasoning about subprograms. Compositionality
is critical for scalability~\cite{carbonneaux2015} and useful for proving
\ndx{soundness}~\cite{keidel2021}, \ie that the analysis technique
produces correct results. However, compositionality is missed by many prominent
program analysis techniques~\cite{carbonneaux2015,schiebel2024}.

Despite this rich inventory of compelling features, ICC has remained largely a
theoretical novelty, with a few exceptions%
~\cite{avanzini2017,avanzini2008,moyen20172,hainry2021,hoffmann2012,feree2018}.
A theory in absence of applications means the power, limitations, and utilities
of the theory are difficult to recognize. In our view~\cite[p.
xxxv]{bishop2003}~\cite[p. 75]{moyen2017}, theory and application are
{symbiotic}. Investigations of both are needed for scientific advancement.

The embedded manifesto of the dissertation research is that \emph{applications
are necessary} to push forward our collective understanding of implicit
computational complexity. It is conceivable that potentially great power is
locked inside the theoretical ICC systems. It remains an open problem to release
that power. Within the ICC community, more investigations are needed to explore
the capabilities of the existing systems. Externally, the ICC-based program
analyses should be exposed to broader research communities. These steps would
motivate enhancements in ICC systems and facilitate discovery of new
applications.

The work in the dissertation is only a small step toward these ambitious
long-term goals. The realistic expectations of the dissertation are about
technical and social advancement. Overall, there are four high-level goals.
\begin{enumerate}
\item Extend {applied} capabilities of ICC in automatic program analysis and
verification.
\item Take ICC a few steps closer to integration into real-world software
development workflows.
\item Initiate conversations on the relevance of ICC applications.
\item Expose ICC to broader research communities.
\end{enumerate}
A guiding intuition behind these goals is that, {if applied}, implicit
computational complexity could provide us several new practical program
analyses. The use cases of such analysis include formal verification of many
non-functional properties\index{non-functional property}.

%! suppress = TooLargeSection
\section{Research Questions and Approaches}
\label{sec:aicc-approaches}

The following \emph{main hypothesis} guides the dissertation investigation.
\begin{quote}
\noindent Implicit computational complexity offers applied utilities when lifted
from the theoretical domain.
\end{quote}
Since the hypothesis is broad, the research projects that challenge it must be
more refined. We focus on two paths. The first applies ICC in automatic analysis
of \emph{resource consumption}. This application is natural because it parallels
the theoretical use of ICC. The second path takes the investigation to a more
unconventional direction. It asks whether we can adjust ICC systems to track
\emph{other} program properties, beyond resource consumption. Stated
differently, by swapping the property of interest, it allows us to assess the
technical flexibility of ICC systems. To summarize, we investigate whether we
can \begin{enumerate*}[label=(\roman*)]
\item apply the systems in the originally designed ways, and
\item discover unexpected applications.
\end{enumerate*}

\subsection{Automatic Resource Analysis}
\label{subsec:aicc-automatic-resource-analysis}

This series of work focuses on an ICC system called the \emph{flow calculus of
mwp-bounds}~\cite{jones2009}\index{mwp-calculus}. The flow calculus is a
canonical example of a theoretical ICC system. It focuses on tracking change in
data size during computation. For \ndx{imperative programs}, it aims to
determine if it is possible to bound the growth of variable values by
polynomials in inputs. The choice to focus on this particular ICC system is
based on intuition. As a syntactical and compositional data-flow analysis, it
{seems like} a good candidate for automatic resource analysis.

Several research questions arise from the flow calculus.

\begin{enumerate}[label={(RQ\arabic*)},leftmargin=*,labelindent=1em]
\item Can we develop an \emph{automatic} program analysis based on the theory?
\item Given its paper proofs, is the theory correct, \ie can we prove formally
the soundness of the flow calculus?\index{soundness of mwp-calculus}
\item Assuming the theory can be automated, what are the use cases of such
analysis?
\end{enumerate}

These questions are still conventional because they target resource analysis. In
other words, given a complexity-theoretic technique, we should reasonably expect
that it can be applied in the designed way. If successful, the investigation
should minimally produce two research artifacts---a program analyzers and
mechanized proof---corresponding to RQ1 and RQ2. Additionally, it should produce
new insights of the flow calculus and its possible applications. Since ICC is
not the only technique for analyzing growth in data size, having an automatic
analyzer would enable comparing it to the alternative approaches.

\subsection{Analyzing Extended Properties}
\label{subsec:extended-props}

The second series of work applies implicit computational complexity to track
{other} non-functional properties\index{non-functional property}. The idea is to
select an ICC system and adjust it to capture a different non-functional
property. The motivation is multifaceted. It requires developing deep
understanding of the mechanics of the chosen technique. Conceptually, it pushes
to think about ICC outside the usual frame of computational complexity. Lastly,
investigations in this direction are important to expose ICC to broader research
communities, since the application is shifted away from (or beyond) resource
consumption.

In this series of work, the starting point is an ICC system based on
quasi-invariants (QI), as formulated in~\cite{moyen20172}. Similar to the flow
calculus, the QI framework is a data-flow analysis of \ndx{imperative programs}.
However, the QI framework supports a richer input language and is founded on a
seemingly adjustable mathematical framework. In the prior
formulation~\cite{moyen20172}, it was used to obtain a compile-time program
optimization. The optimization lifts nested loops outside their containing
loop\index{loop transformation}, such that post-transformation the program
complexity is reduced.

We ask the following questions about the extended uses of the QI framework.
\begin{enumerate}[label={(RQ\arabic*)},leftmargin=*,labelindent=1em]
\setcounter{enumi}{3}
\item How to develop a program transformation to increase \ndx{parallelization
potential}?
\item How to use it to analyze security properties, specifically
\ndx{non-interference}?
\end{enumerate}
Research questions 4 and 5 are posed here with an air of obviousness. However,
arriving to those questions was not obvious. A substantial hidden challenge of
the dissertation research plan involves identifying suitable research
domains---like parallel programming or language-based
security~\cite{sabelfeld2003}---where ICC-based techniques could offer
meaningful benefit.

Another consideration is that adjusting an ICC system may lead to losing the
original guarantee about resource consumption. In return, we gain an alternative
guarantee about program behavior. Whether this trade is beneficial depends on
the application. Therefore, besides tracking properties, this investigation is
also about studying the mechanics of ICC systems, and uncovering how exactly
they provide behavioral guarantees.

Finally, this direction of research contains a bonus\hyp{}challenge. Simply
showing that an ICC-based technique can solve a problem in another application
domain is not enough. It is necessary to show that the solution is
\emph{independently} relevant in the target domain, regardless of the origins of
the solution. Based on experience, this criterion is strongly enforced by the
scientific peer review process.

%! suppress = TooLargeSection
\section{Methodology and Research Deliverables}
\label{sec:aicc-methods}

Following the approaches of~\autoref{sec:aicc-approaches}, we conducted a series
of investigations. These resulted in several
publications~\cite{aubert20222,aubert20232,aubert2023b}, workshop
presentations~\cite{aubert20231,aubert202217,splash22}, and ongoing work.
\IfFileExists{papers}{\documentclass[crop,border={0pt 5pt 0pt 0pt}]{standalone}
\usepackage{tikz}
\usepackage{fontawesome5}
\usepackage{fontspec}
\usepackage{color}
\usetikzlibrary{arrows.meta}
%\usetikzlibrary{backgrounds}
%\usetikzlibrary{arrows,calc,tikzmark,fit,matrix,shapes,arrows,shadows}
%  Augusta Colors
% Cf. http://brand.augusta.edu/color/
% PRIMARY PALETTE
\definecolor{augustablue}{HTML}{002f55}
\definecolor{augustagrey}{HTML}{A5ACAF}
\definecolor{augustaclinical}{HTML}{3CB6CE}
\definecolor{augustaathletic}{HTML}{00AEEF}
\definecolor{augustaaccentgreen}{HTML}{44D62C}
\definecolor{augustaaccentblue}{HTML}{64A0C8}
% Syntax colors
\definecolor{codecomments}{HTML}{455A64}
\definecolor{codekeywords}{HTML}{3D5AFE}
\definecolor{codestrings}{HTML}{00838F}
\setmainfont{TeXGyreTermes}[
    Path = ../fonts/,
    UprightFont = *-Regular,
    BoldFont = *-Bold,
    ItalicFont = *-Italic,
    BoldItalicFont = *-BoldItalic]

\tikzset{font={\selectfont\normalsize}}

\newcommand\authA{Jones \& Kristiansen\@\xspace}    
\newcommand\authB{Aubert, Rubiano, Rusch \& Seiller\@\xspace}    
\newcommand\authC{Moyen, Rubiano \& Seiller\@\xspace}    
\newcommand\authD{Aubert \& Rusch}
\newcommand\authE{Rusch}
\newcommand\authT[2]{\enquote{\textbf{#1}}\\{{#2}}}

\tikzstyle{start} = [
	rectangle, rounded corners=4pt, line width=1.5pt, minimum height=2cm,
	text width=6.5cm, text centered, draw=concept, solid, fill=conceptbase, line cap=round]
	
\tikzstyle{prior} = [start, draw=prior, fill=priorbase, double, line width=.7pt]
\tikzstyle{node} =  [start, draw=paper, fill=paperbase]
\tikzstyle{arrow} = [thick,arrows={-Latex[width=5pt, length=5pt]}]
\tikzstyle{rectangle connector} = [arrow,pos=0.5, to path={(\tikztostart) -- ++(#1,0pt) \tikztonodes |- (\tikztotarget)}]
\tikzstyle{topic icon} = [rectangle, text width=.5cm, yshift=2.3cm, font={\selectfont\large}]
\tikzstyle{topic label} = [rectangle, text width=5cm, yshift=-.25mm, align=left]

\begin{document}
\begin{tikzpicture}[node distance=3cm,align=center,
	block/.style args={#1:#2:#3:#4}{node, 
	name={#1}, draw,
    node contents={\authT{#3}{#4}},
    append after command={
    	node[circle, fill=white, draw=paper, solid, line width=1.5pt,
		inner sep=2pt, outer sep=0.4pt,
        xshift=3mm, yshift=1mm, anchor=north west] at (#1.north west) {\large #2}}}]
   
\node (icc) [start, loosely dotted] at (5,10) {\textbf{Implicit Computational Complexity}};
\node (mwp) [prior, below of=icc, xshift=-3.5cm] {\authT{A Flow Calculus of mwp-Bounds\\for Complexity Analysis}{\authA (2009)}};
\node (qsi) [prior, right of=mwp, xshift=4.3cm] {\authT{Loop Quasi-Invariant Chunk Detection}{\authC (2017)}};
\node [below of=mwp, block=imp:\iconSPA:{mwp-Analysis Improvement and Implementation: Realizing\\Implicit Computational Complexity}:{\authB (2022)}];
\node [below of=imp, block=pym:\iconSPA:{pymwp: A Static Analyzer\\Determining Polynomial Growth Bounds}:{\authB (2023)}];
\node [below of=pym, block=pci:\iconFM:{Polynomial Postconditions\\via mwp-Bounds}:{\authE$^*$}];
\node [below of=qsi, block=ncl:\iconOPT:{Distributing and Parallelizing\\Non-canonical Loops}:{\authB (2023)}];
\node [below of=ncl, block=nil:\iconSEC:{A Logic for\\Anytime Non-Interference}:{\authD$^*$}];
\node [below of=pci, block=cca:\iconFM:{Certifying Complexity Analysis}:{\authB (2023)\\\authE (2023 -- ?)}];

\draw [arrow] (icc) -- (mwp);
\draw [arrow] (mwp) -- (imp);
\draw [arrow] (imp) -- (pym);
\draw [arrow] (pym) -- (pci);
\draw [arrow] (icc) -- (qsi); 
\draw [arrow] (qsi) -- (ncl); 
\draw [arrow] (ncl) -- (nil); 
\draw [rectangle connector=-4cm] (mwp) to (cca);

\node (topics)   [rectangle, below of=nil, text width=5cm,yshift=20pt] {Related Topics};
\node (analysis) [topic icon, xshift=-2.2cm, yshift=-.2cm, below of=topics] {\large\iconSPA};
\node            [topic label, right of=analysis] {Static Program Analysis};
\node (optimize) [topic icon, below of=analysis] {\large\iconOPT};
\node            [topic label, right of=optimize] {Program Optimization};
\node (verify)   [topic icon, below of=optimize] {\large\iconFM};
\node            [topic label, right of=verify] {Formal Methods};
\node (security) [topic icon, below of=verify] {\large\iconSEC};
\node (SL)       [topic label, right of=security] {Security};

   
\fill[draw=concept, line width=1pt, fill=conceptbase, densely dotted] (-2, 12) circle (.75ex) node[right, xshift=3mm]{Concept};
\fill[draw=prior, thick, fill=priorbase, double] (1, 12) circle (.75ex) node[right, xshift=3mm]{Prior foundational work};
\fill[draw=paper, thick, fill=paperbase, solid] (6, 12) circle (.75ex) node[right, xshift=3mm]{Dissertation work};
\node[topic label, text width=1.5cm] at (11, 12) {$\ast$ {preprint}};


\end{tikzpicture}
\end{document}The dependence associations of the
written works are illustrated in~\autoref{fig:papers}.}{}
We can summarize the main contributions as follows.

For RQ1, we enhanced the theory to develop an automatic resource analysis
in~\cite{aubert20222,aubert2023b,rusch2025}. Due to inherent
\ndx{nondeterminism}, developing an efficient and useful
practical analysis was surprisingly difficult, but eventually achievable. The
enhanced flow calculus can answer the same questions as the original
formulation, but automatically and in practice. Moreover, through the
enhancements we discovered how to make the calculus more expressive, allowing it
to cover a larger class of programs than the original theory~\cite{rusch2025}.
The flow calculus is implemented as an open source static analyzer,
\ndx{pymwp}%
\footnote{Pronounced \emph{pa\textsc{i} em double-you p\={e}.}}%
\footnote{https://statycc.github.io/pymwp/}
for analyzing a subset
of C programs~\cite{aubert2023b}. Based on this work, we discovered that the
analysis can support inference of specification conditions~\cite{rusch2025} that
are needed for formal verification. This finding corresponds to RQ3. Proving the
soundness of the flow calculus\index{soundness of mwp-calculus} (RQ2) is still
ongoing work~\cite{aubert20232}. However, given the other developments, a
formalization effort is now even more strongly motivated and justified.

Along the second direction, for RQ4, we applied the QI framework to develop a
program optimization for loop parallelization~\cite{aubert20232}. This is
useful, because the optimization can be applied to loops whose iteration count
is unknown. Parallelizing such loops is outside the capabilities of prior
techniques. In an ongoing investigation, for RQ5, we apply the QI framework to
tracking programs {non\hyp{}interference}~\cite{goguen1982}, a kind of
confidentiality property. Non-interference\index{non-interference} ensures a
program does not reveal secrets; and if it does, we can identify where the
violation occurs. Our working intuition is that, since the analysis can be
adjusted to different programming languages, it enables tracking
\emph{preservation} of the non-interference property at different compilation
stages, while a program is transformed from source code to an executable. This
is different from many existing techniques that analyze programs only at one
stage of representation.

%! suppress = TooLargeSection
\section{Research Findings and Conclusions}
\label{sec:aicc-discussion}

\paragraph*{On automatic resource analysis.} We can now confirm that it is
possible to obtain automatic program analysis from the flow calculus of
mwp-bounds, but only after substantial adjustments. An efficient program
analysis required inventing ways to manage costly sub-computations and solving
additional problems that were exposed in the process. At conclusion, the
enhanced technique is not just automatable, but can analyze more programs than
the original theory. We can now confirm the analysis results are complementary
to alternative state-of-the-art techniques~\cite[p. 5]{aubert2023b}. Thus, it
enables us to analyze different questions about resource consumption. The
implemented flow calculus can also be used in specification
inference\index{specifications}. There are potentially many more applications of
the flow calculus remaining to be discovered.

\paragraph*{On analysis of extended program properties.} The QI framework
provides a case example showing that ICC systems can be flexible enough to track
other program properties. While adjusting the system, we made a surprising
discovery in that changing the property of interest simplified the mathematical
analysis. For example, to track a complexity property, we must compute fixed
points and preserve the program command order. In the adjusted analysis, it was
possible to relax these conditions. More generally, that the QI framework can be
used to track various properties is reminiscent of the \ndx{Dependency Core
Calculus} (DCC)~\cite{abadi1999b}. In DCC, different program properties are
modeled as \emph{instances} of a central notion of \emph{dependency}. This
mirrors our observations about the QI framework. This suggests that the QI
framework encapsulates even wider utility than uncovered so far during this
dissertation work.

\paragraph*{On broader dissertation goals.} Zooming out of the specific
projects, we can reflect on the main hypothesis and the research goals. We have
successfully gathered evidence of the application potential of ICC. The evidence
supports the main hypothesis, but is still too limited to draw generalizing
conclusions. However, our experiences of developing the ICC systems strongly
reinforced the symbiotic view of theory and application. In the process, we
faced many scientific and engineering challenges, and identified compelling
motivations to justify the investigations. At conclusion, we have strengthened
the theories of interest and clarified their practical relevance.

In~\autoref{sec:aicc-goals}, we crafted four goals to define the dissertation
expectations. The first was about extending the applied capabilities of ICC,
which was met successfully. The second was about introducing ICC in development
workflows. Although we have developed a static analyzer, it is a step in
\enquote{isolation.} Integrating ICC analyses into other software development
tools (compilers, verifiers, \etc) remains as an outstanding task. Such
integration is important to promote relevance and continue development of ICC
techniques. Our work in using the flow calculus to infer
specification\index{specifications} conditions~\cite{rusch2025} is a first step
in that direction. The last two goals were social and about disseminating ideas
across research communities. Within the ICC community, our efforts materialized
in seminar talks and informal conversations. Although this record is below
ideal, the ideas in the published works will remain discoverable to others in
written form, asynchronously and in perpetuity, like~\cite{moyen2017}. In regard
to other communities, we were successful at exposing the work at verification,
programming languages, and security venues. Thus, the fourth goal was
satisfactorily met.

\paragraph*{Final thoughts \& future perspectives.} The ICC approach of
guaranteeing program properties demonstrates great applied potential. For
example, if offers a pathway toward formal verification of many non-functional
properties\index{non-functional property}. One research direction, that deserves
more attention, is applying ICC systems to ensure properties by construction.
This would provide guarantees before any program exists and eliminates the need
for \emph{aposteriori} analysis. Although the dissertation research did not
extend that far, exploring this direction is an intriguing future goal. Whether
it is attainable depends crucially on a viewpoint shift. We should regard
\emph{implicit computational complexity} in the broader context it
offers---including as a rich toolbox of techniques for \ndx{static program
analysis}---and continue the exploration of its applied potential.
