%! suppress = TooLargeSection
%! suppress = MissingLabel
%! suppress = NonBreakingSpace
\section{Toward Runtime Quality Guarantees}
\label{sec:aicc-intro}

A gold standard in formal verification is showing that a program is \emph{totally correct}.
Total correctness requires proving two conditions~\cite{leino2023}.
The first is showing that the program meets its specification.
A \emph{specification} is a formally-precise description of program behavior.
Typically, specifications describe \emph{functional} behavior:
a correct program must compute the expected output for all inputs.
The second condition is showing that the program terminates.
Termination guarantees that we will observe the program output \emph{eventually}.
Thus, proving total correctness gives us strong assurance that the program has anticipated behavior at runtime, \ie when the program is executed.

An extended consideration in formal verification concerns \emph{non-functional} properties.
Non-functional properties characterize quality features, like execution latency, power consumption, and information flow security~\cite{terbeek2018}.
Pure functional correctness is {practically unsatisfactory} if a program requires exorbitant time to compute its result, aggressively drains a device battery, or reveals secret information~\cite{heraud2011,aubert20222}.
Thus, engineering high-quality programs requires abilities to formally verify non-functional properties.
Unfortunately, the task is not straightforward.
Among the open challenges are expressing non-functional properties as specifications, and developing techniques to determine whether programs satisfy non-functional properties~\cite{etaps2025}.

The field of \emph{Implicit Computational Complexity (ICC)}~\cite{dallago2011} offers a conceivable pathway toward formal verification of certain non-functional properties.
Broadly, complexity theory focuses on (abstract) reasoning about computational \emph{resource requirements}, and developing a system of \emph{complexity classes} that describes those requirements \wrt a machine model.
In practice, the complexity classes correspond to resources like program running time, memory consumption, \etc
Implicit computational complexity complements classical approach to complexity theory by aiming to discover \emph{machine-independent} characterizations of complexity classes.
ICC systems are designed by introducing a \emph{restriction} at the level of a programming language that guarantees a complexity property.
Then, every program satisfying the restriction is known to belong to a particular complexity class~\cite{pchoux2020}  -- \cf~\autoref{fig:icc_systems}.

Exchanging machine models for syntactic criteria is powerful in multiple ways, including that it facilitates formal verification of programs.
ICC systems enable defining syntactical specifications of resource consumption, and developing fully \emph{automatable} analyses for verifying whether a program satisfies the specification~\cite{heraud2011}.
ICC systems have been designed for numerous combinations of programming languages (imperative,  lambda-calculus, function algebras, quantum, term rewrite systems, \(\pi\)-calculus, \ldots), restriction-techniques (linear logic, data flow analysis, type systems, category theory, \ldots), and complexity classes (\textsc{p}, \textsc{pspace}, \textsc{l}, \textsc{pp}, \ldots)~\cite{baillot2012,moyen2017,pchoux2020}.

\IfFileExists{icc}{\subsection{Implicit Computational Complexity}\label{subsec:icc}

Detailed introductions to ICC can be found in literature
~\cite{dallago2012,moyen2017,pechoux2020,dallago2022}.}{}

\section{Addressed Problem and Goals}
\label{sec:goals}

Besides uses in formal verification, there are many good motivations for reasoning about complexity implicitly.
Notably, ICC allows guaranteeing program properties \emph{by construction}, before any program exists.
ICC drives better understanding of complexity classes and yields more natural definitions and proofs of central results than the classical approach~\cite{kristiansen2017}.
Among the ICC systems, some are expressive enough to write down {natural} algorithms, \eg~\cite{jones2009,marion2011}.
Since the reasoning techniques differ in their foundations, the results they produce are often orthogonal to alternative program analysis techniques~\cite{aubert20222}.
Finally, many ICC systems are {compositional}, \eg~\cite{jones2009,marion2011,hainry2023,atkey2024}.
Compositionality is critical for scalability~\cite{carbonneaux2015} and useful for proving that the analysis itself is correct (\ie soundness)~\cite{keidel2021};
however, compositionality is missed by many other techniques~\cite{carbonneaux2015,schiebel2024}.

Despite this rich inventory of compelling features, implicit computational complexity has remained primarily as a theoretical novelty, with a few exceptions~\cite{avanzini2017,avanzini2008,moyen20172,hoffmann2012,hainry2021,feree2018}.
A theory in absence of applications means the power, limitations, and utilities of the theory are difficult to recognize\footnote{
Reviewer \#2 might even call such theory \enquote{useless}.}.
In our view~\cite[pg. xxxv]{bishop2003}~\cite[p. 75]{moyen2017}, theory and application are \emph{symbiotic} and investigation of both is needed for scientific advancement.

An implied manifesto of this dissertation is that \emph{applications are necessary} to push forward our collective understanding of implicit computational complexity.
Unexplored power and utility is potentially \enquote{locked} in these theoretical systems, and it remains an open problem to release it.
Within the ICC community, more investigations are needed to explore the practical capabilities of the existing techniques.
Externally, the ICC techniques should be exposed to broader research communities.
These actions would motivate further development of ICC systems and facilitate discovery of their use cases.

The work in the dissertation is only a small step toward these ambitious long-term goals.
The realistic expectations and goals concern technical and social advancement.
\begin{enumerate}
\item Extend {applied} capabilities of ICC-based techniques in automatic program analysis and verification.
\item Take ICC techniques a few steps closer to becoming a standard in real world development workflows.
\item Initiate discourse about the relevance of applications within the ICC community.
\item Expose and generate interest in ICC techniques in broader research communities.
\end{enumerate}
A guiding intuition behind these goals is that, \emph{if applied}, implicit computational complexity could provide us practical techniques for formal verification of (many!) non-functional properties.

\section{Research Questions and Approaches}

To make the investigation more concrete, the dissertation defines the following \emph{main hypothesis}.
\begin{quote}
\noindent Implicit computational complexity offers applied utility when lifted outside the theoretical domain.
\end{quote}
Since the hypothesis is not \enquote{research paper-sized}, it must be broken  down further.
The work in the dissertation is organized into smaller investigations along two disjoint {tracks}.
Analogous to programming, we can think of these projects as instantiations of the main hypothesis.
Through the projects, we formulate more specific research objectives that challenge the hypothesis.

\subsection{Extending the flow calculus of mwp-bounds}
The first series of work focuses on the \emph{flow calculus of mwp-bounds}~\cite{jones2009}.
The flow calculus is a canonical example of a theoretical ICC system.
It analyzes imperative programs to determine if it is possible to bound value growth of variables by polynomials.
The motivation to focus on it relies on intuition.
As a syntactical and logic-based data-flow analysis, it \emph{seems like} a good candidate for applications of implicit computational complexity.
Several research questions arise from the flow calculus.
\begin{enumerate}
\item Is it possible to develop automatic static program analysis based on the flow calculus?
\item Is it correct, \ie can we prove \emph{formally} the soundness of the flow calculus?
\item What are the extended use cases of the flow calculus?
\end{enumerate}
The questions are still conventional in that they focus on resource analysis.
In other words, given a complexity-theoretic technique, it is conceivable that we \emph{should} be able to apply such technique to analysis of programs' resource consumption\footnote{
Turns out, this is actually \emph{not} straightforward, but that will be revealed in the manuscripts.},
and prove its soundness based on its paper proofs.
The purpose of the second track of work is to challenge the main hypothesis in more unconventional ways.

\subsection{Applying ICC in extended domains}
The second series concerns applying implicit computational complexity toward tracking \emph{other} semantic non-functional program properties.
The core idea is to take an ICC system and adjust it to a new use case in program analysis.
The motivation is two-fold.
First, it requires deep inspection of the underlying \emph{technique} (rather than a property) and assessing its flexibility.
Second, it requires thinking about implicit complexity outside its usual frame of complexity theory.
Thinking broadly, investigations in this direction are particularly relevant if we wish to expose ICC to wider research communities.

In this line of work, the baseline ICC technique is a data-flow analysis inspired by quasi-invariants (QI) as formulated in~\textcite{moyen20172}.
In that previous work, the technique was used to obtain a compile-time program optimization.
It works by looking for opportunities to lift nested loops outside the containing loop, such that post-transformation the whole-program complexity is reduced.
The technique has many compelling features\footnote{For example, unlike the flow calculus of mwp-bounds, the QI framework supports a richer input language, including arrays.} that make it a suitable candidate for the investigation.
We then ask the following questions about the \emph{extended} uses of QI framework.
\begin{enumerate}
\item How to develop a program transformation to increase parallelization potential?
\item How to use it to analyze security properties of information flow, particularly non-interference?
\end{enumerate}
Although posed here with a sense of obviousness;
a substantial challenge of the research plan is {identifying} potential research domains---like parallel programming or language-based security---where ICC techniques might offer benefit, and how such benefit could be derived compared to the established approaches in those domains.

This track of investigation is no longer concerned with complexity properties.
\Ie we may \enquote{lose} the complexity result in a trade for some other guarantee.
Rather, the investigation aims to uncover whether there is something inherent in the {technique} itself that makes it adjustable to other semantic properties.

Finally, this investigation track embeds a bonus-challenge.
Aside the neat origins, it is not enough to just show that ICC techniques can solve some problems it other application domains.
We must show that the developed solutions are \emph{independently} relevant in the receiving domain.
This criterion is strongly enforced by the peer review process.

\subsection{Methodology}\label{methods}

The identified research questions were investigated in a series of publications~\cite{aubert20222,aubert20232,aubert2023b}, and drafts presented in Sections~\ref{sec:postcond}--\ref{sec:anytime} and ongoing work in \cite{aubert20231}.
We can summarize the main contributions of these publications briefly
\begin{description}
    \item[Implementing the flow calculus] \lipsum[1][1-3]
    \item[Postconditions via mwp-bounds] \lipsum[1][1-3]
    \item[Certifying complexity analysis] \lipsum[1][1-3]
    \item[Distributing non-canonical loops] \lipsum[1][1-3]
    \item[Anytime logic for non-iterference] \lipsum[1][1-3]
\end{description}

\paragraph{Evaluation and artifacts}
% What new understanding, knowledge, methods, or technologies will this research generate?
% What artifacts (tools, theories, methods) will be produced?
% What experiments, prototypes, or studies need to be produced/executed?
% What is the validation strategy?
% How will it show the goal was reached?
% Arguments: Are you going to set up experiments to test these hypotheses?
% If so, how? What are the variables in these experiments?
% How do you plan to control these variables for an unbiased experimental result?
% If not, how else do you plan to convince the field that your hypotheses hold?
The research questions reflect a kind of dynamic process of discovery --
for example, to investigate the applications of the flow calculus requires first to understand its theory deeply.
We can be satisfied that the questions adequately challenge the main hypothesis.
However, that we precisely focus on these questions, techniques, and domains was not predictable in advance.
Conceivably, many other relevant interesting questions remain (\cf~\autoref{future-ops}).
% We limit this presentation to questions we were able to investigate.
Because the work is application-oriented, it expectedly yields software artifacts (\aref{app:sec:artifacts}).
Most notably, the flow calculus of mwp-bounds is implemented as an open source static analyzer,
\href{https://statycc.github.io/pymwp}{pymwp}~\cite{aubert2023b}, for analysis of C programs.

\section{Research results and findings}
\label{subsec:results}

Deriving from the series of work described in~\autoref{methods}, we recognize several findings.

\paragraph{About the flow calculus}
Among the specific approaches, we can now confirm that it is possible to implement the flow calculus of mwp-bounds.
However, the implementation is not obvious and requires careful tuning of the underlying mathematical framework~\cite{aubert20222}.
But adjustment in turn introduces a new problem, and that problem was solved separately only years later~\cite{rusch2025}.
At conclusion, the enhanced flow calculus is applicable and richer than the original system.
This investigation demonstrates the relevance of applications.

\paragraph{On extended applications}
Regarding the quasi-invariant based approaches, we now know that the technique can be applied to track other program properties.
Because it works seamlessly with \pr|while| loops, its application in loop transformations was particularly compelling as
\pr|while| loop constructs are outside the capabilities of related program optimizations.
The same system can be applied to track non-interference, which is a data/information flow property.
The fact that the same system work on two seemingly distant program properties is reminiscent of the Dependency Core Calculus~\cite{abadi1999b},
where different semantic properties are modelled as \emph{instances of dependency}.
This realization suggests that the ICC-based quasi-invariance technique---which is also based on analysis of data flows---possesses wider applicability than yet discovered in this dissertation.

\paragraph{Generated knowledge}
% What new understanding, knowledge, methods, or technologies will this research generate?
% What artifacts (tools, theories, methods) will be produced?
% What experiments, prototypes, or studies need to be produced/executed?
% What is the validation strategy?
% How will it show the goal was reached?
% Arguments: Are you going to set up experiments to test these hypotheses?
% If so, how? What are the variables in these experiments?
% How do you plan to control these variables for an unbiased experimental result?
% If not, how else do you plan to convince the field that your hypotheses hold?

\paragraph{On the hypothesis and broader goals}
Zooming out of individual projects, we can reflect on the main hypothesis.
The findings regarding the main hypothesis are threefold.
\begin{enumerate*}
\item Based on the works on the flow calculus, ICC offers complementary and orthogonal techniques of automatic resource analysis.
However, since the techniques are approximative, convincing other communities of their relevance still remains a challenge.

\item The second demonstrated finding is that it is possible to adjust the ICC techniques to track other semantic properties.
While this direction allows ICC techniques shine outside their usual domain, it requires that the demonstrated application is \emph{actually} relevant, independent of the origin of the technique.

\item The final finding is projective and social.
By restricting the programming languages, ICC offers the potential to achieve desirable properties apriori, via language design.
thus, we can achieve guarantees before any program exists;
never having to run any post-analysis.
While the works in the dissertation did not directly investigate this question, we recognize it as a future goal.
Its attainability depends crucially on a community shift to view ICC in the broader context it offers and the continued exploration of the applied potential of ICC.
\end{enumerate*}

Initially, these research questions were crafter with certain technical and social goals in mind.
%Extend {applied} capabilities of ICC-based techniques in automatic program analysis and verification.
%Take ICC techniques a few steps closer to becoming a standard in real world development workflows.
%Initiate discourse and highlight the relevance of applications within the ICC community.
%Expose and generate interest in techniques of implicit computational complexity in broader research communities.

\section{Conclusions and future work}
