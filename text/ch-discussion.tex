%! suppress = UnresolvedReference
%! suppress = LabelConvention
%! suppress = LabelConvention
% A comprehensive discussion that integrates the findings
% of all research presented in the dissertation
% That identifies how the goals or specific aims of the project were attained.
% How the research has answered the hypotheses.

\section{Technique-Specific Observations}
\label{sec:res-specific}

\subsection{On Advancements of the Flow Calculus}
\label{subsec:res-flow-calc}

The original flow calculus of mwp-bounds provides a purely syntactic, sound and compositional theoretical technique for analyzing value growth of variables in imperative programs.
Although promising, the technique poses various challenges for applications.
In particular, issues arise from nondeterministic inference rules and handling of potential derivation failure.
The developments presented in this dissertation resolve or ease those tensions in several directions.
At conclusion, we obtain an enhanced flow calculus with richer capabilities and utility than the original system.

The essential technical changes involve introducing an \(\infty\)-coefficient (\cf~\aref{sec:fscd}) that tracks derivation failure.
It enables exploring all derivation paths {concurrently}, \ie without the need to back-track during program analysis procedure.
Then,---instead of 0 to exponentially many simple matrices---every analyzed program is characterized by precisely one \emph{complex} mwp-matrix.
Although the introduction of the \(\infty\)-coefficient (and the related technical adjustments) make program analysis automatable,
they also introduce a new problem.
The mwp-bounds, that represent variable value growth, are \enquote{lost} in such a complex matrix until
we discover ways to \emph{interpret} the new complex data structures.

How to completely solve this new problem is described in a second phase of technical developments (\cf~\autoref{sec:postcond}).
The manuscript provides an \emph{evaluation procedure} that enables recovering the mwp-bounds from a complex mwp-matrix.
However, since the matrix contains information about all derivations, it is now possible to also obtain the \emph{optimal} mwp-bounds of each program variable.
A critical step is to \emph{project} mwp-matrices on individual variables.
The projection also enables bounding certain variables in presence of whole-program derivation failure.
This leads to increased expressiveness in abilities to bound more variables than in the original system (\cf~\autoref{tab:pc-results}).

One fascinating side-effect of the enhanced calculus is that it enables extracting \emph{new information} about the analyzed program.
A complex mwp-matrix encapsulates this information, but requires a clever strategy to extract it.
To this end, we view the evaluation procedure as a kind of \enquote{oracle} method, against which we can issue queries (\cf~\autoref{pc-analysis}).
For example, it is possible to pose questions such as the following.
\begin{enumerate}
\item Does a variable always fail the derivation?
\item What variables are sources of failure?
\item By what sequence of derivation choices is a program (not) derivable?
\item What is the optimal mwp-bound assigned to a variable?
\item What is the maximal coefficient of dependency between a pair of variables?
\item Is a pair of variables independent \wrt data flow?
\item Does a derivation exist where many variables obtain optimal mwp-bounds?
\end{enumerate}
All of these questions can be answered in the enhanced calculus.
However, in the original flow calculus, answering \emph{any} of them is either outside the capabilities or would require exhaustive enumeration, which is infeasible.

The developments also reveal deeper insights of the technique.
For example, we now know that the \(0\)-coefficients are interesting, because they track data-flow independence between variables
(refer to~\autoref{dependence-analysis} and~\autoref{subsec:ni} on the relevance of this property).
Although the original formulation of the flow calculus notes that the \(w\)-coefficient is iteration-independent, the significance of that term remained mysterious.
We now know that when a variable is characterized by at most \(w\)-coefficients, the variable is subject to a loop
quasi-invariance property, and will (at some point) become unaffected by increase in loop iteration count (\cf~\autoref{pc-analysis}).
Finally, we also know that focusing the analysis to loops (instead of functions as in the original design) is beneficial for practical analysis.
The modification enables extracting results in some cases where function-based analysis fails (\cf~\autoref{tab:pc-results}).
This reinforces the view that it is justified to limit complexity analyses on loop constructs~\cite{benamram2020}.

To summarize, the enhanced \textsc{mwp}\(^\infty\)-calculus makes the flow calculus automatable for static program analysis,
but is also richer in the information it provides.
The enhanced technique is applicable in automatic analysis of complexity properties of imperative programs, as shown in~\autoref{sec:atva}.
Another use case is providing assistance in formal specification conditions inference (\cf~\autoref{sec:postcond}).
The latter is an unconventional application of complexity-based analyses, but one that \emph{should be} explored more --
the pioneering results of~\autoref{sec:postcond} are encouraging.
At current stage, the flow calculus can derive results efficiently, including for several natural algorithms (\cf~\autoref{tab:pc-results}).
These results are significant to motivate the relevance of formally verifying the core calculus (\cf~\autoref{sec:mwp-calc-formal}).

\subsection{Extended Properties via Quasi Invariants}
\label{subsec:res-qi}

We have applied the framework of quasi invariants in two use cases.
% TODO: general comments, based on those two projects

\subsubsection{Use in Program Optimization}
\label{subsubsec:qi-opt}

In \enquote{Distributing and Parallelizing Non-canonical Loops},
we used the QI framework to design a program optimization to improve programs' runtime execution profile.
The optimization seeks to increase parallelization \emph{potential} in initially sequential imperative loops.
The analysis evaluates \emph{dependence} between commands to discover \emph{independent} sub-computations.
If such independence exists, we replace the original loop with multiple new loops.
The new loops can be distributed between available processors for parallel execution.
The analysis is intended for use as a compiler optimization technique.

The exception attribute of the analysis is that it enables reasoning about
loops whose iteration count is unknown.
Such loops are beyond the capabilities of traditional parallelizing transformations.
However, many ICC systems---including the flow calculus of mwp-bounds---over-approximate loop iterations by assuming them to be infinite\footnote{If a loop terminates earlier, the behavior is still within the over-approximation.}.
This way, the provided guarantees apply across all conceivable iteration behaviors.
The cost of the flexibility is reduced precision;
however, sometimes the trade is beneficial.
For the parallelizing optimization, the property of interest is \emph{independence} of data flow in and between commands.
Therefore, the focus is binary, \ie whether a data flow exists.
Beyond existence, recurrences of data flow are irrelevant.
This enables relaxing the need to know loop iteration count.

Although the goal of the analysis is to improve running time,
the optimization is no longer defined in terms of complexity classes\footnote{
This is different from the prior use of the QI framework~\cite{moyen20172}
that aimed to alter the program's overall complexity by modifying nested loops.}
Thus, in this work, we shift the analysis technique outside complexity theory.
The original complexity result is lost, but we gain a practical program optimization in return.

\subsubsection{Use in Non-Interference}
\label{subsubsec:qi-ni}

In subsequent work (\autoref{sec:anytime}) we have explored
using the QI framework to track non-interference.
Non-interference is a security property for establishing that secret information does not \enquote{leak} to lower security classes (\cf~\autoref{pl-sec}).
Although this property \emph{seems} different from parallelizing loops,
the two applications have a clear commonality.
Both are about data flow independence.
Therefore, the QI framework is suitable for both analyses.

We started developing a non-interference logic, taking as a starting point
the loop parallelization formulation.
A surprising discovery that soon followed, was that we could relax many of the mathematical constraints that were needed previously.
For example, the order of command composition is important if we want to
transform loops and preserve the loop's functional behavior.
For the non-interference analysis, which is read-only, the relevant question is whether an information flow leak occurs.
Once such leak is detected, it is not possible to erase it.
Therefore, the analysis becomes \enquote{composition-insensitive}.
Overall, it was enlightening how changing the evaluated property altered the analysis mechanics.

The non-interference analysis also provides us an opportunity to evaluate a yet another aspect about ICC systems.
Our long-held assumption is that, because the QI framework targets a core imperative \pr|while| language;
it can be applied to any language representation that shares that same core\footnote{The same argument applies to the flow calculus of mwp-bounds.}.
Through non-interference analysis, we can actually challenge this assumption.
We plan to map our non-interference logic to two different language representations, at different stages of compilation.
The intuition is to study whether (or how) the non-interference property is preserved during compilation.
Assuming the idea is achievable, it gives us a different perspective to reason about non-interference.
The typical approach is using a type system, with a singular type check.
We are optimistic that the idea could work, and if so,
we can reinforce the bi-directionality (\cf~\autoref{icc-sec}) between ICC and language-based security.

\section{On Broader Research Findings}
\label{sec:broader-findings}

\subsection{Applications in Automatic Resource Analysis}
\label{subsec:res-resource-analysis}

% What new do we know about ICC now, based on the manuscripts
We can now confirm that it is possible to obtain automatic program analysis from the flow calculus of mwp-bounds, but \emph{only after substantial adjustments}.
An efficient program analysis requires inventing ways to manage costly sub-computations and solving additional problems that arise in the process.
At conclusion, the enhanced technique is not just automatable, but can analyze more programs than the original theory.
We can now also confirm the analysis results a complementary to alternative state-of-the-art techniques~\cite[p. 5]{aubert2023b}, enabling us to analyze different questions about resource consumption.
The implemented flow calculus can also be used in specification inference.
There are potentially many other applications remaining to be discovered.
However, our experience of developing the flow calculus strongly reinforced the symbiotic view of theory and application.
In the process, we faced scientific and engineering challenges, but also identified compelling arguments to justify the investigation.
At conclusion, we have strengthened the theory and clarified its relevance.
% Compositional analysis of subprograms => ICC systems extend to analysis of general programs.

\subsection{Thinking Outside the (Complexity) Box}
\label{subsec:res-other-props}

ICC can perform at any level, and may even be able to prove that some properties are maintained.
Do we want to also give examples of properties that are *not* preserved by the chain;
typically, constant-time properties may be broken by some optimizations.
Through compositional analysis of subprograms, ICC systems extend to analysis of general programs.
Besides complexity, ICC systems can be adjusted to track other semantic properties.
%Research questions 4 and 5 are posed here with an air of obviousness.
%However, arriving to those questions was not obvious.
%A substantial challenge in the research plan involves identifying suitable research domains---like parallel programming or language-based security---
%where ICC techniques could offer meaningful benefit.

\subsection{The \emph{Implicit Costs} of Applied ICC}
\label{subsec:res-meta}

The dissertation manuscripts follow a pattern of pairing ICC with a secondary application domain.
Although such intersectional strategy comes with rich potential, there are also substantial \enquote{costs}.

One cost relates to a learning overhead.
The overhead behind the dissertation manuscripts is reflected in \autoref{sec:pre}.
Making a meaningful contribution requires first sufficient familiarity with secondary domain to identify research gaps.
It also requires learning the terminology and implicit expectations of the secondary domain.
For example, in presenting developments of the flow calculus, we learned through trial-and-error that experiments were an absolute prerequisite to support our findings.

The second cost is about understanding different measures of significance.
While we hold the theoretical origins in high esteem, it becomes minutiae when targeting applications in other domains.
Using an ICC technique to solve a problem in ways that is comparable to existing techniques is not enough for a scientific contribution.
The significance of the application must be standalone and independent of the theoretical origin.
In other words, a technique becomes interesting only after we can show it can solve an interesting problem.

The described research strategy has repeated a few times throughout the dissertation, multiplying these costs.
In reflection, following the strategy requires persistence, and in general not ideal for dissertation research.

\section{Open Problems and Future Opportunities}
\label{sec:res-future}

The ICC approach of guaranteeing program properties demonstrates great applied potential;
for example, toward formal verification of non-functional properties.
One investigative direction that deserves more attention is applying ICC to guarantee properties by construction.
This would provide guarantees before any program exists and eliminates the need for \emph{aposteriori} analysis.
Although the dissertation did not extend that far, exploring this direction is an intriguing future goal.
Whether it is attainable depends crucially on a viewpoint shift.
We should regard ICC in the broader context it offers and continue the exploration of its applied potential.

%limitations/what is left open here/what do we now know
%The ``correct-by-construction'' angle still remains unexplored
%Though: Should explore synthesis over languages that guarantee properties.
%This theme of thinking about complexity outside classical theory is not unique to this dissertation.
%Explorations with a similar theme appears in other recent works in, \eg quantum computing~\cite{avanzini2024}.
%The future of ICC => Quantum etc.
%https://stackoverflow.com/questions/3492188/what-are-the-practical-limitations-of-a-non-turing-complete-language-like-coq
%https://stackoverflow.com/questions/315340/practical-non-turing-complete-languages
%``a general-purpose language that isn't turing complete is pretty impractical.'' or so they say.
%Initial comment on <https://stackoverflow.com/q/315340>
%Thought: (afterward) comment on <https://stackoverflow.com/q/315340>
%Just having Turing-completeness is not super useful (we can show this)
%As language designers we want to add restrinctions... maybe I can find this again to capture the precise comment.

\section{Assessment of Research Goals and Conclusions}
\label{sec:res-summary}

%We have successfully gathered evidence of the application potential of ICC.
%The evidence supports the main hypothesis, but is still too limited to draw any generalizing conclusions.
We crafted four goals to define the dissertation expectations.
The first was about extending the applied capabilities of ICC, which was met successfully.
The second was about introducing ICC in development workflows.
Although we have developed a static analyzer, it is a step in \enquote{isolation.}
Integrating ICC analyses into other development tools (compilers, verifiers, \etc) remains as an outstanding task.
Such integration is important to promote relevance and continue development of ICC {applications}
in particular, since the analyses often produce orthogonal results.
Our work in using the flow calculus to infer specification conditions~\cite{rusch2025} is a first step in that direction.
The last two goals were social, and about disseminating ideas across research communities.
Within the ICC community, our efforts included seminar talks.
Although this record is suboptimal, our written ideas remain discoverable to others asynchronously and in perpetuity, like~\cite{moyen2017}.
In regard to other communities, we were successful at presenting the work at verification, programming languages, formal methods, and security venues.
Thus, the fourth goal was satisfactorily met.

\autoref{fig:findings} summarized the key takeaways presented in long form throughout this discussion.
First, Implicit computational complexity offers complementary techniques of automatic resource analysis.
However, applications of ICC may require substantial adjustment to the theory.
It is important to account for practical challenges in the design of ICC systems.
Second, ICC systems can be adjusted to tracking \emph{other} program properties, beyond complexity theory.
Such adjusted ICC systems can provide analysis techniques in cases that are outside the capabilities of the established techniques.
Finally, attaining broader applied utility from ICC requires a viewpoint shift \wrt its capabilities---\ie fundamentally seeing it as more than tool for complexity-theoretic reasoning---and the continued exploration of its applied potential.

\begin{figure}[h]
\begin{mdframed} % TODO: mirror the hypothesis
\paragraph*{Primary findings.}
\begin{enumerate}[wide, labelwidth=!, labelindent=0pt]
\item ICC offers complementary techniques of automatic resource analysis.
\item ICC systems can be adjusted to tracking \emph{other} program properties, beyond complexity theory.
\item Attaining rich intercommunal utility from ICC requires viewing it in broadly and continuing exploration of its applied potential.
\end{enumerate}
\end{mdframed}
\caption[Primary research findings summarized]{Primary research findings.}
\label{fig:findings}
\end{figure}
