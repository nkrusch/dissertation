Multiple individuals deserve repeated recognitions for their impact in my doctoral journey.
However, in interest of brevity, I have adopted a principle from \emph{linear types}: mentioning each name exactly once.\footnote{Except Clément, because his actions and influence require multiple mentions.}
Let that principle not diminish the significance of anyone's role. % in this achievement.

\noindent\hfil\rule{0.25\textwidth}{.4pt}\hspace{1em}\raisebox{-0.5\height}{\SixFlowerAltPetal}\hspace{1em}\rule{0.25\textwidth}{.4pt}\hfil

\paragraph*{Family.}
Kiitos, Eila ja Lefa.
Oli kiva nähdä ja toivottavasti tavataan taas.
I am thankful to my children, Ileana and Tommy.
They both influenced my doctoral studies in their own ways.
I am incredibly proud of both of you.
Tank, my fondest companion and best buddy, was with me every year I spent at Augusta University, 2011--2025.

\paragraph*{Academic thanks.}
My history at {Augusta University} is extensive.
It involves three institutional names\footnote{Augusta State University, Georgia Regents University, and Augusta University.},
various roles, learning moments, and interactions with many exceptional individuals.
My current achievements are thanks to the last one.
In these acknowledgements, I aim to capture everyone who contributed overall positively to this outcome.\footnote{Though, I suspect I still end with a sound under-approximation.}

From my undergraduate years, 2011--2014, I appreciate every professor who was teaching computer science at the time: Joanne Sexton, Mike Dowell, Onyeka Ezenwoye, and Paul York.
From them, I learned the foundations of my discipline.
My undergraduate experience was encouraging, nurturing, and provided the skills that helped me succeed later, at Georgia Institute of Technology and various companies.
I am thankful to Joseph Hauger, Tom Colbert, and Sam Robinson (Professor Emeritus), from the College of Science and Mathematics, for supporting me financially during those years through the Savannah River Scholar's Program.
I am also thankful for the instruction of Predrag Punoševac as I understood the value of his teachings only years later.
Finally, I appreciate Jurgen Brauer (Professor Emeritus) and Walter Evans (Professor Emeritus),
for creating me opportunities to develop my software engineering skills before I even graduated.

After an enriching hiatus, I returned to Augusta University in 2019.
First as an instructor and soon as a doctoral student.\footnote{In 2020 or 2021, depending on where we set the official start.}
This return was prompted in response to the many invitations and encouragement I received (again) from my undergraduate professors.
I recognize and thank Steve Weldon for his role in supporting this re-entry.
I appreciate my fellow pioneering students, James O'Meara and William Cocke, who were present with me in those early days of the computer science doctoral program, in the spring of 2021.

The years of my doctoral study were made pleasant and memorable thanks to the many friends and colleagues I met at the \emph{Palazzo}\footnote{The name was inspired by Konstantin Britikov and Rodrigo Otoni, and validated by Simone Gazza.} in Summerville.
I thank Peter Hanukaev for his friendship, and leadership of $\Delta\Lambda\Delta$, and organizing with me the programming languages reading group.
I thank Deivid Vale, Gabriele Cecilia, Jason Weeks, Mahady Hassan, Mark Holcomb, and Vignesh Sivakumar for the shared experiences of the reading group and many social events.
A special recognition to my fellow Palazzo ladies and friends.
Zain Haloush, I will always remember the \enquote{escape mission} before the movie Millennium Actress.
Nour Alhussien, my special friend, thank you for everything -- including the many delicious Arabic foods I got to sample.
I am thankful for the many magical moments we all shared as students.
I am excited for all our futures and learning what happens in the next chapters.

Being in the first cohort in a doctoral program was minimally a character-building experience.
It required much resilience and tenacity, since various policies were not in place, or were not yet fitted for computer science.
I accumulated many \enquote{firsts} and uncovered several edge cases in the program logistics, which I hope will be resolved over time.
I appreciate Gagan Agrawal, for his commitment to student success in the early days of the program, and his direct support to me at various stages of my doctoral study.
As a testament to Gagan's momentous influence---although he left Augusta University two years ago, just two years after the launch of the program---%
he had a direct impact on the successes of three students that were first to defended their dissertations.

During these turbulent times, I also greatly appreciate the stability, peace, and support I found at The Graduate School (TGS).
I particularly recognize the strong leadership, commitment, and consistency in action of the TGS Dean, Jennifer Sullivan.
I appreciate Patricia Cameron, for her responsiveness and support, especially during the Three Minute Thesis competition.
Finally, I want to thank Emily Crider, for the many prompt responses and overall support of the graduate students.
Also thanks to The Graduate School for hosting the many wonderful events that fueled and nourished the students of the Palazzo.

Across other departments, I am thankful for the supportive and positive staff at the Center of Writing Excellence,
where I worked with Brilynn Janckila, Candis Bond, Hannah Soblo, Makayla Mathews, and Romana Hinton.
I also appreciate the staff at the Reese Library, which is my special retreat in Summerville.
I am thankful for all the times I had the privilege of attending lectures of John Hayes of Pamplin College of Arts.
Finally, I greatly appreciated the numerous performance I experienced at the Maxwell Performing Arts Theatre, especially the many organized by Matthew Buzzell.

But, beyond friends and events, doctoral study is foremost a phase of individual growth.
Easily the most influential person in my journey was my advisor and scientific mentor, Clément Aubert.
I started my doctoral study without research experience, but through his guidance developed an appreciation for the elegance and beauty of theoretical computer science.
Through years of mentoring and countless conversations, he helped me grow, and instilled the research skills that will carry me in the future.
It saddens me greatly that our interactions are inevitably ending, and that I am completely incapable of adequately expressing my gratitude for the past.
The Augusta University community is infinitely enriched by your presence and mentorship.
I hope many more students get to experience it.

In research, I am thankful for my collaborators, Thomas Rubiano and Thomas Seiller, for the many projects we completed together.
In my mind, our projects represent the ideal model of scientific collaboration, and I am happy to have experienced that model early.
I am also thankful to the members of my doctoral committee.
Bogdan Chlebus, for his impact in those early days of the doctoral program, and for the guidance that led me to finding the best advisor.
Harley Eades III, for instituting programming languages research in Augusta and for introducing me to functional reactive programming.
Martin Avanzini, for graciously hosting me in Sofia Antipolis, and for the many fascinating conversations about the flow calculus.
Yuyan Bao, for the many thoughtful conversations, reading group experiences, and our many shared experiences.

Outside Augusta, my doctoral journey was greatly elevated by the many fascinating people I met on my travels.
Besides the names already mentioned, I appreciate the following individuals %
-- for the conversations, guidance, inspiration, memories, shared adventures, and friendship --
that shaped my doctoral experience.

Ahmed K. Zaher%
, Alaia Solko-Breslin%
, Alan Jeffrey%
, Alfons Laarman%
, Amirali Ebrahim\hyp{}zadeh%
, Anssi Yli-Jyrä (for the reissumies)%
, Arjun Ramesh%
, Ármin Zavada%
, Arthur Correnson%
, Brigitte Pientka%
, Bryan Richlinski%
, Caleb Stanford%
, Chengsong Tan%
, Chris Brzuska%
, Christine Rizkallah%
, Delphine Demange%
, Dhanushka Jayasuriya%
, Dionysios Spiliopoulos%
, Elliot Bobrow%
, Eric Cornelissen%
, Ewen Denney%
, Fabian Muehlboeck%
, Haoyi Zeng%
, Ignacio Ballesteros%
, Jared Pincus (for Computer Junkyard)%
, Jonathan Aldrich%
, João C. Pereira%
, Judith Perera%
, Julian Loss%
, Jérémy Thibault%
, Krystal Maughan%
, Laura Kovacs%
, Leona Odole%
, Lisa Oakley%
, Long Pham%
, Luca Maio%
, Mandana Farhang Ghahfarokhi%
, Marsha Chechik%
, Mihai Nicola%
, Mishel Carelli%
, Mohammed Foughali%
, Natarajan Shankar%
, Norine Coenen%
, Onur Şahin%
, Pamela Zave%
, Paul Patault%
, Remi Desmartin%
, Romain Péchoux%
, Ross Horne%
, Rustan Leino%
, Santiago Bautista%
, Shengyu Huang%
, Shriram Krishnamurthi%
, Sumit Gulwani%
, Thomas Lamiaux%
, Tim Nelson%
, Tobias Nießen%
, Wilf Offord%
, Yannick Forster%
, and Zahra Moezkarimi%
.

I hope we meet again, sometime, somewhere!


\paragraph*{Recognitions of support in dissertation creation.}
This dissertation benefits from many instances of open source software, developed benevolently by contributors around the world.
I thank the development teams of the many excellent scientific programming tools---like the Rocq Prover, Dafny, and Python---that I regularly use in my research.

I thank Clément Aubert for the initiative and action of preparing the general \href{https://github.com/the-au-forml-lab/au_ccs_dissertation_template}{\LaTeX{ }template for doctoral dissertations},
and making it open source.
Now it benefits all students of The Graduate School at Augusta University.

I appreciate Saverio Giallorenzo, Fabrizio Montesi, and Marco Peressotti, for their artistry in styling code blocks in~\cite{giallorenzo2024},
and for sharing those sources on arXiv.
The same style is used throughout this dissertation.

I was greatly impressed by the work of Swaraj Dash, Younesse Kaddar, Hugo Paquet, and Sam Staton in~\cite{dash2023};
and its companion artifact, the Haskell library LazyPPL~\cite{dash2023b}.
The presentation influenced the preparation of the pymwp tool user guide that appears in~\autoref{sec:toolguide}.

Viimeisenä iso kiitos Tonille ja Sipelle monista hienoista biiseistä jotka säesti tämän väitöskirjan kirjoitushetkiä.
Nyt ei pitäisi olla puutteita koodissa!

\paragraph*{Acknowledgements of financial research support.}
During my doctoral study, my scientific development was greatly enhanced by numerous travels and participation in professional events.
Attending these events was largely made possible though financial support by the following agencies (in alphabetic order).
The list is complete up to the date of writing these acknowledgements, which concludes the support that in any way influenced the dissertation research.
The remainder of the funding was supplied by me.

The ACM Special Interest Group on Programming Languages (SIGPLAN)%
, AFCEA Educational Foundation%
, Association Internationale pour les Technologies Objets (AITO) e.V.%
, Booz Allen Hamilton%
, Computing with Infinite Data (CID) programme of the European Commission%
, Department of Informatics at the Technische Universität München%
, ETAPS e.V.%
, Graduate Student Government Association (GSGA) at Augusta University%
, Institute of Electrical and Electronics Engineers (IEEE)%
, NATO Science for Peace and Security Program%
, National Science Foundation (NSF)%
, Office of the Provost at Augusta University%
, the Programming Methodology Group at ETH Zürich%
, SRI International%
, The Graduate School (TGS) at Augusta University%
, Udo Keller Stiftung%
, fortiss GmbH%
, organizers and sponsors of the mentoring workshop at the International Conference on Computer Aided Verification (CAV) 2023%
, organizers and sponsors of the European Conference on Object-Oriented Programming (ECOOP) 2025%
, organizers and sponsors of the European joint conferences on Theory and Practice of Software (ETAPS) 2024%
, organizers and sponsors of the International Conference on Verification, Model Checking, and Abstract Interpretation (VMCAI) 2022%
, organizers and sponsors of the International Programming Language Implementation Summer School (PLISS) 2025%
, the Japan Society for the Promotion of Science (JSPS) Core-to-Core Program%
, the Transatlantic Research Partnership%
, and the Translational Research Program (TRP) of the Department of Medicine at the Medical College of Georgia (MCG).