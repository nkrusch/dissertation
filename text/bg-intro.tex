%Literature review and discussion of the rationale of the project.
%It is expected that the literature review will be more comprehensive
%than those presented in the included publications.

Research papers are often written with a strategy where they arrive to the new discussion as soon as possible.
This style comes from awareness of an imposed page limit, but also from the need to keep the readers interested.
Thus, in writing a research paper, we make certain assumptions about the readers' existing knowledge and sacrifice extensive explanation of the basics.
This description is also accurate for the manuscripts of this dissertation.

However, the dissertation format does not suffer from the constraints of a regular research paper.
There is no page limit, nor a compelling reason to omit comprehensive technical background.
Since it took nearly five years to internalize and conclude the work presented in this dissertation, it would be naive to assume every reader is familiar with the same foundations.
Therefore, this section includes the technical background the reader should be familiar with to enjoy the manuscripts that will follow.

\begin{table}[h!]
\begin{NiceTabularX}{\linewidth}{@{}Xcr@{}}
\toprule
\textbf{Manuscript Title} & \textbf{Section} & \textbf{Background} \\
\midrule
{mwp-Analysis Improvement and Implementation\ldots} & \aref{sec:fscd}
& \ref{icc}, \ref{static-analysis}, \ref{flow-calculus} \\
{Distributing and Parallelizing Non-canonical Loops} & \ref{sec:vmcai}
& \ref{transforms} \\
{pymwp: A Static Analyzer Determining Polynomial\ldots} & \ref{sec:atva}
& \ref{icc}, \ref{static-analysis}, \ref{flow-calculus} \\
{A Logic for Anytime Non-Interference} & \ref{sec:anytime}
& \ref{pl-sec} \\
{Polynomial Postconditions via mwp-Bounds} & \ref{sec:postcond}
& \ref{flow-calculus}, \ref{verification} \\
{Certifying Complexity Analysis} & \ref{sec:mwp-calc-formal}
& \ref{flow-calculus}, \ref{verification} \\
\bottomrule
\end{NiceTabularX}
\caption[Manuscript background dependency association]{Manuscript background dependency association.}
\label{tab:paper-bg}
\end{table}

The background is organized thematically.
The subsections are (mostly) self-contained literature reviews, introducing the identified topic.
Although it is recommended to read first~\autoref{icc}, the other sections can follow in any order.
It is also possible to initially skip the background, and return to this section when a manuscript creates a need to understand the broader context.
Refer to~\autoref{tab:paper-bg} for guidance about connecting the dissertation manuscripts with the background topics.
