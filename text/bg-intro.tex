For research papers, it is often optimal to arrive to new findings as soon as
possible. This need is motivated by a page limit, but also the need to keep
readers interested. Thus, it is natural to make certain assumptions about the
readers' prior knowledge and sacrifice extensive explanations of the basics.
This description also applies to the manuscripts bundled in this dissertation.

However, the dissertation does not have the same constraints as a research
paper. There is neither a page limit nor a compelling reason to omit technical
foundations. Since it took several years to internalize and complete the work
presented in this dissertation, it seems unlikely every reader is familiar with
the same foundations. Therefore, this section presents the technical background
needed to enjoy to full extent the manuscripts that will follow.

\begin{table}[h!]
%! suppress = EscapeAmpersand
\begin{NiceTabularX}{\linewidth}{@{}Xcc@{}}
\toprule
\textbf{Manuscript Title} & \textbf{Section} & \textbf{Background} \\
\midrule
{mwp-Analysis Improvement and Implementation\ldots}
    & \ref{sec:fscd}
    &~\ref{icc},~\ref{static-analysis},~\ref{flow-calculus} \\
{pymwp: A Static Analyzer Determining Polynomial\ldots}
    &~\ref{sec:atva}
    &~\ref{icc},~\ref{static-analysis},~\ref{flow-calculus} \\
{Distributing and Parallelizing Non-canonical Loops}
    &~\ref{sec:vmcai}
    &~\ref{transforms} \\
{Polynomial Postconditions via mwp-Bounds}
    &~\ref{sec:postcond}
    &~\ref{flow-calculus},~\ref{verification} \\
{A Logic for Anytime Non-Interference}
    &~\ref{sec:anytime}
    &~\ref{pl-sec} \\
{Certifying Complexity Analysis}
    &~\ref{sec:mwp-calc-formal}
    &~\ref{flow-calculus},~\ref{verification} \\
\bottomrule
\end{NiceTabularX}
\caption[Manuscript background dependency associations]
{Manuscript background dependency associations.}
\label{tab:paper-bg}
\end{table}

The background is organized thematically. The subsections are (mostly)
self-contained introductions to the specified topic. It is advisable to read
first~\autoref{icc}, but the other sections can follow in any order. It is also
possible to initially skip the background and return to this section when a
manuscript creates a need to understand its foundations. Refer
to~\autoref{tab:paper-bg} for guidance on how to connect the dissertation
manuscripts with the background topics.
