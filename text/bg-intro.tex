For research papers, it is often optimal to arrive to new findings as soon as
possible. This need comes from awareness of a page limit but also from the need
to keep the readers interested. Thus, it is natural to make certain assumptions
about the readers' prior knowledge and sacrifice extensive explanation of the
basics. This description also applies to the manuscripts included in this
dissertation.

However, the dissertation does not have the same constraints as a research
paper. There is neither a page limit or a compelling reason to omit technical
background. Since it took several years to internalize and complete the work
presented in this dissertation, it would be naive to assume every reader is
familiar with the same foundations. Therefore, this section includes the
technical background that the reader should be familiar with to enjoy the
manuscripts that will follow.

\begin{table}[h!]
%! suppress = EscapeAmpersand
\begin{NiceTabularX}{\linewidth}{@{}Xcr@{}}
\toprule
\textbf{Manuscript Title} & \textbf{Section} & \textbf{Background} \\
\midrule
{mwp-Analysis Improvement and Implementation\ldots}
    & \aref{sec:fscd}
    &~\autoref{icc},~\autoref{static-analysis},~\autoref{flow-calculus} \\
{pymwp: A Static Analyzer Determining Polynomial\ldots}
    &~\autoref{sec:atva}
    &~\autoref{icc},~\autoref{static-analysis},~\autoref{flow-calculus} \\
{Distributing and Parallelizing Non-canonical Loops}
    &~\autoref{sec:vmcai}
    &~\autoref{transforms} \\
{Polynomial Postconditions via mwp-Bounds}
    &~\autoref{sec:postcond}
    &~\autoref{flow-calculus},~\autoref{verification} \\
{A Logic for Anytime Non-Interference}
    &~\autoref{sec:anytime}
    &~\autoref{pl-sec} \\
{Certifying Complexity Analysis}
    &~\autoref{sec:mwp-calc-formal}
    &~\autoref{flow-calculus},~\autoref{verification} \\
\bottomrule
\end{NiceTabularX}
\caption[Manuscript background dependency associations]
{Manuscript background dependency associations.}
\label{tab:paper-bg}
\end{table}

The background is organized thematically. The subsections are (mostly)
self-contained introductions to the specified topic. It is advisable to read
first~\autoref{icc}, but the other sections can follow in any order. It is also
possible to initially skip the background, and return to this section when a
manuscript creates a need to understand its foundations. Refer
to~\autoref{tab:paper-bg} for guidance on how to connect the dissertation
manuscripts with the background topics.
