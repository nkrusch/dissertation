\newcommand\CTNT{Clément Aubert, Thomas Rubiano, Neea Rusch, Thomas Seiller}
\newcommand\ainfo[2]{\noindent{#1}\newline\noindent{#2}\newpage}
\newcommand\ainfoX[3]{\noindent{#1}\newline\noindent{#2}\vspace{1em}\newline{#3}\newpage}
\newcommand\abspage[1]{\noindent\textbf{Abstract.}{ }{#1}\newpage}
\newcommand{\nocontentsline}[3]{}
\newcommand{\tocless}[2]{\bgroup\let\addcontentsline=\nocontentsline#1{#2}\egroup}
\newcommand{\aref}[1]{\hyperref[#1]{Appendix~\ref*{#1}}}
\newcommand{\swlink}[2]{\texttt{\footnotesize{\href{#1}{#2}}}}
\newcommand{\ccx}[1]{\textsc{#1}}
\newcommand{\nonterm}[1]{#1}

% apa-style citation fix
\renewcommand{\cite}[2][]{\autocite[#1]{#2}}

% paper category icons
\newcommand{\pageIcon}[2]{
\raisebox{-0.3\height}{\begin{tikzpicture}
\node[circle,inner sep=2pt,outer sep=.4pt,fill=white,draw=black,solid,thick]{#1};
\end{tikzpicture}}\hspace{1em}\textit{Implicit computational complexity \& #2}\par}
\newcommand{\pageIconAnalysis}{\pageIcon{\iconSPA}{static analysis}}
\newcommand{\pageIconFm}{\pageIcon{\iconFM}{formal methods}}
\newcommand{\pageIconOpt}{\pageIcon{\iconOPT}{program optimization}}
\newcommand{\pageIconSecurity}{\pageIcon{\iconSEC}{security}}

% add index terms to \includepdf
\makeatletter
\newcommand\addtoindex[2]{%
\ifnum\AM@page=#2\relax\index{#1}\fi
}
\makeatother

% add symbols to \includepdf
\makeatletter
\newcommand\addtosymbols[2]{%
\ifnum\AM@page=#2\relax\gls{symb:#1}\fi
}
\makeatother
\newcommand\symbo[1]{\gls{symb:#1}}

% Fix \sc undefined
\providecommand{\sc}{}
\renewcommand{\sc}[1]{#1}

% tabularx centered maximally-wide column
\newcolumntype{C}{>{\centering\arraybackslash}X}

% Math operators etc.
%\newcommand\hypo{\Hypo}
%\newcommand\infer{\Infer}
\newcommand{\zmat}{\mathbf{0}}
\newcommand{\umat}{\mathbf{1}}
\DeclareMathOperator{\In}{In}
\DeclareMathOperator{\Out}{Out}
\DeclareMathOperator{\PrD}{PrD}
\DeclareMathOperator{\id}{id}
\DeclareMathOperator{\Id}{Id}
\DeclareMathOperator{\var}{var}
\DeclareMathOperator{\Card}{Card}
\DeclareMathOperator{\Occ}{Occ}
\DeclareMathOperator{\poly}{poly}
\renewcommand{\gets}{=} % algorithms

\DeclarePairedDelimiter{\sem}{\llbracket}{\rrbracket} % Semantics of program.
\newcommand{\BNF}{\enspace \ensuremath{\Vert} \enspace} % BNF separator
\newcommand{\mat}[1]{\left(\begin{smallmatrix}#1\end{smallmatrix}\right)} % Shorthand for matrices
\ebproofnewstyle{small}{separation = 1em, rule margin = .5ex}

% Styles
\theoremstyle{remark}
\newtheorem{thm}{Theorem}
\newtheorem{corollary}[thm]{Corollary}
\newtheorem{lemma}[thm]{Lemma}
\newtheorem{definition}{Definition}
\newtheorem{conjecture}[thm]{Conjecture}
\newtheorem{proposition}{Proposition}[section]
\newtheorem{claim}{Claim}
\newtheorem{remark}{Remark}
\newtheorem{notation}{Notation}
\newtheorem{example}{Example}
\newtheorem{examples}{Examples}
\newtheorem{remarks}{Remarks}
\newtheorem{facts}{Facts}
\newtheorem{fact}{Fact}

% Custom terms
\newcommand\mwp{{mwp}\xspace}
\newcommand{\mwpsc}{\textnormal{\textsc{mwp}}\xspace}
\newcommand{\vdashJK}{\vdash_{\textnormal{\textsc{jk}}}}
\newcommand{\DFG}{\textsc{dfg}\xspace}
\newcommand{\dfg}[1]{\mathbb{M}(\comm{#1})}
\newcommand{\dfgtilde}[1]{\mathbb{\tilde{M}}(\comm{#1})}
\newcommand{\scc}{\textsc{scc}\xspace}
\newcommand{\sccs}{\textnormal{{\scshape SCC}s\xspace}}
\newcommand{\SFM}{\textsc{sfm}\xspace}
\newcommand{\SFMs}{\textsc{sfm}s\xspace}
\newcommand{\sfm}[1]{\mathbb{M}(\comm{#1})}
\newcommand{\condcorr}[1]{\mathrm{Corr}^{\mathtt{if}}(#1)}
\newcommand{\whilecorr}[1]{\mathrm{Corr}^{\mathtt{while}}(#1)}
\newcommand{\corr}[1]{\mathrm{Cr}(#1)}
\newcommand{\corrc}[1]{\mathrm{Corr}(#1)}
\newcommand{\Var}{\mathrm{Vars}}
\newcommand{\lvl}[1]{\ell(#1)}
\newcommand{\SC}{\mathrm{SC}}
\newcommand{\LH}{\mathrm{LH}}
\newcommand{\HMO}{\mathrm{HMO}}
\newcommand{\scl}[1]{\ensuremath{\mathit{#1}}}
\newcommand{\SSG}{\mathrm{SSG}}
\newcommand{\nv}{\cdot} % no violation
\newcommand{\vi}{\textnormal{\scalebox{.65}{\faTint}}} %\usym{1F322}} % Violation

% Comma after eg and ie
\newcommand*{\eg}{e.g.\@,\xspace}
\newcommand*{\Eg}{E.g.\@,\xspace}
\newcommand*{\cf}{cf.\@\xspace}
\newcommand*{\ie}{i.e.\@,\xspace}
\newcommand*{\aka}{a.k.a.\@\xspace}
\newcommand*{\Ie}{I.e.,\@\xspace}
\newcommand{\stt}{s.t.\@\xspace}
\newcommand*{\resp}{resp.\@\xspace}
\newcommand*{\wrt}{w.r.t.\@\xspace}
\newcommand*{\wlg}{w.l.o.g.\@\xspace}
\renewcommand{\aka}{a.k.a.\@\xspace}
\newcommand{\stsup}{$^\text{st}$\@\xspace}
\newcommand{\ndsup}{$^\text{nd}$\@\xspace}
\newcommand{\thsup}{$^\text{th}$\@\xspace}
\makeatletter
\newcommand*{\etc}{%
    \@ifnextchar{.}%
{etc}%
{etc.\@\xspace}%
}
\makeatother