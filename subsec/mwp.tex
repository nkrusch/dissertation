\subsection{\mwp-Bounds Analysis}\label{subsec:mwp}

The \mwp-flow analysis~\cite{jones2009} is a static analysis technique for evaluating resource usage of imperative programs.
It analyzes input variables' value growth, and aims to discover a polynomially bounded data-flow relation between variables \emph{initial} values and \emph{final} values. 
When all variables are bounded by polynomials in inputs, the analysis succeeds, \ie a program is derivable in the underlying calculus.
If one or more variables is not bounded, such bound cannot be established.
The soundness theorem of the analysis guarantees that if a derivation exists, the program's value growth is polynomially bounded in inputs. Furthermore, as a sound technique, it offers a computational method to certify programs data size growth properties at runtime.

%! suppress = EscapeUnderscore
%! Author = neearusch
%! Date = 6/30/23

\begin{figure*}[!h]

To develop intuition of what \mwp-flow analysis computes, consider the following example.
Let \pr{$\texttt{C}^{\prime} \equiv$ X1:=X2+X3; X1:=X1+X1} and
\pr{$\texttt{C}^{\prime\prime} \equiv$ X1:=1; loop X2 {X1:=X1+X1$\texttt{\}}$} be imperative programs with standard operational semantics.
For each variable \pr{X}$_i$, let $x_i$ denote its initial value and $x_i^\prime$ its final value.

\begin{itemize}    
\item Program \prc{C}$^\prime$. Observe by inspection that variable \pr{X1}'s final value is $x_1' \leq 2x_2 + 2x_3$. Variables \pr{X2} and \pr{X3} do not change and therefore are bounded by their initial values $x_2' \leq x_2$ and $x_3' \leq x_3$. We conclude all variables have a polynomial growth bound, and the program has the property of interest.
\item Program \prc{C}$^{\prime\prime}$. Since variable \pr{X2} does not change, its growth bound is $x_2' \leq x_2$. However, \pr{X1} grows exponentially with bound $x_1' \leq 2^{x_2}$. We conclude program does not have a polynomial growth bound.
\end{itemize}

    % \caption{}
    \label{fig:pbounds}
\end{figure*}



Instead of manual inspection, the \mwp-flow analysis provides an \emph{automatable} technique to distinguish programs that have polynomial growth bounds. 
Internally is works my applying inference rules to program commands and tracking variable dependencies by coefficients (or \emph{flows}).
A detailed description of the system is provided in the original work by Jones and Kristiansen~\cite{jones2009} and refined by Aubert et. al~\cite{aubert2022b}. The remainder of this section elaborates on the central notions and terminology to understand the mechanics of the analysis, as it relates to the publications in \autoref{sec:published-manuscripts}.

