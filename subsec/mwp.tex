\subsection{\mwp-Bounds Analysis}\label{subsec:mwp}

Blah~\cite{jones2009}

%The \mwp-flow analysis~\cite{jones2009} is one example of an ICC-based system.
%It is a sound computational method that certifies polynomial bounds on the size of the values manipulated by an imperative program.
%It provides a certificate guaranteeing that the program uses throughout its execution at most a polynomial amount of space and, if a program terminates, it will do so in polynomial time in the size of its inputs.
%The analysis computes, for each program variable, a vector tracking how it depends on other variables.
%The vector values are determined by applying the nondeterministic rules of the \mwp-calculus to the commands of the program.
%Those vectors are collected in a matrix.
%A program is assigned a matrix only if all the values in it are bounded by a polynomial in the input sizes.
%This technique is compositional and operates on a simple, imperative language.

%! suppress = EscapeUnderscore
%! Author = neearusch
%! Date = 3/22/23

\begin{figure*}
    \begin{centering}
    \begin{tabularx}{\textwidth}{c Y c}
    \begin{prooftree}[small]
        \infer0[E1]{ \vdashJK \text{\pr|Xi|} : \{_{\text{\pr|i|}}^{m}\}}
    \end{prooftree}
    &
    \begin{prooftree}[small]
        \infer0[E2]{ \vdashJK \text{\pr{e}} : \{ _{\text{\pr|i|}}^{w} \mid \text{\pr|Xi|} \in \var(\text{\pr{e}}) \}}
    \end{prooftree}
    &
    \begin{prooftree}[small]
        \hypo{\vdashJK \text{\pr{Xi}} : V_1}
        \hypo{\vdashJK \text{\pr{Xj}} : V_2}
        \infer[left label={\(\star\in\{+, -\}\)}]2[E3]{\vdashJK \text{\pr|Xi $\star$ Xj|} : pV_1 \oplus V_2}
    \end{prooftree}
    \end{tabularx}
    \\[2em]
    \begin{tabularx}{\textwidth}{c Y c}
    \begin{prooftree}[small]
        \hypo{\vdashJK \text{\pr{Xi}} : V_1}
        \hypo{\vdashJK \text{\pr{Xj}} : V_2}
        \infer[left label={\(\star\in\{+, -\}\)}]2[E4]{\vdashJK \text{\pr|Xi $\star$ Xj|} : V_1 \oplus pV_2}
    \end{prooftree}
    &
    \begin{prooftree}[small]
        \hypo{ \vdashJK \text{\pr{e}} : V}
        \infer1[A]{\vdashJK \text{\pr|Xj = e|} : \umat \xleftarrow{\text{\pr|j|}} V}
    \end{prooftree}
    &
    \begin{prooftree}[small]
        \hypo{ \vdashJK \text{\pr{C1}} : M_1}
        \hypo{ \vdashJK \text{\pr{C2}} : M_2}
        \infer2[C]{\vdashJK \text{\pr{C1; C2}} : M_1 \otimes M_2}
    \end{prooftree}
    \end{tabularx}
    \\[2em]
    \begin{prooftree}[small]
        \hypo{ \vdashJK \text{\pr{C1}} : M_1}
        \hypo{ \vdashJK \text{\pr{C2}} : M_2}
        \infer2[I]{\vdashJK \text{\pr|if b then C1 $\text{\hspace{.3em}}$ else C2|} : M_1 \oplus M_2}
    \end{prooftree}
    \\[2em]
    \begin{prooftree}[small]
        \hypo{\vdashJK \text{\pr|C|} : M}
        \infer[left label={\(\forall i, M_{ii}^* = m\)}]1[L]{\vdashJK \text{\pr|loop X$_\ell$ \{C\}|} : M^* \oplus \{_{\text{\pr|$\ell$|}}^{p}\rightarrow j \mid \exists i, M_{ij}^* = p\}}
    \end{prooftree}
    \\[2em]
    \begin{center}
    \begin{prooftree}[small]
        \hypo{\vdashJK \text{\pr|C|} : M}
        \infer[left label={\(\forall i, M_{ii}^* = m\) and \(\forall i, j, M^*_{ij} \neq p\)}]1[W]{\vdashJK \text{\pr|while b do  \{C\}|} : M^*}
    \end{prooftree}
    \end{center}
\end{centering}

    \caption{\mwp-bounds flow analysis inference rules}
    \label{fig:orig-rules}
\end{figure*}

