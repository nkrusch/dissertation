\subsection{Static Program Analysis}\label{subsec:static}

Static program analysis aims to automatically answer questions about the possible behaviors of programs.
Static analysis concerned with analysis semantic properties (not syntactic, lexical, testing, etc.)
Analogous to hierarchy of “languages”, static analysis is concerned with what is called context-sensitive languages or context-sensitive grammars, but not bothering to deal with regular languages and context-free languages (those are for lexers and parsers).

Foundational approaches include: basic type analysis, lattice theory, control flow graphs, dataflow analysis, fixed-point algorithms, widening and narrowing, path sensitivity, relational analysis, inter-procedural analysis, context sensitivity, control flow analysis, several flavors of pointer analysis, and abstract interpretation.

It is impossible to construct an ideal program correctness analyzer~\cite[p. 15]{moller2023}

notes/approx.png is from
% Course Script "Static analysis and all that" IN5440 / autumn 2020  by Martin Steffen
% url: https://www.uio.no/studier/emner/matnat/ifi/IN5440/h20/script/intro-script.pdf

\subsubsection{Comment on Rice's Theorem}

At first, [Rice's theorem]~\cite{rice1953} seems like a discouraging result, however, this theoretical result does not prevent approximative answers.

